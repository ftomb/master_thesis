% !TEX root = ../main.tex


\chapter{Feature Extraction}\label{chap:feature-extraction}

 In this section, I will describe how feature extraction is carried out in the proposed methodology.
 
This section is divided into three parts.
In the first part, I will discuss the two most commonly used approaches for sampling speech data, and I will present the adaptive sampling method implemented for my methodology.
In the second part, I will provide a description of the textual information that I decided to include and justification for each chosen linguistic feature.
In the third part, I will present the technique I used for the dynamic encoding of the interpolated \ac{F0}.

\section{Sampling}

One major preliminary step in approaching the problem of \ac{F0} modeling is to decide how the pitch information and the corresponding linguistic labels should be sampled.
Most \ac{F0} estimation tools produce as their output a sequence of \ac{F0} values sampled at a constant rate.
Typically, for most \ac{TTS} systems, acoustic information is sampled at a 5~ms time interval.
However, different researchers might decide to use this data in very different ways depending on how they approach the problem.

In this section, I will present the sampling approach used in my implementation.
The proposed sampling scheme is based on the selection of a support level and a default time interval. 
The size of the interval is adapted based on the duration of the support level.
This approach is a hybrid of two common sampling approaches: frame-by-frame and anchor-point methods.
The advantages and disadvantages of these methods are discussed and used as motivation for the adaptive sampling rate method presented here.

\subsection{Frame-by-Frame Approach}

Under this approach, the data is sampled at a fixed time interval, typically at a 5~ms time interval.
This is what we typically find in \ac{TTS} systems such as Merlin \citep{Wu2016Merlin}, where the \ac{F0} and other acoustic features are modeled jointly in a frame-by-frame fashion. 

The main advantage of frame-by-frame approaches is that they are very convenient and they make the least assumptions about the data. 
However, it is also needlessly wasteful, as we would have to make a prediction for every single frame.
This can be problematic, if we wish to include very rich input features such as word embeddings.
Under this approach, using word embeddings with hundreds (if not thousands) of inputs per frame could become prohibitively expensive.

As a lot of the information contained in the \ac{F0} contour is redundant, most of it can be discarded, as it is easily reconstructed from partial or parametrized contours by means of interpolation.
Not only does this allow for larger inputs, but having fewer points to predict also means faster and more efficient training, especially for longer sequences.

Despite the recent introduction of \acp{RNN}, which allow for much better modeling of long sequences, long distance dependencies still remain a tricky aspect to model, because of the limited time memory that these architectures offers.
Reducing the number of intonational events that we want to predict per utterance will make the learning process much easier.

However, we cannot simply compress the data by naively downsampling the output produced by \ac{F0} estimation tools. 
The main problem with a lower fixed sampling rate is that the sampled points would be located in linguistically irrelevant positions.
For instance some points might be located very close to the syllable boundaries, sometimes far from them. 
Sometimes certain syllables might be skipped entirely because they are too short.
Notice for instance in \autoref{fig:fixed-anchor}, how a fixed sampling rate causes the sampled \ac{F0} to be distributed erratically with respect to linguistic segments.



\begin{figure}[h]
\centering
\resizebox{\textwidth}{!}{%% Creator: Matplotlib, PGF backend
%%
%% To include the figure in your LaTeX document, write
%%   \input{<filename>.pgf}
%%
%% Make sure the required packages are loaded in your preamble
%%   \usepackage{pgf}
%%
%% Figures using additional raster images can only be included by \input if
%% they are in the same directory as the main LaTeX file. For loading figures
%% from other directories you can use the `import` package
%%   \usepackage{import}
%% and then include the figures with
%%   \import{<path to file>}{<filename>.pgf}
%%
%% Matplotlib used the following preamble
%%   \usepackage[utf8x]{inputenc}
%%   \usepackage[T1]{fontenc}
%%   \usepackage{cmbright}
%%
\begingroup%
\makeatletter%
\begin{pgfpicture}%
\pgfpathrectangle{\pgfpointorigin}{\pgfqpoint{8.000000in}{4.000000in}}%
\pgfusepath{use as bounding box, clip}%
\begin{pgfscope}%
\pgfsetbuttcap%
\pgfsetmiterjoin%
\definecolor{currentfill}{rgb}{1.000000,1.000000,1.000000}%
\pgfsetfillcolor{currentfill}%
\pgfsetlinewidth{0.000000pt}%
\definecolor{currentstroke}{rgb}{1.000000,1.000000,1.000000}%
\pgfsetstrokecolor{currentstroke}%
\pgfsetdash{}{0pt}%
\pgfpathmoveto{\pgfqpoint{0.000000in}{0.000000in}}%
\pgfpathlineto{\pgfqpoint{8.000000in}{0.000000in}}%
\pgfpathlineto{\pgfqpoint{8.000000in}{4.000000in}}%
\pgfpathlineto{\pgfqpoint{0.000000in}{4.000000in}}%
\pgfpathclose%
\pgfusepath{fill}%
\end{pgfscope}%
\begin{pgfscope}%
\pgfsetbuttcap%
\pgfsetmiterjoin%
\definecolor{currentfill}{rgb}{1.000000,1.000000,1.000000}%
\pgfsetfillcolor{currentfill}%
\pgfsetlinewidth{0.000000pt}%
\definecolor{currentstroke}{rgb}{0.000000,0.000000,0.000000}%
\pgfsetstrokecolor{currentstroke}%
\pgfsetstrokeopacity{0.000000}%
\pgfsetdash{}{0pt}%
\pgfpathmoveto{\pgfqpoint{1.000000in}{0.440000in}}%
\pgfpathlineto{\pgfqpoint{7.200000in}{0.440000in}}%
\pgfpathlineto{\pgfqpoint{7.200000in}{3.520000in}}%
\pgfpathlineto{\pgfqpoint{1.000000in}{3.520000in}}%
\pgfpathclose%
\pgfusepath{fill}%
\end{pgfscope}%
\begin{pgfscope}%
\pgfsetbuttcap%
\pgfsetroundjoin%
\definecolor{currentfill}{rgb}{0.000000,0.000000,0.000000}%
\pgfsetfillcolor{currentfill}%
\pgfsetlinewidth{0.803000pt}%
\definecolor{currentstroke}{rgb}{0.000000,0.000000,0.000000}%
\pgfsetstrokecolor{currentstroke}%
\pgfsetdash{}{0pt}%
\pgfsys@defobject{currentmarker}{\pgfqpoint{0.000000in}{-0.048611in}}{\pgfqpoint{0.000000in}{0.000000in}}{%
\pgfpathmoveto{\pgfqpoint{0.000000in}{0.000000in}}%
\pgfpathlineto{\pgfqpoint{0.000000in}{-0.048611in}}%
\pgfusepath{stroke,fill}%
}%
\begin{pgfscope}%
\pgfsys@transformshift{1.281818in}{0.440000in}%
\pgfsys@useobject{currentmarker}{}%
\end{pgfscope}%
\end{pgfscope}%
\begin{pgfscope}%
\pgftext[x=1.281818in,y=0.342778in,,top]{\rmfamily\fontsize{10.000000}{12.000000}\selectfont \(\displaystyle 0.00\)}%
\end{pgfscope}%
\begin{pgfscope}%
\pgfsetbuttcap%
\pgfsetroundjoin%
\definecolor{currentfill}{rgb}{0.000000,0.000000,0.000000}%
\pgfsetfillcolor{currentfill}%
\pgfsetlinewidth{0.803000pt}%
\definecolor{currentstroke}{rgb}{0.000000,0.000000,0.000000}%
\pgfsetstrokecolor{currentstroke}%
\pgfsetdash{}{0pt}%
\pgfsys@defobject{currentmarker}{\pgfqpoint{0.000000in}{-0.048611in}}{\pgfqpoint{0.000000in}{0.000000in}}{%
\pgfpathmoveto{\pgfqpoint{0.000000in}{0.000000in}}%
\pgfpathlineto{\pgfqpoint{0.000000in}{-0.048611in}}%
\pgfusepath{stroke,fill}%
}%
\begin{pgfscope}%
\pgfsys@transformshift{2.043489in}{0.440000in}%
\pgfsys@useobject{currentmarker}{}%
\end{pgfscope}%
\end{pgfscope}%
\begin{pgfscope}%
\pgftext[x=2.043489in,y=0.342778in,,top]{\rmfamily\fontsize{10.000000}{12.000000}\selectfont \(\displaystyle 0.25\)}%
\end{pgfscope}%
\begin{pgfscope}%
\pgfsetbuttcap%
\pgfsetroundjoin%
\definecolor{currentfill}{rgb}{0.000000,0.000000,0.000000}%
\pgfsetfillcolor{currentfill}%
\pgfsetlinewidth{0.803000pt}%
\definecolor{currentstroke}{rgb}{0.000000,0.000000,0.000000}%
\pgfsetstrokecolor{currentstroke}%
\pgfsetdash{}{0pt}%
\pgfsys@defobject{currentmarker}{\pgfqpoint{0.000000in}{-0.048611in}}{\pgfqpoint{0.000000in}{0.000000in}}{%
\pgfpathmoveto{\pgfqpoint{0.000000in}{0.000000in}}%
\pgfpathlineto{\pgfqpoint{0.000000in}{-0.048611in}}%
\pgfusepath{stroke,fill}%
}%
\begin{pgfscope}%
\pgfsys@transformshift{2.805160in}{0.440000in}%
\pgfsys@useobject{currentmarker}{}%
\end{pgfscope}%
\end{pgfscope}%
\begin{pgfscope}%
\pgftext[x=2.805160in,y=0.342778in,,top]{\rmfamily\fontsize{10.000000}{12.000000}\selectfont \(\displaystyle 0.50\)}%
\end{pgfscope}%
\begin{pgfscope}%
\pgfsetbuttcap%
\pgfsetroundjoin%
\definecolor{currentfill}{rgb}{0.000000,0.000000,0.000000}%
\pgfsetfillcolor{currentfill}%
\pgfsetlinewidth{0.803000pt}%
\definecolor{currentstroke}{rgb}{0.000000,0.000000,0.000000}%
\pgfsetstrokecolor{currentstroke}%
\pgfsetdash{}{0pt}%
\pgfsys@defobject{currentmarker}{\pgfqpoint{0.000000in}{-0.048611in}}{\pgfqpoint{0.000000in}{0.000000in}}{%
\pgfpathmoveto{\pgfqpoint{0.000000in}{0.000000in}}%
\pgfpathlineto{\pgfqpoint{0.000000in}{-0.048611in}}%
\pgfusepath{stroke,fill}%
}%
\begin{pgfscope}%
\pgfsys@transformshift{3.566830in}{0.440000in}%
\pgfsys@useobject{currentmarker}{}%
\end{pgfscope}%
\end{pgfscope}%
\begin{pgfscope}%
\pgftext[x=3.566830in,y=0.342778in,,top]{\rmfamily\fontsize{10.000000}{12.000000}\selectfont \(\displaystyle 0.75\)}%
\end{pgfscope}%
\begin{pgfscope}%
\pgfsetbuttcap%
\pgfsetroundjoin%
\definecolor{currentfill}{rgb}{0.000000,0.000000,0.000000}%
\pgfsetfillcolor{currentfill}%
\pgfsetlinewidth{0.803000pt}%
\definecolor{currentstroke}{rgb}{0.000000,0.000000,0.000000}%
\pgfsetstrokecolor{currentstroke}%
\pgfsetdash{}{0pt}%
\pgfsys@defobject{currentmarker}{\pgfqpoint{0.000000in}{-0.048611in}}{\pgfqpoint{0.000000in}{0.000000in}}{%
\pgfpathmoveto{\pgfqpoint{0.000000in}{0.000000in}}%
\pgfpathlineto{\pgfqpoint{0.000000in}{-0.048611in}}%
\pgfusepath{stroke,fill}%
}%
\begin{pgfscope}%
\pgfsys@transformshift{4.328501in}{0.440000in}%
\pgfsys@useobject{currentmarker}{}%
\end{pgfscope}%
\end{pgfscope}%
\begin{pgfscope}%
\pgftext[x=4.328501in,y=0.342778in,,top]{\rmfamily\fontsize{10.000000}{12.000000}\selectfont \(\displaystyle 1.00\)}%
\end{pgfscope}%
\begin{pgfscope}%
\pgfsetbuttcap%
\pgfsetroundjoin%
\definecolor{currentfill}{rgb}{0.000000,0.000000,0.000000}%
\pgfsetfillcolor{currentfill}%
\pgfsetlinewidth{0.803000pt}%
\definecolor{currentstroke}{rgb}{0.000000,0.000000,0.000000}%
\pgfsetstrokecolor{currentstroke}%
\pgfsetdash{}{0pt}%
\pgfsys@defobject{currentmarker}{\pgfqpoint{0.000000in}{-0.048611in}}{\pgfqpoint{0.000000in}{0.000000in}}{%
\pgfpathmoveto{\pgfqpoint{0.000000in}{0.000000in}}%
\pgfpathlineto{\pgfqpoint{0.000000in}{-0.048611in}}%
\pgfusepath{stroke,fill}%
}%
\begin{pgfscope}%
\pgfsys@transformshift{5.090172in}{0.440000in}%
\pgfsys@useobject{currentmarker}{}%
\end{pgfscope}%
\end{pgfscope}%
\begin{pgfscope}%
\pgftext[x=5.090172in,y=0.342778in,,top]{\rmfamily\fontsize{10.000000}{12.000000}\selectfont \(\displaystyle 1.25\)}%
\end{pgfscope}%
\begin{pgfscope}%
\pgfsetbuttcap%
\pgfsetroundjoin%
\definecolor{currentfill}{rgb}{0.000000,0.000000,0.000000}%
\pgfsetfillcolor{currentfill}%
\pgfsetlinewidth{0.803000pt}%
\definecolor{currentstroke}{rgb}{0.000000,0.000000,0.000000}%
\pgfsetstrokecolor{currentstroke}%
\pgfsetdash{}{0pt}%
\pgfsys@defobject{currentmarker}{\pgfqpoint{0.000000in}{-0.048611in}}{\pgfqpoint{0.000000in}{0.000000in}}{%
\pgfpathmoveto{\pgfqpoint{0.000000in}{0.000000in}}%
\pgfpathlineto{\pgfqpoint{0.000000in}{-0.048611in}}%
\pgfusepath{stroke,fill}%
}%
\begin{pgfscope}%
\pgfsys@transformshift{5.851843in}{0.440000in}%
\pgfsys@useobject{currentmarker}{}%
\end{pgfscope}%
\end{pgfscope}%
\begin{pgfscope}%
\pgftext[x=5.851843in,y=0.342778in,,top]{\rmfamily\fontsize{10.000000}{12.000000}\selectfont \(\displaystyle 1.50\)}%
\end{pgfscope}%
\begin{pgfscope}%
\pgfsetbuttcap%
\pgfsetroundjoin%
\definecolor{currentfill}{rgb}{0.000000,0.000000,0.000000}%
\pgfsetfillcolor{currentfill}%
\pgfsetlinewidth{0.803000pt}%
\definecolor{currentstroke}{rgb}{0.000000,0.000000,0.000000}%
\pgfsetstrokecolor{currentstroke}%
\pgfsetdash{}{0pt}%
\pgfsys@defobject{currentmarker}{\pgfqpoint{0.000000in}{-0.048611in}}{\pgfqpoint{0.000000in}{0.000000in}}{%
\pgfpathmoveto{\pgfqpoint{0.000000in}{0.000000in}}%
\pgfpathlineto{\pgfqpoint{0.000000in}{-0.048611in}}%
\pgfusepath{stroke,fill}%
}%
\begin{pgfscope}%
\pgfsys@transformshift{6.613514in}{0.440000in}%
\pgfsys@useobject{currentmarker}{}%
\end{pgfscope}%
\end{pgfscope}%
\begin{pgfscope}%
\pgftext[x=6.613514in,y=0.342778in,,top]{\rmfamily\fontsize{10.000000}{12.000000}\selectfont \(\displaystyle 1.75\)}%
\end{pgfscope}%
\begin{pgfscope}%
\pgftext[x=4.100000in,y=0.164567in,,top]{\rmfamily\fontsize{10.000000}{12.000000}\selectfont Time (s)}%
\end{pgfscope}%
\begin{pgfscope}%
\pgfsetbuttcap%
\pgfsetroundjoin%
\definecolor{currentfill}{rgb}{0.000000,0.000000,0.000000}%
\pgfsetfillcolor{currentfill}%
\pgfsetlinewidth{0.803000pt}%
\definecolor{currentstroke}{rgb}{0.000000,0.000000,0.000000}%
\pgfsetstrokecolor{currentstroke}%
\pgfsetdash{}{0pt}%
\pgfsys@defobject{currentmarker}{\pgfqpoint{-0.048611in}{0.000000in}}{\pgfqpoint{0.000000in}{0.000000in}}{%
\pgfpathmoveto{\pgfqpoint{0.000000in}{0.000000in}}%
\pgfpathlineto{\pgfqpoint{-0.048611in}{0.000000in}}%
\pgfusepath{stroke,fill}%
}%
\begin{pgfscope}%
\pgfsys@transformshift{1.000000in}{0.747896in}%
\pgfsys@useobject{currentmarker}{}%
\end{pgfscope}%
\end{pgfscope}%
\begin{pgfscope}%
\pgftext[x=0.684030in,y=0.700068in,left,base]{\rmfamily\fontsize{10.000000}{12.000000}\selectfont \(\displaystyle 140\)}%
\end{pgfscope}%
\begin{pgfscope}%
\pgfsetbuttcap%
\pgfsetroundjoin%
\definecolor{currentfill}{rgb}{0.000000,0.000000,0.000000}%
\pgfsetfillcolor{currentfill}%
\pgfsetlinewidth{0.803000pt}%
\definecolor{currentstroke}{rgb}{0.000000,0.000000,0.000000}%
\pgfsetstrokecolor{currentstroke}%
\pgfsetdash{}{0pt}%
\pgfsys@defobject{currentmarker}{\pgfqpoint{-0.048611in}{0.000000in}}{\pgfqpoint{0.000000in}{0.000000in}}{%
\pgfpathmoveto{\pgfqpoint{0.000000in}{0.000000in}}%
\pgfpathlineto{\pgfqpoint{-0.048611in}{0.000000in}}%
\pgfusepath{stroke,fill}%
}%
\begin{pgfscope}%
\pgfsys@transformshift{1.000000in}{1.175589in}%
\pgfsys@useobject{currentmarker}{}%
\end{pgfscope}%
\end{pgfscope}%
\begin{pgfscope}%
\pgftext[x=0.684030in,y=1.127761in,left,base]{\rmfamily\fontsize{10.000000}{12.000000}\selectfont \(\displaystyle 160\)}%
\end{pgfscope}%
\begin{pgfscope}%
\pgfsetbuttcap%
\pgfsetroundjoin%
\definecolor{currentfill}{rgb}{0.000000,0.000000,0.000000}%
\pgfsetfillcolor{currentfill}%
\pgfsetlinewidth{0.803000pt}%
\definecolor{currentstroke}{rgb}{0.000000,0.000000,0.000000}%
\pgfsetstrokecolor{currentstroke}%
\pgfsetdash{}{0pt}%
\pgfsys@defobject{currentmarker}{\pgfqpoint{-0.048611in}{0.000000in}}{\pgfqpoint{0.000000in}{0.000000in}}{%
\pgfpathmoveto{\pgfqpoint{0.000000in}{0.000000in}}%
\pgfpathlineto{\pgfqpoint{-0.048611in}{0.000000in}}%
\pgfusepath{stroke,fill}%
}%
\begin{pgfscope}%
\pgfsys@transformshift{1.000000in}{1.603282in}%
\pgfsys@useobject{currentmarker}{}%
\end{pgfscope}%
\end{pgfscope}%
\begin{pgfscope}%
\pgftext[x=0.684030in,y=1.555454in,left,base]{\rmfamily\fontsize{10.000000}{12.000000}\selectfont \(\displaystyle 180\)}%
\end{pgfscope}%
\begin{pgfscope}%
\pgfsetbuttcap%
\pgfsetroundjoin%
\definecolor{currentfill}{rgb}{0.000000,0.000000,0.000000}%
\pgfsetfillcolor{currentfill}%
\pgfsetlinewidth{0.803000pt}%
\definecolor{currentstroke}{rgb}{0.000000,0.000000,0.000000}%
\pgfsetstrokecolor{currentstroke}%
\pgfsetdash{}{0pt}%
\pgfsys@defobject{currentmarker}{\pgfqpoint{-0.048611in}{0.000000in}}{\pgfqpoint{0.000000in}{0.000000in}}{%
\pgfpathmoveto{\pgfqpoint{0.000000in}{0.000000in}}%
\pgfpathlineto{\pgfqpoint{-0.048611in}{0.000000in}}%
\pgfusepath{stroke,fill}%
}%
\begin{pgfscope}%
\pgfsys@transformshift{1.000000in}{2.030975in}%
\pgfsys@useobject{currentmarker}{}%
\end{pgfscope}%
\end{pgfscope}%
\begin{pgfscope}%
\pgftext[x=0.684030in,y=1.983147in,left,base]{\rmfamily\fontsize{10.000000}{12.000000}\selectfont \(\displaystyle 200\)}%
\end{pgfscope}%
\begin{pgfscope}%
\pgfsetbuttcap%
\pgfsetroundjoin%
\definecolor{currentfill}{rgb}{0.000000,0.000000,0.000000}%
\pgfsetfillcolor{currentfill}%
\pgfsetlinewidth{0.803000pt}%
\definecolor{currentstroke}{rgb}{0.000000,0.000000,0.000000}%
\pgfsetstrokecolor{currentstroke}%
\pgfsetdash{}{0pt}%
\pgfsys@defobject{currentmarker}{\pgfqpoint{-0.048611in}{0.000000in}}{\pgfqpoint{0.000000in}{0.000000in}}{%
\pgfpathmoveto{\pgfqpoint{0.000000in}{0.000000in}}%
\pgfpathlineto{\pgfqpoint{-0.048611in}{0.000000in}}%
\pgfusepath{stroke,fill}%
}%
\begin{pgfscope}%
\pgfsys@transformshift{1.000000in}{2.458668in}%
\pgfsys@useobject{currentmarker}{}%
\end{pgfscope}%
\end{pgfscope}%
\begin{pgfscope}%
\pgftext[x=0.684030in,y=2.410840in,left,base]{\rmfamily\fontsize{10.000000}{12.000000}\selectfont \(\displaystyle 220\)}%
\end{pgfscope}%
\begin{pgfscope}%
\pgfsetbuttcap%
\pgfsetroundjoin%
\definecolor{currentfill}{rgb}{0.000000,0.000000,0.000000}%
\pgfsetfillcolor{currentfill}%
\pgfsetlinewidth{0.803000pt}%
\definecolor{currentstroke}{rgb}{0.000000,0.000000,0.000000}%
\pgfsetstrokecolor{currentstroke}%
\pgfsetdash{}{0pt}%
\pgfsys@defobject{currentmarker}{\pgfqpoint{-0.048611in}{0.000000in}}{\pgfqpoint{0.000000in}{0.000000in}}{%
\pgfpathmoveto{\pgfqpoint{0.000000in}{0.000000in}}%
\pgfpathlineto{\pgfqpoint{-0.048611in}{0.000000in}}%
\pgfusepath{stroke,fill}%
}%
\begin{pgfscope}%
\pgfsys@transformshift{1.000000in}{2.886361in}%
\pgfsys@useobject{currentmarker}{}%
\end{pgfscope}%
\end{pgfscope}%
\begin{pgfscope}%
\pgftext[x=0.684030in,y=2.838533in,left,base]{\rmfamily\fontsize{10.000000}{12.000000}\selectfont \(\displaystyle 240\)}%
\end{pgfscope}%
\begin{pgfscope}%
\pgfsetbuttcap%
\pgfsetroundjoin%
\definecolor{currentfill}{rgb}{0.000000,0.000000,0.000000}%
\pgfsetfillcolor{currentfill}%
\pgfsetlinewidth{0.803000pt}%
\definecolor{currentstroke}{rgb}{0.000000,0.000000,0.000000}%
\pgfsetstrokecolor{currentstroke}%
\pgfsetdash{}{0pt}%
\pgfsys@defobject{currentmarker}{\pgfqpoint{-0.048611in}{0.000000in}}{\pgfqpoint{0.000000in}{0.000000in}}{%
\pgfpathmoveto{\pgfqpoint{0.000000in}{0.000000in}}%
\pgfpathlineto{\pgfqpoint{-0.048611in}{0.000000in}}%
\pgfusepath{stroke,fill}%
}%
\begin{pgfscope}%
\pgfsys@transformshift{1.000000in}{3.314054in}%
\pgfsys@useobject{currentmarker}{}%
\end{pgfscope}%
\end{pgfscope}%
\begin{pgfscope}%
\pgftext[x=0.684030in,y=3.266226in,left,base]{\rmfamily\fontsize{10.000000}{12.000000}\selectfont \(\displaystyle 260\)}%
\end{pgfscope}%
\begin{pgfscope}%
\pgftext[x=0.628474in,y=1.980000in,,bottom,rotate=90.000000]{\rmfamily\fontsize{10.000000}{12.000000}\selectfont Frequency (Hz)}%
\end{pgfscope}%
\begin{pgfscope}%
\pgfpathrectangle{\pgfqpoint{1.000000in}{0.440000in}}{\pgfqpoint{6.200000in}{3.080000in}} %
\pgfusepath{clip}%
\pgfsetrectcap%
\pgfsetroundjoin%
\pgfsetlinewidth{0.501875pt}%
\definecolor{currentstroke}{rgb}{0.800000,0.000000,0.266667}%
\pgfsetstrokecolor{currentstroke}%
\pgfsetdash{}{0pt}%
\pgfpathmoveto{\pgfqpoint{1.769287in}{0.440000in}}%
\pgfpathlineto{\pgfqpoint{1.769287in}{3.520000in}}%
\pgfusepath{stroke}%
\end{pgfscope}%
\begin{pgfscope}%
\pgfpathrectangle{\pgfqpoint{1.000000in}{0.440000in}}{\pgfqpoint{6.200000in}{3.080000in}} %
\pgfusepath{clip}%
\pgfsetrectcap%
\pgfsetroundjoin%
\pgfsetlinewidth{0.501875pt}%
\definecolor{currentstroke}{rgb}{0.800000,0.000000,0.266667}%
\pgfsetstrokecolor{currentstroke}%
\pgfsetdash{}{0pt}%
\pgfpathmoveto{\pgfqpoint{2.165356in}{0.440000in}}%
\pgfpathlineto{\pgfqpoint{2.165356in}{3.520000in}}%
\pgfusepath{stroke}%
\end{pgfscope}%
\begin{pgfscope}%
\pgfpathrectangle{\pgfqpoint{1.000000in}{0.440000in}}{\pgfqpoint{6.200000in}{3.080000in}} %
\pgfusepath{clip}%
\pgfsetrectcap%
\pgfsetroundjoin%
\pgfsetlinewidth{0.501875pt}%
\definecolor{currentstroke}{rgb}{0.800000,0.000000,0.266667}%
\pgfsetstrokecolor{currentstroke}%
\pgfsetdash{}{0pt}%
\pgfpathmoveto{\pgfqpoint{2.165356in}{0.440000in}}%
\pgfpathlineto{\pgfqpoint{2.165356in}{3.520000in}}%
\pgfusepath{stroke}%
\end{pgfscope}%
\begin{pgfscope}%
\pgfpathrectangle{\pgfqpoint{1.000000in}{0.440000in}}{\pgfqpoint{6.200000in}{3.080000in}} %
\pgfusepath{clip}%
\pgfsetrectcap%
\pgfsetroundjoin%
\pgfsetlinewidth{0.501875pt}%
\definecolor{currentstroke}{rgb}{0.800000,0.000000,0.266667}%
\pgfsetstrokecolor{currentstroke}%
\pgfsetdash{}{0pt}%
\pgfpathmoveto{\pgfqpoint{2.622359in}{0.440000in}}%
\pgfpathlineto{\pgfqpoint{2.622359in}{3.520000in}}%
\pgfusepath{stroke}%
\end{pgfscope}%
\begin{pgfscope}%
\pgfpathrectangle{\pgfqpoint{1.000000in}{0.440000in}}{\pgfqpoint{6.200000in}{3.080000in}} %
\pgfusepath{clip}%
\pgfsetrectcap%
\pgfsetroundjoin%
\pgfsetlinewidth{0.501875pt}%
\definecolor{currentstroke}{rgb}{0.800000,0.000000,0.266667}%
\pgfsetstrokecolor{currentstroke}%
\pgfsetdash{}{0pt}%
\pgfpathmoveto{\pgfqpoint{2.622359in}{0.440000in}}%
\pgfpathlineto{\pgfqpoint{2.622359in}{3.520000in}}%
\pgfusepath{stroke}%
\end{pgfscope}%
\begin{pgfscope}%
\pgfpathrectangle{\pgfqpoint{1.000000in}{0.440000in}}{\pgfqpoint{6.200000in}{3.080000in}} %
\pgfusepath{clip}%
\pgfsetrectcap%
\pgfsetroundjoin%
\pgfsetlinewidth{0.501875pt}%
\definecolor{currentstroke}{rgb}{0.800000,0.000000,0.266667}%
\pgfsetstrokecolor{currentstroke}%
\pgfsetdash{}{0pt}%
\pgfpathmoveto{\pgfqpoint{3.444963in}{0.440000in}}%
\pgfpathlineto{\pgfqpoint{3.444963in}{3.520000in}}%
\pgfusepath{stroke}%
\end{pgfscope}%
\begin{pgfscope}%
\pgfpathrectangle{\pgfqpoint{1.000000in}{0.440000in}}{\pgfqpoint{6.200000in}{3.080000in}} %
\pgfusepath{clip}%
\pgfsetrectcap%
\pgfsetroundjoin%
\pgfsetlinewidth{0.501875pt}%
\definecolor{currentstroke}{rgb}{0.800000,0.000000,0.266667}%
\pgfsetstrokecolor{currentstroke}%
\pgfsetdash{}{0pt}%
\pgfpathmoveto{\pgfqpoint{3.444963in}{0.440000in}}%
\pgfpathlineto{\pgfqpoint{3.444963in}{3.520000in}}%
\pgfusepath{stroke}%
\end{pgfscope}%
\begin{pgfscope}%
\pgfpathrectangle{\pgfqpoint{1.000000in}{0.440000in}}{\pgfqpoint{6.200000in}{3.080000in}} %
\pgfusepath{clip}%
\pgfsetrectcap%
\pgfsetroundjoin%
\pgfsetlinewidth{0.501875pt}%
\definecolor{currentstroke}{rgb}{0.800000,0.000000,0.266667}%
\pgfsetstrokecolor{currentstroke}%
\pgfsetdash{}{0pt}%
\pgfpathmoveto{\pgfqpoint{3.901966in}{0.440000in}}%
\pgfpathlineto{\pgfqpoint{3.901966in}{3.520000in}}%
\pgfusepath{stroke}%
\end{pgfscope}%
\begin{pgfscope}%
\pgfpathrectangle{\pgfqpoint{1.000000in}{0.440000in}}{\pgfqpoint{6.200000in}{3.080000in}} %
\pgfusepath{clip}%
\pgfsetrectcap%
\pgfsetroundjoin%
\pgfsetlinewidth{0.501875pt}%
\definecolor{currentstroke}{rgb}{0.800000,0.000000,0.266667}%
\pgfsetstrokecolor{currentstroke}%
\pgfsetdash{}{0pt}%
\pgfpathmoveto{\pgfqpoint{3.901966in}{0.440000in}}%
\pgfpathlineto{\pgfqpoint{3.901966in}{3.520000in}}%
\pgfusepath{stroke}%
\end{pgfscope}%
\begin{pgfscope}%
\pgfpathrectangle{\pgfqpoint{1.000000in}{0.440000in}}{\pgfqpoint{6.200000in}{3.080000in}} %
\pgfusepath{clip}%
\pgfsetrectcap%
\pgfsetroundjoin%
\pgfsetlinewidth{0.501875pt}%
\definecolor{currentstroke}{rgb}{0.800000,0.000000,0.266667}%
\pgfsetstrokecolor{currentstroke}%
\pgfsetdash{}{0pt}%
\pgfpathmoveto{\pgfqpoint{5.059705in}{0.440000in}}%
\pgfpathlineto{\pgfqpoint{5.059705in}{3.520000in}}%
\pgfusepath{stroke}%
\end{pgfscope}%
\begin{pgfscope}%
\pgfpathrectangle{\pgfqpoint{1.000000in}{0.440000in}}{\pgfqpoint{6.200000in}{3.080000in}} %
\pgfusepath{clip}%
\pgfsetrectcap%
\pgfsetroundjoin%
\pgfsetlinewidth{0.501875pt}%
\definecolor{currentstroke}{rgb}{0.800000,0.000000,0.266667}%
\pgfsetstrokecolor{currentstroke}%
\pgfsetdash{}{0pt}%
\pgfpathmoveto{\pgfqpoint{5.059705in}{0.440000in}}%
\pgfpathlineto{\pgfqpoint{5.059705in}{3.520000in}}%
\pgfusepath{stroke}%
\end{pgfscope}%
\begin{pgfscope}%
\pgfpathrectangle{\pgfqpoint{1.000000in}{0.440000in}}{\pgfqpoint{6.200000in}{3.080000in}} %
\pgfusepath{clip}%
\pgfsetrectcap%
\pgfsetroundjoin%
\pgfsetlinewidth{0.501875pt}%
\definecolor{currentstroke}{rgb}{0.800000,0.000000,0.266667}%
\pgfsetstrokecolor{currentstroke}%
\pgfsetdash{}{0pt}%
\pgfpathmoveto{\pgfqpoint{6.765848in}{0.440000in}}%
\pgfpathlineto{\pgfqpoint{6.765848in}{3.520000in}}%
\pgfusepath{stroke}%
\end{pgfscope}%
\begin{pgfscope}%
\pgfpathrectangle{\pgfqpoint{1.000000in}{0.440000in}}{\pgfqpoint{6.200000in}{3.080000in}} %
\pgfusepath{clip}%
\pgfsetrectcap%
\pgfsetroundjoin%
\pgfsetlinewidth{1.505625pt}%
\definecolor{currentstroke}{rgb}{0.364706,0.517647,0.572549}%
\pgfsetstrokecolor{currentstroke}%
\pgfsetdash{}{0pt}%
\pgfpathmoveto{\pgfqpoint{1.281818in}{1.169484in}}%
\pgfpathlineto{\pgfqpoint{1.997789in}{1.169484in}}%
\pgfpathlineto{\pgfqpoint{2.013022in}{1.180018in}}%
\pgfpathlineto{\pgfqpoint{2.028256in}{1.213117in}}%
\pgfpathlineto{\pgfqpoint{2.043489in}{1.243508in}}%
\pgfpathlineto{\pgfqpoint{2.058722in}{1.249626in}}%
\pgfpathlineto{\pgfqpoint{2.073956in}{1.241922in}}%
\pgfpathlineto{\pgfqpoint{2.104423in}{1.216067in}}%
\pgfpathlineto{\pgfqpoint{2.119656in}{1.180239in}}%
\pgfpathlineto{\pgfqpoint{2.134889in}{1.096283in}}%
\pgfpathlineto{\pgfqpoint{2.150123in}{1.000565in}}%
\pgfpathlineto{\pgfqpoint{2.165356in}{0.981155in}}%
\pgfpathlineto{\pgfqpoint{2.180590in}{1.022198in}}%
\pgfpathlineto{\pgfqpoint{2.195823in}{1.051990in}}%
\pgfpathlineto{\pgfqpoint{2.211057in}{1.073655in}}%
\pgfpathlineto{\pgfqpoint{2.226290in}{1.117670in}}%
\pgfpathlineto{\pgfqpoint{2.241523in}{1.190799in}}%
\pgfpathlineto{\pgfqpoint{2.256757in}{1.283890in}}%
\pgfpathlineto{\pgfqpoint{2.287224in}{1.477771in}}%
\pgfpathlineto{\pgfqpoint{2.302457in}{1.553343in}}%
\pgfpathlineto{\pgfqpoint{2.317690in}{1.610443in}}%
\pgfpathlineto{\pgfqpoint{2.332924in}{1.660888in}}%
\pgfpathlineto{\pgfqpoint{2.348157in}{1.726165in}}%
\pgfpathlineto{\pgfqpoint{2.363391in}{1.855295in}}%
\pgfpathlineto{\pgfqpoint{2.378624in}{2.077817in}}%
\pgfpathlineto{\pgfqpoint{2.393857in}{2.280446in}}%
\pgfpathlineto{\pgfqpoint{2.409091in}{2.393129in}}%
\pgfpathlineto{\pgfqpoint{2.439558in}{2.525077in}}%
\pgfpathlineto{\pgfqpoint{2.454791in}{2.620606in}}%
\pgfpathlineto{\pgfqpoint{2.470025in}{2.727577in}}%
\pgfpathlineto{\pgfqpoint{2.485258in}{2.823790in}}%
\pgfpathlineto{\pgfqpoint{2.500491in}{2.910174in}}%
\pgfpathlineto{\pgfqpoint{2.530958in}{3.061822in}}%
\pgfpathlineto{\pgfqpoint{2.561425in}{3.243144in}}%
\pgfpathlineto{\pgfqpoint{2.576658in}{3.315514in}}%
\pgfpathlineto{\pgfqpoint{2.591892in}{3.372327in}}%
\pgfpathlineto{\pgfqpoint{2.607125in}{3.380000in}}%
\pgfpathlineto{\pgfqpoint{2.622359in}{3.283246in}}%
\pgfpathlineto{\pgfqpoint{2.637592in}{3.165543in}}%
\pgfpathlineto{\pgfqpoint{2.652826in}{3.121637in}}%
\pgfpathlineto{\pgfqpoint{2.698526in}{3.099111in}}%
\pgfpathlineto{\pgfqpoint{2.713759in}{3.086715in}}%
\pgfpathlineto{\pgfqpoint{2.774693in}{2.995830in}}%
\pgfpathlineto{\pgfqpoint{2.835627in}{2.850865in}}%
\pgfpathlineto{\pgfqpoint{2.896560in}{2.655827in}}%
\pgfpathlineto{\pgfqpoint{2.927027in}{2.538655in}}%
\pgfpathlineto{\pgfqpoint{2.972727in}{2.360070in}}%
\pgfpathlineto{\pgfqpoint{3.018428in}{2.189141in}}%
\pgfpathlineto{\pgfqpoint{3.033661in}{2.137035in}}%
\pgfpathlineto{\pgfqpoint{3.094595in}{1.969662in}}%
\pgfpathlineto{\pgfqpoint{3.155528in}{1.853076in}}%
\pgfpathlineto{\pgfqpoint{3.201229in}{1.801005in}}%
\pgfpathlineto{\pgfqpoint{3.216462in}{1.783924in}}%
\pgfpathlineto{\pgfqpoint{3.231695in}{1.775219in}}%
\pgfpathlineto{\pgfqpoint{3.277396in}{1.758339in}}%
\pgfpathlineto{\pgfqpoint{3.292629in}{1.722644in}}%
\pgfpathlineto{\pgfqpoint{3.307862in}{1.673207in}}%
\pgfpathlineto{\pgfqpoint{3.323096in}{1.630925in}}%
\pgfpathlineto{\pgfqpoint{3.338329in}{1.596906in}}%
\pgfpathlineto{\pgfqpoint{3.353563in}{1.570437in}}%
\pgfpathlineto{\pgfqpoint{3.384029in}{1.526751in}}%
\pgfpathlineto{\pgfqpoint{3.399263in}{1.498475in}}%
\pgfpathlineto{\pgfqpoint{3.429730in}{1.406330in}}%
\pgfpathlineto{\pgfqpoint{3.444963in}{1.392157in}}%
\pgfpathlineto{\pgfqpoint{3.460197in}{1.388183in}}%
\pgfpathlineto{\pgfqpoint{3.505897in}{1.337844in}}%
\pgfpathlineto{\pgfqpoint{3.536364in}{1.315695in}}%
\pgfpathlineto{\pgfqpoint{3.551597in}{1.306303in}}%
\pgfpathlineto{\pgfqpoint{3.566830in}{1.309460in}}%
\pgfpathlineto{\pgfqpoint{3.582064in}{1.356383in}}%
\pgfpathlineto{\pgfqpoint{3.597297in}{1.442296in}}%
\pgfpathlineto{\pgfqpoint{3.612531in}{1.498590in}}%
\pgfpathlineto{\pgfqpoint{3.627764in}{1.509395in}}%
\pgfpathlineto{\pgfqpoint{3.642998in}{1.504952in}}%
\pgfpathlineto{\pgfqpoint{3.658231in}{1.497308in}}%
\pgfpathlineto{\pgfqpoint{3.673464in}{1.491746in}}%
\pgfpathlineto{\pgfqpoint{3.703931in}{1.484544in}}%
\pgfpathlineto{\pgfqpoint{3.719165in}{1.485587in}}%
\pgfpathlineto{\pgfqpoint{3.734398in}{1.492379in}}%
\pgfpathlineto{\pgfqpoint{3.749631in}{1.497810in}}%
\pgfpathlineto{\pgfqpoint{3.764865in}{1.495599in}}%
\pgfpathlineto{\pgfqpoint{3.780098in}{1.488639in}}%
\pgfpathlineto{\pgfqpoint{3.795332in}{1.479567in}}%
\pgfpathlineto{\pgfqpoint{3.810565in}{1.462231in}}%
\pgfpathlineto{\pgfqpoint{3.825799in}{1.425528in}}%
\pgfpathlineto{\pgfqpoint{3.841032in}{1.367895in}}%
\pgfpathlineto{\pgfqpoint{3.856265in}{1.338152in}}%
\pgfpathlineto{\pgfqpoint{3.871499in}{1.346327in}}%
\pgfpathlineto{\pgfqpoint{3.886732in}{1.336390in}}%
\pgfpathlineto{\pgfqpoint{3.917199in}{1.267346in}}%
\pgfpathlineto{\pgfqpoint{3.947666in}{1.212521in}}%
\pgfpathlineto{\pgfqpoint{3.962899in}{1.178892in}}%
\pgfpathlineto{\pgfqpoint{3.993366in}{1.100211in}}%
\pgfpathlineto{\pgfqpoint{4.008600in}{1.060065in}}%
\pgfpathlineto{\pgfqpoint{4.023833in}{1.024709in}}%
\pgfpathlineto{\pgfqpoint{4.039066in}{0.999380in}}%
\pgfpathlineto{\pgfqpoint{4.054300in}{0.996954in}}%
\pgfpathlineto{\pgfqpoint{4.084767in}{1.071116in}}%
\pgfpathlineto{\pgfqpoint{4.100000in}{1.068886in}}%
\pgfpathlineto{\pgfqpoint{4.145700in}{1.001995in}}%
\pgfpathlineto{\pgfqpoint{4.160934in}{0.973546in}}%
\pgfpathlineto{\pgfqpoint{4.176167in}{0.947798in}}%
\pgfpathlineto{\pgfqpoint{4.191400in}{0.931365in}}%
\pgfpathlineto{\pgfqpoint{4.206634in}{0.924262in}}%
\pgfpathlineto{\pgfqpoint{4.221867in}{0.925338in}}%
\pgfpathlineto{\pgfqpoint{4.237101in}{0.938405in}}%
\pgfpathlineto{\pgfqpoint{4.252334in}{0.961615in}}%
\pgfpathlineto{\pgfqpoint{4.267568in}{0.991227in}}%
\pgfpathlineto{\pgfqpoint{4.282801in}{1.028989in}}%
\pgfpathlineto{\pgfqpoint{4.298034in}{1.062745in}}%
\pgfpathlineto{\pgfqpoint{4.313268in}{1.084881in}}%
\pgfpathlineto{\pgfqpoint{4.343735in}{1.222592in}}%
\pgfpathlineto{\pgfqpoint{4.358968in}{1.280546in}}%
\pgfpathlineto{\pgfqpoint{4.374201in}{1.343299in}}%
\pgfpathlineto{\pgfqpoint{4.404668in}{1.409018in}}%
\pgfpathlineto{\pgfqpoint{4.419902in}{1.447560in}}%
\pgfpathlineto{\pgfqpoint{4.435135in}{1.462780in}}%
\pgfpathlineto{\pgfqpoint{4.450369in}{1.459128in}}%
\pgfpathlineto{\pgfqpoint{4.465602in}{1.451235in}}%
\pgfpathlineto{\pgfqpoint{4.480835in}{1.446877in}}%
\pgfpathlineto{\pgfqpoint{4.572236in}{1.439168in}}%
\pgfpathlineto{\pgfqpoint{4.663636in}{1.416321in}}%
\pgfpathlineto{\pgfqpoint{4.678870in}{1.411657in}}%
\pgfpathlineto{\pgfqpoint{4.770270in}{1.373638in}}%
\pgfpathlineto{\pgfqpoint{4.800737in}{1.356787in}}%
\pgfpathlineto{\pgfqpoint{4.876904in}{1.312738in}}%
\pgfpathlineto{\pgfqpoint{5.059705in}{1.181455in}}%
\pgfpathlineto{\pgfqpoint{5.074939in}{1.170636in}}%
\pgfpathlineto{\pgfqpoint{5.166339in}{1.120895in}}%
\pgfpathlineto{\pgfqpoint{5.181572in}{1.113443in}}%
\pgfpathlineto{\pgfqpoint{5.272973in}{1.078524in}}%
\pgfpathlineto{\pgfqpoint{5.303440in}{1.070970in}}%
\pgfpathlineto{\pgfqpoint{5.379607in}{1.053974in}}%
\pgfpathlineto{\pgfqpoint{5.471007in}{1.047085in}}%
\pgfpathlineto{\pgfqpoint{5.486241in}{1.038835in}}%
\pgfpathlineto{\pgfqpoint{5.516708in}{1.017903in}}%
\pgfpathlineto{\pgfqpoint{5.547174in}{0.987236in}}%
\pgfpathlineto{\pgfqpoint{5.562408in}{0.972718in}}%
\pgfpathlineto{\pgfqpoint{5.592875in}{0.922519in}}%
\pgfpathlineto{\pgfqpoint{5.608108in}{0.905856in}}%
\pgfpathlineto{\pgfqpoint{5.623342in}{0.870672in}}%
\pgfpathlineto{\pgfqpoint{5.638575in}{0.823965in}}%
\pgfpathlineto{\pgfqpoint{5.653808in}{0.794213in}}%
\pgfpathlineto{\pgfqpoint{5.684275in}{0.761011in}}%
\pgfpathlineto{\pgfqpoint{5.714742in}{0.673842in}}%
\pgfpathlineto{\pgfqpoint{5.745209in}{0.632286in}}%
\pgfpathlineto{\pgfqpoint{5.760442in}{0.605600in}}%
\pgfpathlineto{\pgfqpoint{5.775676in}{0.591469in}}%
\pgfpathlineto{\pgfqpoint{5.790909in}{0.607326in}}%
\pgfpathlineto{\pgfqpoint{5.806143in}{0.626000in}}%
\pgfpathlineto{\pgfqpoint{5.821376in}{0.632378in}}%
\pgfpathlineto{\pgfqpoint{5.836609in}{0.640955in}}%
\pgfpathlineto{\pgfqpoint{5.851843in}{0.642687in}}%
\pgfpathlineto{\pgfqpoint{5.867076in}{0.627918in}}%
\pgfpathlineto{\pgfqpoint{5.897543in}{0.580000in}}%
\pgfpathlineto{\pgfqpoint{5.912776in}{0.588304in}}%
\pgfpathlineto{\pgfqpoint{5.928010in}{0.629685in}}%
\pgfpathlineto{\pgfqpoint{5.943243in}{0.631689in}}%
\pgfpathlineto{\pgfqpoint{5.958477in}{0.607162in}}%
\pgfpathlineto{\pgfqpoint{5.973710in}{0.608787in}}%
\pgfpathlineto{\pgfqpoint{5.988943in}{0.619218in}}%
\pgfpathlineto{\pgfqpoint{6.004177in}{0.619497in}}%
\pgfpathlineto{\pgfqpoint{6.019410in}{0.622254in}}%
\pgfpathlineto{\pgfqpoint{6.034644in}{0.631110in}}%
\pgfpathlineto{\pgfqpoint{6.065111in}{0.642581in}}%
\pgfpathlineto{\pgfqpoint{6.080344in}{0.658514in}}%
\pgfpathlineto{\pgfqpoint{6.095577in}{0.684377in}}%
\pgfpathlineto{\pgfqpoint{6.110811in}{0.722313in}}%
\pgfpathlineto{\pgfqpoint{6.126044in}{0.772044in}}%
\pgfpathlineto{\pgfqpoint{6.141278in}{0.812010in}}%
\pgfpathlineto{\pgfqpoint{6.156511in}{0.821943in}}%
\pgfpathlineto{\pgfqpoint{6.171744in}{0.811626in}}%
\pgfpathlineto{\pgfqpoint{6.186978in}{0.796743in}}%
\pgfpathlineto{\pgfqpoint{6.202211in}{0.789968in}}%
\pgfpathlineto{\pgfqpoint{6.918182in}{0.789968in}}%
\pgfpathlineto{\pgfqpoint{6.918182in}{0.789968in}}%
\pgfusepath{stroke}%
\end{pgfscope}%
\begin{pgfscope}%
\pgfpathrectangle{\pgfqpoint{1.000000in}{0.440000in}}{\pgfqpoint{6.200000in}{3.080000in}} %
\pgfusepath{clip}%
\pgfsetrectcap%
\pgfsetroundjoin%
\pgfsetlinewidth{0.501875pt}%
\definecolor{currentstroke}{rgb}{0.501961,0.501961,0.501961}%
\pgfsetstrokecolor{currentstroke}%
\pgfsetdash{}{0pt}%
\pgfpathmoveto{\pgfqpoint{1.891155in}{0.440000in}}%
\pgfpathlineto{\pgfqpoint{1.891155in}{3.520000in}}%
\pgfusepath{stroke}%
\end{pgfscope}%
\begin{pgfscope}%
\pgfpathrectangle{\pgfqpoint{1.000000in}{0.440000in}}{\pgfqpoint{6.200000in}{3.080000in}} %
\pgfusepath{clip}%
\pgfsetrectcap%
\pgfsetroundjoin%
\pgfsetlinewidth{0.501875pt}%
\definecolor{currentstroke}{rgb}{0.501961,0.501961,0.501961}%
\pgfsetstrokecolor{currentstroke}%
\pgfsetdash{}{0pt}%
\pgfpathmoveto{\pgfqpoint{2.195823in}{0.440000in}}%
\pgfpathlineto{\pgfqpoint{2.195823in}{3.520000in}}%
\pgfusepath{stroke}%
\end{pgfscope}%
\begin{pgfscope}%
\pgfpathrectangle{\pgfqpoint{1.000000in}{0.440000in}}{\pgfqpoint{6.200000in}{3.080000in}} %
\pgfusepath{clip}%
\pgfsetrectcap%
\pgfsetroundjoin%
\pgfsetlinewidth{0.501875pt}%
\definecolor{currentstroke}{rgb}{0.501961,0.501961,0.501961}%
\pgfsetstrokecolor{currentstroke}%
\pgfsetdash{}{0pt}%
\pgfpathmoveto{\pgfqpoint{2.500491in}{0.440000in}}%
\pgfpathlineto{\pgfqpoint{2.500491in}{3.520000in}}%
\pgfusepath{stroke}%
\end{pgfscope}%
\begin{pgfscope}%
\pgfpathrectangle{\pgfqpoint{1.000000in}{0.440000in}}{\pgfqpoint{6.200000in}{3.080000in}} %
\pgfusepath{clip}%
\pgfsetrectcap%
\pgfsetroundjoin%
\pgfsetlinewidth{0.501875pt}%
\definecolor{currentstroke}{rgb}{0.501961,0.501961,0.501961}%
\pgfsetstrokecolor{currentstroke}%
\pgfsetdash{}{0pt}%
\pgfpathmoveto{\pgfqpoint{2.805160in}{0.440000in}}%
\pgfpathlineto{\pgfqpoint{2.805160in}{3.520000in}}%
\pgfusepath{stroke}%
\end{pgfscope}%
\begin{pgfscope}%
\pgfpathrectangle{\pgfqpoint{1.000000in}{0.440000in}}{\pgfqpoint{6.200000in}{3.080000in}} %
\pgfusepath{clip}%
\pgfsetrectcap%
\pgfsetroundjoin%
\pgfsetlinewidth{0.501875pt}%
\definecolor{currentstroke}{rgb}{0.501961,0.501961,0.501961}%
\pgfsetstrokecolor{currentstroke}%
\pgfsetdash{}{0pt}%
\pgfpathmoveto{\pgfqpoint{3.109828in}{0.440000in}}%
\pgfpathlineto{\pgfqpoint{3.109828in}{3.520000in}}%
\pgfusepath{stroke}%
\end{pgfscope}%
\begin{pgfscope}%
\pgfpathrectangle{\pgfqpoint{1.000000in}{0.440000in}}{\pgfqpoint{6.200000in}{3.080000in}} %
\pgfusepath{clip}%
\pgfsetrectcap%
\pgfsetroundjoin%
\pgfsetlinewidth{0.501875pt}%
\definecolor{currentstroke}{rgb}{0.501961,0.501961,0.501961}%
\pgfsetstrokecolor{currentstroke}%
\pgfsetdash{}{0pt}%
\pgfpathmoveto{\pgfqpoint{3.414496in}{0.440000in}}%
\pgfpathlineto{\pgfqpoint{3.414496in}{3.520000in}}%
\pgfusepath{stroke}%
\end{pgfscope}%
\begin{pgfscope}%
\pgfpathrectangle{\pgfqpoint{1.000000in}{0.440000in}}{\pgfqpoint{6.200000in}{3.080000in}} %
\pgfusepath{clip}%
\pgfsetrectcap%
\pgfsetroundjoin%
\pgfsetlinewidth{0.501875pt}%
\definecolor{currentstroke}{rgb}{0.501961,0.501961,0.501961}%
\pgfsetstrokecolor{currentstroke}%
\pgfsetdash{}{0pt}%
\pgfpathmoveto{\pgfqpoint{3.719165in}{0.440000in}}%
\pgfpathlineto{\pgfqpoint{3.719165in}{3.520000in}}%
\pgfusepath{stroke}%
\end{pgfscope}%
\begin{pgfscope}%
\pgfpathrectangle{\pgfqpoint{1.000000in}{0.440000in}}{\pgfqpoint{6.200000in}{3.080000in}} %
\pgfusepath{clip}%
\pgfsetrectcap%
\pgfsetroundjoin%
\pgfsetlinewidth{0.501875pt}%
\definecolor{currentstroke}{rgb}{0.501961,0.501961,0.501961}%
\pgfsetstrokecolor{currentstroke}%
\pgfsetdash{}{0pt}%
\pgfpathmoveto{\pgfqpoint{4.023833in}{0.440000in}}%
\pgfpathlineto{\pgfqpoint{4.023833in}{3.520000in}}%
\pgfusepath{stroke}%
\end{pgfscope}%
\begin{pgfscope}%
\pgfpathrectangle{\pgfqpoint{1.000000in}{0.440000in}}{\pgfqpoint{6.200000in}{3.080000in}} %
\pgfusepath{clip}%
\pgfsetrectcap%
\pgfsetroundjoin%
\pgfsetlinewidth{0.501875pt}%
\definecolor{currentstroke}{rgb}{0.501961,0.501961,0.501961}%
\pgfsetstrokecolor{currentstroke}%
\pgfsetdash{}{0pt}%
\pgfpathmoveto{\pgfqpoint{4.328501in}{0.440000in}}%
\pgfpathlineto{\pgfqpoint{4.328501in}{3.520000in}}%
\pgfusepath{stroke}%
\end{pgfscope}%
\begin{pgfscope}%
\pgfpathrectangle{\pgfqpoint{1.000000in}{0.440000in}}{\pgfqpoint{6.200000in}{3.080000in}} %
\pgfusepath{clip}%
\pgfsetrectcap%
\pgfsetroundjoin%
\pgfsetlinewidth{0.501875pt}%
\definecolor{currentstroke}{rgb}{0.501961,0.501961,0.501961}%
\pgfsetstrokecolor{currentstroke}%
\pgfsetdash{}{0pt}%
\pgfpathmoveto{\pgfqpoint{4.633170in}{0.440000in}}%
\pgfpathlineto{\pgfqpoint{4.633170in}{3.520000in}}%
\pgfusepath{stroke}%
\end{pgfscope}%
\begin{pgfscope}%
\pgfpathrectangle{\pgfqpoint{1.000000in}{0.440000in}}{\pgfqpoint{6.200000in}{3.080000in}} %
\pgfusepath{clip}%
\pgfsetrectcap%
\pgfsetroundjoin%
\pgfsetlinewidth{0.501875pt}%
\definecolor{currentstroke}{rgb}{0.501961,0.501961,0.501961}%
\pgfsetstrokecolor{currentstroke}%
\pgfsetdash{}{0pt}%
\pgfpathmoveto{\pgfqpoint{4.937838in}{0.440000in}}%
\pgfpathlineto{\pgfqpoint{4.937838in}{3.520000in}}%
\pgfusepath{stroke}%
\end{pgfscope}%
\begin{pgfscope}%
\pgfpathrectangle{\pgfqpoint{1.000000in}{0.440000in}}{\pgfqpoint{6.200000in}{3.080000in}} %
\pgfusepath{clip}%
\pgfsetrectcap%
\pgfsetroundjoin%
\pgfsetlinewidth{0.501875pt}%
\definecolor{currentstroke}{rgb}{0.501961,0.501961,0.501961}%
\pgfsetstrokecolor{currentstroke}%
\pgfsetdash{}{0pt}%
\pgfpathmoveto{\pgfqpoint{5.242506in}{0.440000in}}%
\pgfpathlineto{\pgfqpoint{5.242506in}{3.520000in}}%
\pgfusepath{stroke}%
\end{pgfscope}%
\begin{pgfscope}%
\pgfpathrectangle{\pgfqpoint{1.000000in}{0.440000in}}{\pgfqpoint{6.200000in}{3.080000in}} %
\pgfusepath{clip}%
\pgfsetrectcap%
\pgfsetroundjoin%
\pgfsetlinewidth{0.501875pt}%
\definecolor{currentstroke}{rgb}{0.501961,0.501961,0.501961}%
\pgfsetstrokecolor{currentstroke}%
\pgfsetdash{}{0pt}%
\pgfpathmoveto{\pgfqpoint{5.547174in}{0.440000in}}%
\pgfpathlineto{\pgfqpoint{5.547174in}{3.520000in}}%
\pgfusepath{stroke}%
\end{pgfscope}%
\begin{pgfscope}%
\pgfpathrectangle{\pgfqpoint{1.000000in}{0.440000in}}{\pgfqpoint{6.200000in}{3.080000in}} %
\pgfusepath{clip}%
\pgfsetrectcap%
\pgfsetroundjoin%
\pgfsetlinewidth{0.501875pt}%
\definecolor{currentstroke}{rgb}{0.501961,0.501961,0.501961}%
\pgfsetstrokecolor{currentstroke}%
\pgfsetdash{}{0pt}%
\pgfpathmoveto{\pgfqpoint{5.851843in}{0.440000in}}%
\pgfpathlineto{\pgfqpoint{5.851843in}{3.520000in}}%
\pgfusepath{stroke}%
\end{pgfscope}%
\begin{pgfscope}%
\pgfpathrectangle{\pgfqpoint{1.000000in}{0.440000in}}{\pgfqpoint{6.200000in}{3.080000in}} %
\pgfusepath{clip}%
\pgfsetrectcap%
\pgfsetroundjoin%
\pgfsetlinewidth{0.501875pt}%
\definecolor{currentstroke}{rgb}{0.501961,0.501961,0.501961}%
\pgfsetstrokecolor{currentstroke}%
\pgfsetdash{}{0pt}%
\pgfpathmoveto{\pgfqpoint{6.156511in}{0.440000in}}%
\pgfpathlineto{\pgfqpoint{6.156511in}{3.520000in}}%
\pgfusepath{stroke}%
\end{pgfscope}%
\begin{pgfscope}%
\pgfpathrectangle{\pgfqpoint{1.000000in}{0.440000in}}{\pgfqpoint{6.200000in}{3.080000in}} %
\pgfusepath{clip}%
\pgfsetrectcap%
\pgfsetroundjoin%
\pgfsetlinewidth{0.501875pt}%
\definecolor{currentstroke}{rgb}{0.501961,0.501961,0.501961}%
\pgfsetstrokecolor{currentstroke}%
\pgfsetdash{}{0pt}%
\pgfpathmoveto{\pgfqpoint{6.461179in}{0.440000in}}%
\pgfpathlineto{\pgfqpoint{6.461179in}{3.520000in}}%
\pgfusepath{stroke}%
\end{pgfscope}%
\begin{pgfscope}%
\pgfpathrectangle{\pgfqpoint{1.000000in}{0.440000in}}{\pgfqpoint{6.200000in}{3.080000in}} %
\pgfusepath{clip}%
\pgfsetbuttcap%
\pgfsetroundjoin%
\definecolor{currentfill}{rgb}{0.364706,0.517647,0.572549}%
\pgfsetfillcolor{currentfill}%
\pgfsetlinewidth{1.003750pt}%
\definecolor{currentstroke}{rgb}{0.364706,0.517647,0.572549}%
\pgfsetstrokecolor{currentstroke}%
\pgfsetdash{}{0pt}%
\pgfsys@defobject{currentmarker}{\pgfqpoint{-0.027778in}{-0.027778in}}{\pgfqpoint{0.027778in}{0.027778in}}{%
\pgfpathmoveto{\pgfqpoint{0.000000in}{-0.027778in}}%
\pgfpathcurveto{\pgfqpoint{0.007367in}{-0.027778in}}{\pgfqpoint{0.014433in}{-0.024851in}}{\pgfqpoint{0.019642in}{-0.019642in}}%
\pgfpathcurveto{\pgfqpoint{0.024851in}{-0.014433in}}{\pgfqpoint{0.027778in}{-0.007367in}}{\pgfqpoint{0.027778in}{0.000000in}}%
\pgfpathcurveto{\pgfqpoint{0.027778in}{0.007367in}}{\pgfqpoint{0.024851in}{0.014433in}}{\pgfqpoint{0.019642in}{0.019642in}}%
\pgfpathcurveto{\pgfqpoint{0.014433in}{0.024851in}}{\pgfqpoint{0.007367in}{0.027778in}}{\pgfqpoint{0.000000in}{0.027778in}}%
\pgfpathcurveto{\pgfqpoint{-0.007367in}{0.027778in}}{\pgfqpoint{-0.014433in}{0.024851in}}{\pgfqpoint{-0.019642in}{0.019642in}}%
\pgfpathcurveto{\pgfqpoint{-0.024851in}{0.014433in}}{\pgfqpoint{-0.027778in}{0.007367in}}{\pgfqpoint{-0.027778in}{0.000000in}}%
\pgfpathcurveto{\pgfqpoint{-0.027778in}{-0.007367in}}{\pgfqpoint{-0.024851in}{-0.014433in}}{\pgfqpoint{-0.019642in}{-0.019642in}}%
\pgfpathcurveto{\pgfqpoint{-0.014433in}{-0.024851in}}{\pgfqpoint{-0.007367in}{-0.027778in}}{\pgfqpoint{0.000000in}{-0.027778in}}%
\pgfpathclose%
\pgfusepath{stroke,fill}%
}%
\begin{pgfscope}%
\pgfsys@transformshift{1.891155in}{1.169484in}%
\pgfsys@useobject{currentmarker}{}%
\end{pgfscope}%
\begin{pgfscope}%
\pgfsys@transformshift{2.195823in}{1.051990in}%
\pgfsys@useobject{currentmarker}{}%
\end{pgfscope}%
\begin{pgfscope}%
\pgfsys@transformshift{2.500491in}{2.910174in}%
\pgfsys@useobject{currentmarker}{}%
\end{pgfscope}%
\begin{pgfscope}%
\pgfsys@transformshift{2.805160in}{2.923348in}%
\pgfsys@useobject{currentmarker}{}%
\end{pgfscope}%
\begin{pgfscope}%
\pgfsys@transformshift{3.109828in}{1.940515in}%
\pgfsys@useobject{currentmarker}{}%
\end{pgfscope}%
\begin{pgfscope}%
\pgfsys@transformshift{3.414496in}{1.453579in}%
\pgfsys@useobject{currentmarker}{}%
\end{pgfscope}%
\begin{pgfscope}%
\pgfsys@transformshift{3.719165in}{1.485587in}%
\pgfsys@useobject{currentmarker}{}%
\end{pgfscope}%
\begin{pgfscope}%
\pgfsys@transformshift{4.023833in}{1.024709in}%
\pgfsys@useobject{currentmarker}{}%
\end{pgfscope}%
\begin{pgfscope}%
\pgfsys@transformshift{4.328501in}{1.151806in}%
\pgfsys@useobject{currentmarker}{}%
\end{pgfscope}%
\begin{pgfscope}%
\pgfsys@transformshift{4.633170in}{1.423995in}%
\pgfsys@useobject{currentmarker}{}%
\end{pgfscope}%
\begin{pgfscope}%
\pgfsys@transformshift{4.937838in}{1.268560in}%
\pgfsys@useobject{currentmarker}{}%
\end{pgfscope}%
\begin{pgfscope}%
\pgfsys@transformshift{5.242506in}{1.090163in}%
\pgfsys@useobject{currentmarker}{}%
\end{pgfscope}%
\begin{pgfscope}%
\pgfsys@transformshift{5.547174in}{0.987236in}%
\pgfsys@useobject{currentmarker}{}%
\end{pgfscope}%
\begin{pgfscope}%
\pgfsys@transformshift{5.851843in}{0.642687in}%
\pgfsys@useobject{currentmarker}{}%
\end{pgfscope}%
\begin{pgfscope}%
\pgfsys@transformshift{6.156511in}{0.821943in}%
\pgfsys@useobject{currentmarker}{}%
\end{pgfscope}%
\begin{pgfscope}%
\pgfsys@transformshift{6.461179in}{0.789968in}%
\pgfsys@useobject{currentmarker}{}%
\end{pgfscope}%
\end{pgfscope}%
\begin{pgfscope}%
\pgfsetrectcap%
\pgfsetmiterjoin%
\pgfsetlinewidth{0.803000pt}%
\definecolor{currentstroke}{rgb}{0.000000,0.000000,0.000000}%
\pgfsetstrokecolor{currentstroke}%
\pgfsetdash{}{0pt}%
\pgfpathmoveto{\pgfqpoint{1.000000in}{0.440000in}}%
\pgfpathlineto{\pgfqpoint{1.000000in}{3.520000in}}%
\pgfusepath{stroke}%
\end{pgfscope}%
\begin{pgfscope}%
\pgfsetrectcap%
\pgfsetmiterjoin%
\pgfsetlinewidth{0.803000pt}%
\definecolor{currentstroke}{rgb}{0.000000,0.000000,0.000000}%
\pgfsetstrokecolor{currentstroke}%
\pgfsetdash{}{0pt}%
\pgfpathmoveto{\pgfqpoint{7.200000in}{0.440000in}}%
\pgfpathlineto{\pgfqpoint{7.200000in}{3.520000in}}%
\pgfusepath{stroke}%
\end{pgfscope}%
\begin{pgfscope}%
\pgfsetrectcap%
\pgfsetmiterjoin%
\pgfsetlinewidth{0.803000pt}%
\definecolor{currentstroke}{rgb}{0.000000,0.000000,0.000000}%
\pgfsetstrokecolor{currentstroke}%
\pgfsetdash{}{0pt}%
\pgfpathmoveto{\pgfqpoint{1.000000in}{0.440000in}}%
\pgfpathlineto{\pgfqpoint{7.200000in}{0.440000in}}%
\pgfusepath{stroke}%
\end{pgfscope}%
\begin{pgfscope}%
\pgfsetrectcap%
\pgfsetmiterjoin%
\pgfsetlinewidth{0.803000pt}%
\definecolor{currentstroke}{rgb}{0.000000,0.000000,0.000000}%
\pgfsetstrokecolor{currentstroke}%
\pgfsetdash{}{0pt}%
\pgfpathmoveto{\pgfqpoint{1.000000in}{3.520000in}}%
\pgfpathlineto{\pgfqpoint{7.200000in}{3.520000in}}%
\pgfusepath{stroke}%
\end{pgfscope}%
\begin{pgfscope}%
\pgftext[x=1.835299in,y=0.580000in,left,base]{\rmfamily\fontsize{12.000000}{14.400000}\selectfont @}%
\end{pgfscope}%
\begin{pgfscope}%
\pgftext[x=2.241523in,y=0.580000in,left,base]{\rmfamily\fontsize{12.000000}{14.400000}\selectfont mid}%
\end{pgfscope}%
\begin{pgfscope}%
\pgftext[x=2.759459in,y=0.580000in,left,base]{\rmfamily\fontsize{12.000000}{14.400000}\selectfont suh}%
\end{pgfscope}%
\begin{pgfscope}%
\pgftext[x=3.521130in,y=0.580000in,left,base]{\rmfamily\fontsize{12.000000}{14.400000}\selectfont m@}%
\end{pgfscope}%
\begin{pgfscope}%
\pgftext[x=4.094922in,y=0.580000in,left,base]{\rmfamily\fontsize{12.000000}{14.400000}\selectfont naits}%
\end{pgfscope}%
\begin{pgfscope}%
\pgftext[x=5.344062in,y=0.580000in,left,base]{\rmfamily\fontsize{12.000000}{14.400000}\selectfont driim}%
\end{pgfscope}%
\begin{pgfscope}%
\pgfsetbuttcap%
\pgfsetmiterjoin%
\definecolor{currentfill}{rgb}{1.000000,1.000000,1.000000}%
\pgfsetfillcolor{currentfill}%
\pgfsetfillopacity{0.800000}%
\pgfsetlinewidth{1.003750pt}%
\definecolor{currentstroke}{rgb}{0.800000,0.800000,0.800000}%
\pgfsetstrokecolor{currentstroke}%
\pgfsetstrokeopacity{0.800000}%
\pgfsetdash{}{0pt}%
\pgfpathmoveto{\pgfqpoint{5.541322in}{3.021556in}}%
\pgfpathlineto{\pgfqpoint{7.102778in}{3.021556in}}%
\pgfpathquadraticcurveto{\pgfqpoint{7.130556in}{3.021556in}}{\pgfqpoint{7.130556in}{3.049334in}}%
\pgfpathlineto{\pgfqpoint{7.130556in}{3.422778in}}%
\pgfpathquadraticcurveto{\pgfqpoint{7.130556in}{3.450556in}}{\pgfqpoint{7.102778in}{3.450556in}}%
\pgfpathlineto{\pgfqpoint{5.541322in}{3.450556in}}%
\pgfpathquadraticcurveto{\pgfqpoint{5.513544in}{3.450556in}}{\pgfqpoint{5.513544in}{3.422778in}}%
\pgfpathlineto{\pgfqpoint{5.513544in}{3.049334in}}%
\pgfpathquadraticcurveto{\pgfqpoint{5.513544in}{3.021556in}}{\pgfqpoint{5.541322in}{3.021556in}}%
\pgfpathclose%
\pgfusepath{stroke,fill}%
\end{pgfscope}%
\begin{pgfscope}%
\pgfsetbuttcap%
\pgfsetmiterjoin%
\definecolor{currentfill}{rgb}{0.364706,0.517647,0.572549}%
\pgfsetfillcolor{currentfill}%
\pgfsetlinewidth{1.003750pt}%
\definecolor{currentstroke}{rgb}{0.364706,0.517647,0.572549}%
\pgfsetstrokecolor{currentstroke}%
\pgfsetdash{}{0pt}%
\pgfpathmoveto{\pgfqpoint{5.569100in}{3.297778in}}%
\pgfpathlineto{\pgfqpoint{5.846878in}{3.297778in}}%
\pgfpathlineto{\pgfqpoint{5.846878in}{3.395000in}}%
\pgfpathlineto{\pgfqpoint{5.569100in}{3.395000in}}%
\pgfpathclose%
\pgfusepath{stroke,fill}%
\end{pgfscope}%
\begin{pgfscope}%
\pgftext[x=5.957989in,y=3.297778in,left,base]{\rmfamily\fontsize{10.000000}{12.000000}\selectfont Interpolated \(\displaystyle F_0\)}%
\end{pgfscope}%
\begin{pgfscope}%
\pgfsetbuttcap%
\pgfsetmiterjoin%
\definecolor{currentfill}{rgb}{0.800000,0.000000,0.266667}%
\pgfsetfillcolor{currentfill}%
\pgfsetlinewidth{1.003750pt}%
\definecolor{currentstroke}{rgb}{0.800000,0.000000,0.266667}%
\pgfsetstrokecolor{currentstroke}%
\pgfsetdash{}{0pt}%
\pgfpathmoveto{\pgfqpoint{5.569100in}{3.104112in}}%
\pgfpathlineto{\pgfqpoint{5.846878in}{3.104112in}}%
\pgfpathlineto{\pgfqpoint{5.846878in}{3.201334in}}%
\pgfpathlineto{\pgfqpoint{5.569100in}{3.201334in}}%
\pgfpathclose%
\pgfusepath{stroke,fill}%
\end{pgfscope}%
\begin{pgfscope}%
\pgftext[x=5.957989in,y=3.104112in,left,base]{\rmfamily\fontsize{10.000000}{12.000000}\selectfont Syllable Boundary}%
\end{pgfscope}%
\end{pgfpicture}%
\makeatother%
\endgroup%
}
\caption[Fixed sampling rate]{Location of extraction points in an \ac{F0} contour at a fixed 10~Hz sampling rate.}
\label{fig:fixed-anchor}
\end{figure}



\subsection{Fixed Anchor-Point Approaches}

A commonly adopted solution to the problems that arises from the adoption of low fixed sampling rates is to use so-called ``anchor points'', i.e., linguistically relevant positions within the utterance \citep[e.g.,][]{Santen1997Modeling}.
One might, for example, choose to sample \ac{F0} measurements at the start, center, and end of each syllable.

The advantage of this approach is that the contour can be represented into a much more compact format, which makes training a lot easier.
Under this approach we also make sure that we have some information about each and every instance of the chosen support level (usually  the syllable).

Unlike the frame-by-frame approach, this approach has to make a number of assumptions about pitch and its perception.
The first assumption that we have to make is that phenomena that are lost as a consequence of the downsampling process are not that important and that reconstructed contours are still accepted by humans as natural.
Additionally, we have to assume that the timing of prosodic events corresponds to linguistic units such as syllables or syllable sub-units, which might not always be the case for all natural languages.
An additional assumption that we have to make if we simply use the same number of anchor points per linguistic segments is that duration has no bearing on the shape of a pitch template (unless of course duration is explicitly provided as an input feature to the model).
Under this assumption, we might fail to account for physiological constraints about the vocal tract, such as how fast pitch rises or falls can be produced.

One disadvantage illustrated by using the same number of anchor points per syllable is illustrated by \autoref{fig:low-anchor}.
As we can see, with a vanilla implementation of the anchor-point approach we are unable to use more data points for longer syllables where we might observe more complex patterns.


\begin{figure}[h]
\centering
\resizebox{\textwidth}{!}{%% Creator: Matplotlib, PGF backend
%%
%% To include the figure in your LaTeX document, write
%%   \input{<filename>.pgf}
%%
%% Make sure the required packages are loaded in your preamble
%%   \usepackage{pgf}
%%
%% Figures using additional raster images can only be included by \input if
%% they are in the same directory as the main LaTeX file. For loading figures
%% from other directories you can use the `import` package
%%   \usepackage{import}
%% and then include the figures with
%%   \import{<path to file>}{<filename>.pgf}
%%
%% Matplotlib used the following preamble
%%   \usepackage[utf8x]{inputenc}
%%   \usepackage[T1]{fontenc}
%%   \usepackage{cmbright}
%%
\begingroup%
\makeatletter%
\begin{pgfpicture}%
\pgfpathrectangle{\pgfpointorigin}{\pgfqpoint{8.000000in}{4.000000in}}%
\pgfusepath{use as bounding box, clip}%
\begin{pgfscope}%
\pgfsetbuttcap%
\pgfsetmiterjoin%
\definecolor{currentfill}{rgb}{1.000000,1.000000,1.000000}%
\pgfsetfillcolor{currentfill}%
\pgfsetlinewidth{0.000000pt}%
\definecolor{currentstroke}{rgb}{1.000000,1.000000,1.000000}%
\pgfsetstrokecolor{currentstroke}%
\pgfsetdash{}{0pt}%
\pgfpathmoveto{\pgfqpoint{0.000000in}{0.000000in}}%
\pgfpathlineto{\pgfqpoint{8.000000in}{0.000000in}}%
\pgfpathlineto{\pgfqpoint{8.000000in}{4.000000in}}%
\pgfpathlineto{\pgfqpoint{0.000000in}{4.000000in}}%
\pgfpathclose%
\pgfusepath{fill}%
\end{pgfscope}%
\begin{pgfscope}%
\pgfsetbuttcap%
\pgfsetmiterjoin%
\definecolor{currentfill}{rgb}{1.000000,1.000000,1.000000}%
\pgfsetfillcolor{currentfill}%
\pgfsetlinewidth{0.000000pt}%
\definecolor{currentstroke}{rgb}{0.000000,0.000000,0.000000}%
\pgfsetstrokecolor{currentstroke}%
\pgfsetstrokeopacity{0.000000}%
\pgfsetdash{}{0pt}%
\pgfpathmoveto{\pgfqpoint{1.000000in}{0.440000in}}%
\pgfpathlineto{\pgfqpoint{7.200000in}{0.440000in}}%
\pgfpathlineto{\pgfqpoint{7.200000in}{3.520000in}}%
\pgfpathlineto{\pgfqpoint{1.000000in}{3.520000in}}%
\pgfpathclose%
\pgfusepath{fill}%
\end{pgfscope}%
\begin{pgfscope}%
\pgfsetbuttcap%
\pgfsetroundjoin%
\definecolor{currentfill}{rgb}{0.000000,0.000000,0.000000}%
\pgfsetfillcolor{currentfill}%
\pgfsetlinewidth{0.803000pt}%
\definecolor{currentstroke}{rgb}{0.000000,0.000000,0.000000}%
\pgfsetstrokecolor{currentstroke}%
\pgfsetdash{}{0pt}%
\pgfsys@defobject{currentmarker}{\pgfqpoint{0.000000in}{-0.048611in}}{\pgfqpoint{0.000000in}{0.000000in}}{%
\pgfpathmoveto{\pgfqpoint{0.000000in}{0.000000in}}%
\pgfpathlineto{\pgfqpoint{0.000000in}{-0.048611in}}%
\pgfusepath{stroke,fill}%
}%
\begin{pgfscope}%
\pgfsys@transformshift{1.281818in}{0.440000in}%
\pgfsys@useobject{currentmarker}{}%
\end{pgfscope}%
\end{pgfscope}%
\begin{pgfscope}%
\pgftext[x=1.281818in,y=0.342778in,,top]{\rmfamily\fontsize{10.000000}{12.000000}\selectfont \(\displaystyle 0.00\)}%
\end{pgfscope}%
\begin{pgfscope}%
\pgfsetbuttcap%
\pgfsetroundjoin%
\definecolor{currentfill}{rgb}{0.000000,0.000000,0.000000}%
\pgfsetfillcolor{currentfill}%
\pgfsetlinewidth{0.803000pt}%
\definecolor{currentstroke}{rgb}{0.000000,0.000000,0.000000}%
\pgfsetstrokecolor{currentstroke}%
\pgfsetdash{}{0pt}%
\pgfsys@defobject{currentmarker}{\pgfqpoint{0.000000in}{-0.048611in}}{\pgfqpoint{0.000000in}{0.000000in}}{%
\pgfpathmoveto{\pgfqpoint{0.000000in}{0.000000in}}%
\pgfpathlineto{\pgfqpoint{0.000000in}{-0.048611in}}%
\pgfusepath{stroke,fill}%
}%
\begin{pgfscope}%
\pgfsys@transformshift{2.043489in}{0.440000in}%
\pgfsys@useobject{currentmarker}{}%
\end{pgfscope}%
\end{pgfscope}%
\begin{pgfscope}%
\pgftext[x=2.043489in,y=0.342778in,,top]{\rmfamily\fontsize{10.000000}{12.000000}\selectfont \(\displaystyle 0.25\)}%
\end{pgfscope}%
\begin{pgfscope}%
\pgfsetbuttcap%
\pgfsetroundjoin%
\definecolor{currentfill}{rgb}{0.000000,0.000000,0.000000}%
\pgfsetfillcolor{currentfill}%
\pgfsetlinewidth{0.803000pt}%
\definecolor{currentstroke}{rgb}{0.000000,0.000000,0.000000}%
\pgfsetstrokecolor{currentstroke}%
\pgfsetdash{}{0pt}%
\pgfsys@defobject{currentmarker}{\pgfqpoint{0.000000in}{-0.048611in}}{\pgfqpoint{0.000000in}{0.000000in}}{%
\pgfpathmoveto{\pgfqpoint{0.000000in}{0.000000in}}%
\pgfpathlineto{\pgfqpoint{0.000000in}{-0.048611in}}%
\pgfusepath{stroke,fill}%
}%
\begin{pgfscope}%
\pgfsys@transformshift{2.805160in}{0.440000in}%
\pgfsys@useobject{currentmarker}{}%
\end{pgfscope}%
\end{pgfscope}%
\begin{pgfscope}%
\pgftext[x=2.805160in,y=0.342778in,,top]{\rmfamily\fontsize{10.000000}{12.000000}\selectfont \(\displaystyle 0.50\)}%
\end{pgfscope}%
\begin{pgfscope}%
\pgfsetbuttcap%
\pgfsetroundjoin%
\definecolor{currentfill}{rgb}{0.000000,0.000000,0.000000}%
\pgfsetfillcolor{currentfill}%
\pgfsetlinewidth{0.803000pt}%
\definecolor{currentstroke}{rgb}{0.000000,0.000000,0.000000}%
\pgfsetstrokecolor{currentstroke}%
\pgfsetdash{}{0pt}%
\pgfsys@defobject{currentmarker}{\pgfqpoint{0.000000in}{-0.048611in}}{\pgfqpoint{0.000000in}{0.000000in}}{%
\pgfpathmoveto{\pgfqpoint{0.000000in}{0.000000in}}%
\pgfpathlineto{\pgfqpoint{0.000000in}{-0.048611in}}%
\pgfusepath{stroke,fill}%
}%
\begin{pgfscope}%
\pgfsys@transformshift{3.566830in}{0.440000in}%
\pgfsys@useobject{currentmarker}{}%
\end{pgfscope}%
\end{pgfscope}%
\begin{pgfscope}%
\pgftext[x=3.566830in,y=0.342778in,,top]{\rmfamily\fontsize{10.000000}{12.000000}\selectfont \(\displaystyle 0.75\)}%
\end{pgfscope}%
\begin{pgfscope}%
\pgfsetbuttcap%
\pgfsetroundjoin%
\definecolor{currentfill}{rgb}{0.000000,0.000000,0.000000}%
\pgfsetfillcolor{currentfill}%
\pgfsetlinewidth{0.803000pt}%
\definecolor{currentstroke}{rgb}{0.000000,0.000000,0.000000}%
\pgfsetstrokecolor{currentstroke}%
\pgfsetdash{}{0pt}%
\pgfsys@defobject{currentmarker}{\pgfqpoint{0.000000in}{-0.048611in}}{\pgfqpoint{0.000000in}{0.000000in}}{%
\pgfpathmoveto{\pgfqpoint{0.000000in}{0.000000in}}%
\pgfpathlineto{\pgfqpoint{0.000000in}{-0.048611in}}%
\pgfusepath{stroke,fill}%
}%
\begin{pgfscope}%
\pgfsys@transformshift{4.328501in}{0.440000in}%
\pgfsys@useobject{currentmarker}{}%
\end{pgfscope}%
\end{pgfscope}%
\begin{pgfscope}%
\pgftext[x=4.328501in,y=0.342778in,,top]{\rmfamily\fontsize{10.000000}{12.000000}\selectfont \(\displaystyle 1.00\)}%
\end{pgfscope}%
\begin{pgfscope}%
\pgfsetbuttcap%
\pgfsetroundjoin%
\definecolor{currentfill}{rgb}{0.000000,0.000000,0.000000}%
\pgfsetfillcolor{currentfill}%
\pgfsetlinewidth{0.803000pt}%
\definecolor{currentstroke}{rgb}{0.000000,0.000000,0.000000}%
\pgfsetstrokecolor{currentstroke}%
\pgfsetdash{}{0pt}%
\pgfsys@defobject{currentmarker}{\pgfqpoint{0.000000in}{-0.048611in}}{\pgfqpoint{0.000000in}{0.000000in}}{%
\pgfpathmoveto{\pgfqpoint{0.000000in}{0.000000in}}%
\pgfpathlineto{\pgfqpoint{0.000000in}{-0.048611in}}%
\pgfusepath{stroke,fill}%
}%
\begin{pgfscope}%
\pgfsys@transformshift{5.090172in}{0.440000in}%
\pgfsys@useobject{currentmarker}{}%
\end{pgfscope}%
\end{pgfscope}%
\begin{pgfscope}%
\pgftext[x=5.090172in,y=0.342778in,,top]{\rmfamily\fontsize{10.000000}{12.000000}\selectfont \(\displaystyle 1.25\)}%
\end{pgfscope}%
\begin{pgfscope}%
\pgfsetbuttcap%
\pgfsetroundjoin%
\definecolor{currentfill}{rgb}{0.000000,0.000000,0.000000}%
\pgfsetfillcolor{currentfill}%
\pgfsetlinewidth{0.803000pt}%
\definecolor{currentstroke}{rgb}{0.000000,0.000000,0.000000}%
\pgfsetstrokecolor{currentstroke}%
\pgfsetdash{}{0pt}%
\pgfsys@defobject{currentmarker}{\pgfqpoint{0.000000in}{-0.048611in}}{\pgfqpoint{0.000000in}{0.000000in}}{%
\pgfpathmoveto{\pgfqpoint{0.000000in}{0.000000in}}%
\pgfpathlineto{\pgfqpoint{0.000000in}{-0.048611in}}%
\pgfusepath{stroke,fill}%
}%
\begin{pgfscope}%
\pgfsys@transformshift{5.851843in}{0.440000in}%
\pgfsys@useobject{currentmarker}{}%
\end{pgfscope}%
\end{pgfscope}%
\begin{pgfscope}%
\pgftext[x=5.851843in,y=0.342778in,,top]{\rmfamily\fontsize{10.000000}{12.000000}\selectfont \(\displaystyle 1.50\)}%
\end{pgfscope}%
\begin{pgfscope}%
\pgfsetbuttcap%
\pgfsetroundjoin%
\definecolor{currentfill}{rgb}{0.000000,0.000000,0.000000}%
\pgfsetfillcolor{currentfill}%
\pgfsetlinewidth{0.803000pt}%
\definecolor{currentstroke}{rgb}{0.000000,0.000000,0.000000}%
\pgfsetstrokecolor{currentstroke}%
\pgfsetdash{}{0pt}%
\pgfsys@defobject{currentmarker}{\pgfqpoint{0.000000in}{-0.048611in}}{\pgfqpoint{0.000000in}{0.000000in}}{%
\pgfpathmoveto{\pgfqpoint{0.000000in}{0.000000in}}%
\pgfpathlineto{\pgfqpoint{0.000000in}{-0.048611in}}%
\pgfusepath{stroke,fill}%
}%
\begin{pgfscope}%
\pgfsys@transformshift{6.613514in}{0.440000in}%
\pgfsys@useobject{currentmarker}{}%
\end{pgfscope}%
\end{pgfscope}%
\begin{pgfscope}%
\pgftext[x=6.613514in,y=0.342778in,,top]{\rmfamily\fontsize{10.000000}{12.000000}\selectfont \(\displaystyle 1.75\)}%
\end{pgfscope}%
\begin{pgfscope}%
\pgftext[x=4.100000in,y=0.164567in,,top]{\rmfamily\fontsize{10.000000}{12.000000}\selectfont Time (s)}%
\end{pgfscope}%
\begin{pgfscope}%
\pgfsetbuttcap%
\pgfsetroundjoin%
\definecolor{currentfill}{rgb}{0.000000,0.000000,0.000000}%
\pgfsetfillcolor{currentfill}%
\pgfsetlinewidth{0.803000pt}%
\definecolor{currentstroke}{rgb}{0.000000,0.000000,0.000000}%
\pgfsetstrokecolor{currentstroke}%
\pgfsetdash{}{0pt}%
\pgfsys@defobject{currentmarker}{\pgfqpoint{-0.048611in}{0.000000in}}{\pgfqpoint{0.000000in}{0.000000in}}{%
\pgfpathmoveto{\pgfqpoint{0.000000in}{0.000000in}}%
\pgfpathlineto{\pgfqpoint{-0.048611in}{0.000000in}}%
\pgfusepath{stroke,fill}%
}%
\begin{pgfscope}%
\pgfsys@transformshift{1.000000in}{0.747896in}%
\pgfsys@useobject{currentmarker}{}%
\end{pgfscope}%
\end{pgfscope}%
\begin{pgfscope}%
\pgftext[x=0.684030in,y=0.700068in,left,base]{\rmfamily\fontsize{10.000000}{12.000000}\selectfont \(\displaystyle 140\)}%
\end{pgfscope}%
\begin{pgfscope}%
\pgfsetbuttcap%
\pgfsetroundjoin%
\definecolor{currentfill}{rgb}{0.000000,0.000000,0.000000}%
\pgfsetfillcolor{currentfill}%
\pgfsetlinewidth{0.803000pt}%
\definecolor{currentstroke}{rgb}{0.000000,0.000000,0.000000}%
\pgfsetstrokecolor{currentstroke}%
\pgfsetdash{}{0pt}%
\pgfsys@defobject{currentmarker}{\pgfqpoint{-0.048611in}{0.000000in}}{\pgfqpoint{0.000000in}{0.000000in}}{%
\pgfpathmoveto{\pgfqpoint{0.000000in}{0.000000in}}%
\pgfpathlineto{\pgfqpoint{-0.048611in}{0.000000in}}%
\pgfusepath{stroke,fill}%
}%
\begin{pgfscope}%
\pgfsys@transformshift{1.000000in}{1.175589in}%
\pgfsys@useobject{currentmarker}{}%
\end{pgfscope}%
\end{pgfscope}%
\begin{pgfscope}%
\pgftext[x=0.684030in,y=1.127761in,left,base]{\rmfamily\fontsize{10.000000}{12.000000}\selectfont \(\displaystyle 160\)}%
\end{pgfscope}%
\begin{pgfscope}%
\pgfsetbuttcap%
\pgfsetroundjoin%
\definecolor{currentfill}{rgb}{0.000000,0.000000,0.000000}%
\pgfsetfillcolor{currentfill}%
\pgfsetlinewidth{0.803000pt}%
\definecolor{currentstroke}{rgb}{0.000000,0.000000,0.000000}%
\pgfsetstrokecolor{currentstroke}%
\pgfsetdash{}{0pt}%
\pgfsys@defobject{currentmarker}{\pgfqpoint{-0.048611in}{0.000000in}}{\pgfqpoint{0.000000in}{0.000000in}}{%
\pgfpathmoveto{\pgfqpoint{0.000000in}{0.000000in}}%
\pgfpathlineto{\pgfqpoint{-0.048611in}{0.000000in}}%
\pgfusepath{stroke,fill}%
}%
\begin{pgfscope}%
\pgfsys@transformshift{1.000000in}{1.603282in}%
\pgfsys@useobject{currentmarker}{}%
\end{pgfscope}%
\end{pgfscope}%
\begin{pgfscope}%
\pgftext[x=0.684030in,y=1.555454in,left,base]{\rmfamily\fontsize{10.000000}{12.000000}\selectfont \(\displaystyle 180\)}%
\end{pgfscope}%
\begin{pgfscope}%
\pgfsetbuttcap%
\pgfsetroundjoin%
\definecolor{currentfill}{rgb}{0.000000,0.000000,0.000000}%
\pgfsetfillcolor{currentfill}%
\pgfsetlinewidth{0.803000pt}%
\definecolor{currentstroke}{rgb}{0.000000,0.000000,0.000000}%
\pgfsetstrokecolor{currentstroke}%
\pgfsetdash{}{0pt}%
\pgfsys@defobject{currentmarker}{\pgfqpoint{-0.048611in}{0.000000in}}{\pgfqpoint{0.000000in}{0.000000in}}{%
\pgfpathmoveto{\pgfqpoint{0.000000in}{0.000000in}}%
\pgfpathlineto{\pgfqpoint{-0.048611in}{0.000000in}}%
\pgfusepath{stroke,fill}%
}%
\begin{pgfscope}%
\pgfsys@transformshift{1.000000in}{2.030975in}%
\pgfsys@useobject{currentmarker}{}%
\end{pgfscope}%
\end{pgfscope}%
\begin{pgfscope}%
\pgftext[x=0.684030in,y=1.983147in,left,base]{\rmfamily\fontsize{10.000000}{12.000000}\selectfont \(\displaystyle 200\)}%
\end{pgfscope}%
\begin{pgfscope}%
\pgfsetbuttcap%
\pgfsetroundjoin%
\definecolor{currentfill}{rgb}{0.000000,0.000000,0.000000}%
\pgfsetfillcolor{currentfill}%
\pgfsetlinewidth{0.803000pt}%
\definecolor{currentstroke}{rgb}{0.000000,0.000000,0.000000}%
\pgfsetstrokecolor{currentstroke}%
\pgfsetdash{}{0pt}%
\pgfsys@defobject{currentmarker}{\pgfqpoint{-0.048611in}{0.000000in}}{\pgfqpoint{0.000000in}{0.000000in}}{%
\pgfpathmoveto{\pgfqpoint{0.000000in}{0.000000in}}%
\pgfpathlineto{\pgfqpoint{-0.048611in}{0.000000in}}%
\pgfusepath{stroke,fill}%
}%
\begin{pgfscope}%
\pgfsys@transformshift{1.000000in}{2.458668in}%
\pgfsys@useobject{currentmarker}{}%
\end{pgfscope}%
\end{pgfscope}%
\begin{pgfscope}%
\pgftext[x=0.684030in,y=2.410840in,left,base]{\rmfamily\fontsize{10.000000}{12.000000}\selectfont \(\displaystyle 220\)}%
\end{pgfscope}%
\begin{pgfscope}%
\pgfsetbuttcap%
\pgfsetroundjoin%
\definecolor{currentfill}{rgb}{0.000000,0.000000,0.000000}%
\pgfsetfillcolor{currentfill}%
\pgfsetlinewidth{0.803000pt}%
\definecolor{currentstroke}{rgb}{0.000000,0.000000,0.000000}%
\pgfsetstrokecolor{currentstroke}%
\pgfsetdash{}{0pt}%
\pgfsys@defobject{currentmarker}{\pgfqpoint{-0.048611in}{0.000000in}}{\pgfqpoint{0.000000in}{0.000000in}}{%
\pgfpathmoveto{\pgfqpoint{0.000000in}{0.000000in}}%
\pgfpathlineto{\pgfqpoint{-0.048611in}{0.000000in}}%
\pgfusepath{stroke,fill}%
}%
\begin{pgfscope}%
\pgfsys@transformshift{1.000000in}{2.886361in}%
\pgfsys@useobject{currentmarker}{}%
\end{pgfscope}%
\end{pgfscope}%
\begin{pgfscope}%
\pgftext[x=0.684030in,y=2.838533in,left,base]{\rmfamily\fontsize{10.000000}{12.000000}\selectfont \(\displaystyle 240\)}%
\end{pgfscope}%
\begin{pgfscope}%
\pgfsetbuttcap%
\pgfsetroundjoin%
\definecolor{currentfill}{rgb}{0.000000,0.000000,0.000000}%
\pgfsetfillcolor{currentfill}%
\pgfsetlinewidth{0.803000pt}%
\definecolor{currentstroke}{rgb}{0.000000,0.000000,0.000000}%
\pgfsetstrokecolor{currentstroke}%
\pgfsetdash{}{0pt}%
\pgfsys@defobject{currentmarker}{\pgfqpoint{-0.048611in}{0.000000in}}{\pgfqpoint{0.000000in}{0.000000in}}{%
\pgfpathmoveto{\pgfqpoint{0.000000in}{0.000000in}}%
\pgfpathlineto{\pgfqpoint{-0.048611in}{0.000000in}}%
\pgfusepath{stroke,fill}%
}%
\begin{pgfscope}%
\pgfsys@transformshift{1.000000in}{3.314054in}%
\pgfsys@useobject{currentmarker}{}%
\end{pgfscope}%
\end{pgfscope}%
\begin{pgfscope}%
\pgftext[x=0.684030in,y=3.266226in,left,base]{\rmfamily\fontsize{10.000000}{12.000000}\selectfont \(\displaystyle 260\)}%
\end{pgfscope}%
\begin{pgfscope}%
\pgftext[x=0.628474in,y=1.980000in,,bottom,rotate=90.000000]{\rmfamily\fontsize{10.000000}{12.000000}\selectfont Frequency (Hz)}%
\end{pgfscope}%
\begin{pgfscope}%
\pgfpathrectangle{\pgfqpoint{1.000000in}{0.440000in}}{\pgfqpoint{6.200000in}{3.080000in}} %
\pgfusepath{clip}%
\pgfsetrectcap%
\pgfsetroundjoin%
\pgfsetlinewidth{0.501875pt}%
\definecolor{currentstroke}{rgb}{0.800000,0.000000,0.266667}%
\pgfsetstrokecolor{currentstroke}%
\pgfsetdash{}{0pt}%
\pgfpathmoveto{\pgfqpoint{1.769287in}{0.440000in}}%
\pgfpathlineto{\pgfqpoint{1.769287in}{3.520000in}}%
\pgfusepath{stroke}%
\end{pgfscope}%
\begin{pgfscope}%
\pgfpathrectangle{\pgfqpoint{1.000000in}{0.440000in}}{\pgfqpoint{6.200000in}{3.080000in}} %
\pgfusepath{clip}%
\pgfsetrectcap%
\pgfsetroundjoin%
\pgfsetlinewidth{0.501875pt}%
\definecolor{currentstroke}{rgb}{0.800000,0.000000,0.266667}%
\pgfsetstrokecolor{currentstroke}%
\pgfsetdash{}{0pt}%
\pgfpathmoveto{\pgfqpoint{2.165356in}{0.440000in}}%
\pgfpathlineto{\pgfqpoint{2.165356in}{3.520000in}}%
\pgfusepath{stroke}%
\end{pgfscope}%
\begin{pgfscope}%
\pgfpathrectangle{\pgfqpoint{1.000000in}{0.440000in}}{\pgfqpoint{6.200000in}{3.080000in}} %
\pgfusepath{clip}%
\pgfsetrectcap%
\pgfsetroundjoin%
\pgfsetlinewidth{0.501875pt}%
\definecolor{currentstroke}{rgb}{0.501961,0.501961,0.501961}%
\pgfsetstrokecolor{currentstroke}%
\pgfsetdash{}{0pt}%
\pgfpathmoveto{\pgfqpoint{1.967322in}{0.440000in}}%
\pgfpathlineto{\pgfqpoint{1.967322in}{3.520000in}}%
\pgfusepath{stroke}%
\end{pgfscope}%
\begin{pgfscope}%
\pgfpathrectangle{\pgfqpoint{1.000000in}{0.440000in}}{\pgfqpoint{6.200000in}{3.080000in}} %
\pgfusepath{clip}%
\pgfsetrectcap%
\pgfsetroundjoin%
\pgfsetlinewidth{0.501875pt}%
\definecolor{currentstroke}{rgb}{0.800000,0.000000,0.266667}%
\pgfsetstrokecolor{currentstroke}%
\pgfsetdash{}{0pt}%
\pgfpathmoveto{\pgfqpoint{2.165356in}{0.440000in}}%
\pgfpathlineto{\pgfqpoint{2.165356in}{3.520000in}}%
\pgfusepath{stroke}%
\end{pgfscope}%
\begin{pgfscope}%
\pgfpathrectangle{\pgfqpoint{1.000000in}{0.440000in}}{\pgfqpoint{6.200000in}{3.080000in}} %
\pgfusepath{clip}%
\pgfsetrectcap%
\pgfsetroundjoin%
\pgfsetlinewidth{0.501875pt}%
\definecolor{currentstroke}{rgb}{0.800000,0.000000,0.266667}%
\pgfsetstrokecolor{currentstroke}%
\pgfsetdash{}{0pt}%
\pgfpathmoveto{\pgfqpoint{2.622359in}{0.440000in}}%
\pgfpathlineto{\pgfqpoint{2.622359in}{3.520000in}}%
\pgfusepath{stroke}%
\end{pgfscope}%
\begin{pgfscope}%
\pgfpathrectangle{\pgfqpoint{1.000000in}{0.440000in}}{\pgfqpoint{6.200000in}{3.080000in}} %
\pgfusepath{clip}%
\pgfsetrectcap%
\pgfsetroundjoin%
\pgfsetlinewidth{0.501875pt}%
\definecolor{currentstroke}{rgb}{0.501961,0.501961,0.501961}%
\pgfsetstrokecolor{currentstroke}%
\pgfsetdash{}{0pt}%
\pgfpathmoveto{\pgfqpoint{2.393857in}{0.440000in}}%
\pgfpathlineto{\pgfqpoint{2.393857in}{3.520000in}}%
\pgfusepath{stroke}%
\end{pgfscope}%
\begin{pgfscope}%
\pgfpathrectangle{\pgfqpoint{1.000000in}{0.440000in}}{\pgfqpoint{6.200000in}{3.080000in}} %
\pgfusepath{clip}%
\pgfsetrectcap%
\pgfsetroundjoin%
\pgfsetlinewidth{0.501875pt}%
\definecolor{currentstroke}{rgb}{0.800000,0.000000,0.266667}%
\pgfsetstrokecolor{currentstroke}%
\pgfsetdash{}{0pt}%
\pgfpathmoveto{\pgfqpoint{2.622359in}{0.440000in}}%
\pgfpathlineto{\pgfqpoint{2.622359in}{3.520000in}}%
\pgfusepath{stroke}%
\end{pgfscope}%
\begin{pgfscope}%
\pgfpathrectangle{\pgfqpoint{1.000000in}{0.440000in}}{\pgfqpoint{6.200000in}{3.080000in}} %
\pgfusepath{clip}%
\pgfsetrectcap%
\pgfsetroundjoin%
\pgfsetlinewidth{0.501875pt}%
\definecolor{currentstroke}{rgb}{0.800000,0.000000,0.266667}%
\pgfsetstrokecolor{currentstroke}%
\pgfsetdash{}{0pt}%
\pgfpathmoveto{\pgfqpoint{3.444963in}{0.440000in}}%
\pgfpathlineto{\pgfqpoint{3.444963in}{3.520000in}}%
\pgfusepath{stroke}%
\end{pgfscope}%
\begin{pgfscope}%
\pgfpathrectangle{\pgfqpoint{1.000000in}{0.440000in}}{\pgfqpoint{6.200000in}{3.080000in}} %
\pgfusepath{clip}%
\pgfsetrectcap%
\pgfsetroundjoin%
\pgfsetlinewidth{0.501875pt}%
\definecolor{currentstroke}{rgb}{0.501961,0.501961,0.501961}%
\pgfsetstrokecolor{currentstroke}%
\pgfsetdash{}{0pt}%
\pgfpathmoveto{\pgfqpoint{3.033661in}{0.440000in}}%
\pgfpathlineto{\pgfqpoint{3.033661in}{3.520000in}}%
\pgfusepath{stroke}%
\end{pgfscope}%
\begin{pgfscope}%
\pgfpathrectangle{\pgfqpoint{1.000000in}{0.440000in}}{\pgfqpoint{6.200000in}{3.080000in}} %
\pgfusepath{clip}%
\pgfsetrectcap%
\pgfsetroundjoin%
\pgfsetlinewidth{0.501875pt}%
\definecolor{currentstroke}{rgb}{0.800000,0.000000,0.266667}%
\pgfsetstrokecolor{currentstroke}%
\pgfsetdash{}{0pt}%
\pgfpathmoveto{\pgfqpoint{3.444963in}{0.440000in}}%
\pgfpathlineto{\pgfqpoint{3.444963in}{3.520000in}}%
\pgfusepath{stroke}%
\end{pgfscope}%
\begin{pgfscope}%
\pgfpathrectangle{\pgfqpoint{1.000000in}{0.440000in}}{\pgfqpoint{6.200000in}{3.080000in}} %
\pgfusepath{clip}%
\pgfsetrectcap%
\pgfsetroundjoin%
\pgfsetlinewidth{0.501875pt}%
\definecolor{currentstroke}{rgb}{0.800000,0.000000,0.266667}%
\pgfsetstrokecolor{currentstroke}%
\pgfsetdash{}{0pt}%
\pgfpathmoveto{\pgfqpoint{3.901966in}{0.440000in}}%
\pgfpathlineto{\pgfqpoint{3.901966in}{3.520000in}}%
\pgfusepath{stroke}%
\end{pgfscope}%
\begin{pgfscope}%
\pgfpathrectangle{\pgfqpoint{1.000000in}{0.440000in}}{\pgfqpoint{6.200000in}{3.080000in}} %
\pgfusepath{clip}%
\pgfsetrectcap%
\pgfsetroundjoin%
\pgfsetlinewidth{0.501875pt}%
\definecolor{currentstroke}{rgb}{0.501961,0.501961,0.501961}%
\pgfsetstrokecolor{currentstroke}%
\pgfsetdash{}{0pt}%
\pgfpathmoveto{\pgfqpoint{3.673464in}{0.440000in}}%
\pgfpathlineto{\pgfqpoint{3.673464in}{3.520000in}}%
\pgfusepath{stroke}%
\end{pgfscope}%
\begin{pgfscope}%
\pgfpathrectangle{\pgfqpoint{1.000000in}{0.440000in}}{\pgfqpoint{6.200000in}{3.080000in}} %
\pgfusepath{clip}%
\pgfsetrectcap%
\pgfsetroundjoin%
\pgfsetlinewidth{0.501875pt}%
\definecolor{currentstroke}{rgb}{0.800000,0.000000,0.266667}%
\pgfsetstrokecolor{currentstroke}%
\pgfsetdash{}{0pt}%
\pgfpathmoveto{\pgfqpoint{3.901966in}{0.440000in}}%
\pgfpathlineto{\pgfqpoint{3.901966in}{3.520000in}}%
\pgfusepath{stroke}%
\end{pgfscope}%
\begin{pgfscope}%
\pgfpathrectangle{\pgfqpoint{1.000000in}{0.440000in}}{\pgfqpoint{6.200000in}{3.080000in}} %
\pgfusepath{clip}%
\pgfsetrectcap%
\pgfsetroundjoin%
\pgfsetlinewidth{0.501875pt}%
\definecolor{currentstroke}{rgb}{0.800000,0.000000,0.266667}%
\pgfsetstrokecolor{currentstroke}%
\pgfsetdash{}{0pt}%
\pgfpathmoveto{\pgfqpoint{5.059705in}{0.440000in}}%
\pgfpathlineto{\pgfqpoint{5.059705in}{3.520000in}}%
\pgfusepath{stroke}%
\end{pgfscope}%
\begin{pgfscope}%
\pgfpathrectangle{\pgfqpoint{1.000000in}{0.440000in}}{\pgfqpoint{6.200000in}{3.080000in}} %
\pgfusepath{clip}%
\pgfsetrectcap%
\pgfsetroundjoin%
\pgfsetlinewidth{0.501875pt}%
\definecolor{currentstroke}{rgb}{0.501961,0.501961,0.501961}%
\pgfsetstrokecolor{currentstroke}%
\pgfsetdash{}{0pt}%
\pgfpathmoveto{\pgfqpoint{4.480835in}{0.440000in}}%
\pgfpathlineto{\pgfqpoint{4.480835in}{3.520000in}}%
\pgfusepath{stroke}%
\end{pgfscope}%
\begin{pgfscope}%
\pgfpathrectangle{\pgfqpoint{1.000000in}{0.440000in}}{\pgfqpoint{6.200000in}{3.080000in}} %
\pgfusepath{clip}%
\pgfsetrectcap%
\pgfsetroundjoin%
\pgfsetlinewidth{0.501875pt}%
\definecolor{currentstroke}{rgb}{0.800000,0.000000,0.266667}%
\pgfsetstrokecolor{currentstroke}%
\pgfsetdash{}{0pt}%
\pgfpathmoveto{\pgfqpoint{5.059705in}{0.440000in}}%
\pgfpathlineto{\pgfqpoint{5.059705in}{3.520000in}}%
\pgfusepath{stroke}%
\end{pgfscope}%
\begin{pgfscope}%
\pgfpathrectangle{\pgfqpoint{1.000000in}{0.440000in}}{\pgfqpoint{6.200000in}{3.080000in}} %
\pgfusepath{clip}%
\pgfsetrectcap%
\pgfsetroundjoin%
\pgfsetlinewidth{0.501875pt}%
\definecolor{currentstroke}{rgb}{0.800000,0.000000,0.266667}%
\pgfsetstrokecolor{currentstroke}%
\pgfsetdash{}{0pt}%
\pgfpathmoveto{\pgfqpoint{6.765848in}{0.440000in}}%
\pgfpathlineto{\pgfqpoint{6.765848in}{3.520000in}}%
\pgfusepath{stroke}%
\end{pgfscope}%
\begin{pgfscope}%
\pgfpathrectangle{\pgfqpoint{1.000000in}{0.440000in}}{\pgfqpoint{6.200000in}{3.080000in}} %
\pgfusepath{clip}%
\pgfsetrectcap%
\pgfsetroundjoin%
\pgfsetlinewidth{0.501875pt}%
\definecolor{currentstroke}{rgb}{0.501961,0.501961,0.501961}%
\pgfsetstrokecolor{currentstroke}%
\pgfsetdash{}{0pt}%
\pgfpathmoveto{\pgfqpoint{5.912776in}{0.440000in}}%
\pgfpathlineto{\pgfqpoint{5.912776in}{3.520000in}}%
\pgfusepath{stroke}%
\end{pgfscope}%
\begin{pgfscope}%
\pgfpathrectangle{\pgfqpoint{1.000000in}{0.440000in}}{\pgfqpoint{6.200000in}{3.080000in}} %
\pgfusepath{clip}%
\pgfsetrectcap%
\pgfsetroundjoin%
\pgfsetlinewidth{0.501875pt}%
\definecolor{currentstroke}{rgb}{0.800000,0.000000,0.266667}%
\pgfsetstrokecolor{currentstroke}%
\pgfsetdash{}{0pt}%
\pgfpathmoveto{\pgfqpoint{6.765848in}{0.440000in}}%
\pgfpathlineto{\pgfqpoint{6.765848in}{3.520000in}}%
\pgfusepath{stroke}%
\end{pgfscope}%
\begin{pgfscope}%
\pgfpathrectangle{\pgfqpoint{1.000000in}{0.440000in}}{\pgfqpoint{6.200000in}{3.080000in}} %
\pgfusepath{clip}%
\pgfsetrectcap%
\pgfsetroundjoin%
\pgfsetlinewidth{1.505625pt}%
\definecolor{currentstroke}{rgb}{0.364706,0.517647,0.572549}%
\pgfsetstrokecolor{currentstroke}%
\pgfsetdash{}{0pt}%
\pgfpathmoveto{\pgfqpoint{1.281818in}{1.169484in}}%
\pgfpathlineto{\pgfqpoint{1.997789in}{1.169484in}}%
\pgfpathlineto{\pgfqpoint{2.013022in}{1.180018in}}%
\pgfpathlineto{\pgfqpoint{2.028256in}{1.213117in}}%
\pgfpathlineto{\pgfqpoint{2.043489in}{1.243508in}}%
\pgfpathlineto{\pgfqpoint{2.058722in}{1.249626in}}%
\pgfpathlineto{\pgfqpoint{2.073956in}{1.241922in}}%
\pgfpathlineto{\pgfqpoint{2.104423in}{1.216067in}}%
\pgfpathlineto{\pgfqpoint{2.119656in}{1.180239in}}%
\pgfpathlineto{\pgfqpoint{2.134889in}{1.096283in}}%
\pgfpathlineto{\pgfqpoint{2.150123in}{1.000565in}}%
\pgfpathlineto{\pgfqpoint{2.165356in}{0.981155in}}%
\pgfpathlineto{\pgfqpoint{2.180590in}{1.022198in}}%
\pgfpathlineto{\pgfqpoint{2.195823in}{1.051990in}}%
\pgfpathlineto{\pgfqpoint{2.211057in}{1.073655in}}%
\pgfpathlineto{\pgfqpoint{2.226290in}{1.117670in}}%
\pgfpathlineto{\pgfqpoint{2.241523in}{1.190799in}}%
\pgfpathlineto{\pgfqpoint{2.256757in}{1.283890in}}%
\pgfpathlineto{\pgfqpoint{2.287224in}{1.477771in}}%
\pgfpathlineto{\pgfqpoint{2.302457in}{1.553343in}}%
\pgfpathlineto{\pgfqpoint{2.317690in}{1.610443in}}%
\pgfpathlineto{\pgfqpoint{2.332924in}{1.660888in}}%
\pgfpathlineto{\pgfqpoint{2.348157in}{1.726165in}}%
\pgfpathlineto{\pgfqpoint{2.363391in}{1.855295in}}%
\pgfpathlineto{\pgfqpoint{2.378624in}{2.077817in}}%
\pgfpathlineto{\pgfqpoint{2.393857in}{2.280446in}}%
\pgfpathlineto{\pgfqpoint{2.409091in}{2.393129in}}%
\pgfpathlineto{\pgfqpoint{2.439558in}{2.525077in}}%
\pgfpathlineto{\pgfqpoint{2.454791in}{2.620606in}}%
\pgfpathlineto{\pgfqpoint{2.470025in}{2.727577in}}%
\pgfpathlineto{\pgfqpoint{2.485258in}{2.823790in}}%
\pgfpathlineto{\pgfqpoint{2.500491in}{2.910174in}}%
\pgfpathlineto{\pgfqpoint{2.530958in}{3.061822in}}%
\pgfpathlineto{\pgfqpoint{2.561425in}{3.243144in}}%
\pgfpathlineto{\pgfqpoint{2.576658in}{3.315514in}}%
\pgfpathlineto{\pgfqpoint{2.591892in}{3.372327in}}%
\pgfpathlineto{\pgfqpoint{2.607125in}{3.380000in}}%
\pgfpathlineto{\pgfqpoint{2.622359in}{3.283246in}}%
\pgfpathlineto{\pgfqpoint{2.637592in}{3.165543in}}%
\pgfpathlineto{\pgfqpoint{2.652826in}{3.121637in}}%
\pgfpathlineto{\pgfqpoint{2.698526in}{3.099111in}}%
\pgfpathlineto{\pgfqpoint{2.713759in}{3.086715in}}%
\pgfpathlineto{\pgfqpoint{2.774693in}{2.995830in}}%
\pgfpathlineto{\pgfqpoint{2.835627in}{2.850865in}}%
\pgfpathlineto{\pgfqpoint{2.896560in}{2.655827in}}%
\pgfpathlineto{\pgfqpoint{2.927027in}{2.538655in}}%
\pgfpathlineto{\pgfqpoint{2.972727in}{2.360070in}}%
\pgfpathlineto{\pgfqpoint{3.018428in}{2.189141in}}%
\pgfpathlineto{\pgfqpoint{3.033661in}{2.137035in}}%
\pgfpathlineto{\pgfqpoint{3.094595in}{1.969662in}}%
\pgfpathlineto{\pgfqpoint{3.155528in}{1.853076in}}%
\pgfpathlineto{\pgfqpoint{3.201229in}{1.801005in}}%
\pgfpathlineto{\pgfqpoint{3.216462in}{1.783924in}}%
\pgfpathlineto{\pgfqpoint{3.231695in}{1.775219in}}%
\pgfpathlineto{\pgfqpoint{3.277396in}{1.758339in}}%
\pgfpathlineto{\pgfqpoint{3.292629in}{1.722644in}}%
\pgfpathlineto{\pgfqpoint{3.307862in}{1.673207in}}%
\pgfpathlineto{\pgfqpoint{3.323096in}{1.630925in}}%
\pgfpathlineto{\pgfqpoint{3.338329in}{1.596906in}}%
\pgfpathlineto{\pgfqpoint{3.353563in}{1.570437in}}%
\pgfpathlineto{\pgfqpoint{3.384029in}{1.526751in}}%
\pgfpathlineto{\pgfqpoint{3.399263in}{1.498475in}}%
\pgfpathlineto{\pgfqpoint{3.429730in}{1.406330in}}%
\pgfpathlineto{\pgfqpoint{3.444963in}{1.392157in}}%
\pgfpathlineto{\pgfqpoint{3.460197in}{1.388183in}}%
\pgfpathlineto{\pgfqpoint{3.505897in}{1.337844in}}%
\pgfpathlineto{\pgfqpoint{3.536364in}{1.315695in}}%
\pgfpathlineto{\pgfqpoint{3.551597in}{1.306303in}}%
\pgfpathlineto{\pgfqpoint{3.566830in}{1.309460in}}%
\pgfpathlineto{\pgfqpoint{3.582064in}{1.356383in}}%
\pgfpathlineto{\pgfqpoint{3.597297in}{1.442296in}}%
\pgfpathlineto{\pgfqpoint{3.612531in}{1.498590in}}%
\pgfpathlineto{\pgfqpoint{3.627764in}{1.509395in}}%
\pgfpathlineto{\pgfqpoint{3.642998in}{1.504952in}}%
\pgfpathlineto{\pgfqpoint{3.658231in}{1.497308in}}%
\pgfpathlineto{\pgfqpoint{3.673464in}{1.491746in}}%
\pgfpathlineto{\pgfqpoint{3.703931in}{1.484544in}}%
\pgfpathlineto{\pgfqpoint{3.719165in}{1.485587in}}%
\pgfpathlineto{\pgfqpoint{3.734398in}{1.492379in}}%
\pgfpathlineto{\pgfqpoint{3.749631in}{1.497810in}}%
\pgfpathlineto{\pgfqpoint{3.764865in}{1.495599in}}%
\pgfpathlineto{\pgfqpoint{3.780098in}{1.488639in}}%
\pgfpathlineto{\pgfqpoint{3.795332in}{1.479567in}}%
\pgfpathlineto{\pgfqpoint{3.810565in}{1.462231in}}%
\pgfpathlineto{\pgfqpoint{3.825799in}{1.425528in}}%
\pgfpathlineto{\pgfqpoint{3.841032in}{1.367895in}}%
\pgfpathlineto{\pgfqpoint{3.856265in}{1.338152in}}%
\pgfpathlineto{\pgfqpoint{3.871499in}{1.346327in}}%
\pgfpathlineto{\pgfqpoint{3.886732in}{1.336390in}}%
\pgfpathlineto{\pgfqpoint{3.917199in}{1.267346in}}%
\pgfpathlineto{\pgfqpoint{3.947666in}{1.212521in}}%
\pgfpathlineto{\pgfqpoint{3.962899in}{1.178892in}}%
\pgfpathlineto{\pgfqpoint{3.993366in}{1.100211in}}%
\pgfpathlineto{\pgfqpoint{4.008600in}{1.060065in}}%
\pgfpathlineto{\pgfqpoint{4.023833in}{1.024709in}}%
\pgfpathlineto{\pgfqpoint{4.039066in}{0.999380in}}%
\pgfpathlineto{\pgfqpoint{4.054300in}{0.996954in}}%
\pgfpathlineto{\pgfqpoint{4.084767in}{1.071116in}}%
\pgfpathlineto{\pgfqpoint{4.100000in}{1.068886in}}%
\pgfpathlineto{\pgfqpoint{4.145700in}{1.001995in}}%
\pgfpathlineto{\pgfqpoint{4.160934in}{0.973546in}}%
\pgfpathlineto{\pgfqpoint{4.176167in}{0.947798in}}%
\pgfpathlineto{\pgfqpoint{4.191400in}{0.931365in}}%
\pgfpathlineto{\pgfqpoint{4.206634in}{0.924262in}}%
\pgfpathlineto{\pgfqpoint{4.221867in}{0.925338in}}%
\pgfpathlineto{\pgfqpoint{4.237101in}{0.938405in}}%
\pgfpathlineto{\pgfqpoint{4.252334in}{0.961615in}}%
\pgfpathlineto{\pgfqpoint{4.267568in}{0.991227in}}%
\pgfpathlineto{\pgfqpoint{4.282801in}{1.028989in}}%
\pgfpathlineto{\pgfqpoint{4.298034in}{1.062745in}}%
\pgfpathlineto{\pgfqpoint{4.313268in}{1.084881in}}%
\pgfpathlineto{\pgfqpoint{4.343735in}{1.222592in}}%
\pgfpathlineto{\pgfqpoint{4.358968in}{1.280546in}}%
\pgfpathlineto{\pgfqpoint{4.374201in}{1.343299in}}%
\pgfpathlineto{\pgfqpoint{4.404668in}{1.409018in}}%
\pgfpathlineto{\pgfqpoint{4.419902in}{1.447560in}}%
\pgfpathlineto{\pgfqpoint{4.435135in}{1.462780in}}%
\pgfpathlineto{\pgfqpoint{4.450369in}{1.459128in}}%
\pgfpathlineto{\pgfqpoint{4.465602in}{1.451235in}}%
\pgfpathlineto{\pgfqpoint{4.480835in}{1.446877in}}%
\pgfpathlineto{\pgfqpoint{4.572236in}{1.439168in}}%
\pgfpathlineto{\pgfqpoint{4.663636in}{1.416321in}}%
\pgfpathlineto{\pgfqpoint{4.678870in}{1.411657in}}%
\pgfpathlineto{\pgfqpoint{4.770270in}{1.373638in}}%
\pgfpathlineto{\pgfqpoint{4.800737in}{1.356787in}}%
\pgfpathlineto{\pgfqpoint{4.876904in}{1.312738in}}%
\pgfpathlineto{\pgfqpoint{5.059705in}{1.181455in}}%
\pgfpathlineto{\pgfqpoint{5.074939in}{1.170636in}}%
\pgfpathlineto{\pgfqpoint{5.166339in}{1.120895in}}%
\pgfpathlineto{\pgfqpoint{5.181572in}{1.113443in}}%
\pgfpathlineto{\pgfqpoint{5.272973in}{1.078524in}}%
\pgfpathlineto{\pgfqpoint{5.303440in}{1.070970in}}%
\pgfpathlineto{\pgfqpoint{5.379607in}{1.053974in}}%
\pgfpathlineto{\pgfqpoint{5.471007in}{1.047085in}}%
\pgfpathlineto{\pgfqpoint{5.486241in}{1.038835in}}%
\pgfpathlineto{\pgfqpoint{5.516708in}{1.017903in}}%
\pgfpathlineto{\pgfqpoint{5.547174in}{0.987236in}}%
\pgfpathlineto{\pgfqpoint{5.562408in}{0.972718in}}%
\pgfpathlineto{\pgfqpoint{5.592875in}{0.922519in}}%
\pgfpathlineto{\pgfqpoint{5.608108in}{0.905856in}}%
\pgfpathlineto{\pgfqpoint{5.623342in}{0.870672in}}%
\pgfpathlineto{\pgfqpoint{5.638575in}{0.823965in}}%
\pgfpathlineto{\pgfqpoint{5.653808in}{0.794213in}}%
\pgfpathlineto{\pgfqpoint{5.684275in}{0.761011in}}%
\pgfpathlineto{\pgfqpoint{5.714742in}{0.673842in}}%
\pgfpathlineto{\pgfqpoint{5.745209in}{0.632286in}}%
\pgfpathlineto{\pgfqpoint{5.760442in}{0.605600in}}%
\pgfpathlineto{\pgfqpoint{5.775676in}{0.591469in}}%
\pgfpathlineto{\pgfqpoint{5.790909in}{0.607326in}}%
\pgfpathlineto{\pgfqpoint{5.806143in}{0.626000in}}%
\pgfpathlineto{\pgfqpoint{5.821376in}{0.632378in}}%
\pgfpathlineto{\pgfqpoint{5.836609in}{0.640955in}}%
\pgfpathlineto{\pgfqpoint{5.851843in}{0.642687in}}%
\pgfpathlineto{\pgfqpoint{5.867076in}{0.627918in}}%
\pgfpathlineto{\pgfqpoint{5.897543in}{0.580000in}}%
\pgfpathlineto{\pgfqpoint{5.912776in}{0.588304in}}%
\pgfpathlineto{\pgfqpoint{5.928010in}{0.629685in}}%
\pgfpathlineto{\pgfqpoint{5.943243in}{0.631689in}}%
\pgfpathlineto{\pgfqpoint{5.958477in}{0.607162in}}%
\pgfpathlineto{\pgfqpoint{5.973710in}{0.608787in}}%
\pgfpathlineto{\pgfqpoint{5.988943in}{0.619218in}}%
\pgfpathlineto{\pgfqpoint{6.004177in}{0.619497in}}%
\pgfpathlineto{\pgfqpoint{6.019410in}{0.622254in}}%
\pgfpathlineto{\pgfqpoint{6.034644in}{0.631110in}}%
\pgfpathlineto{\pgfqpoint{6.065111in}{0.642581in}}%
\pgfpathlineto{\pgfqpoint{6.080344in}{0.658514in}}%
\pgfpathlineto{\pgfqpoint{6.095577in}{0.684377in}}%
\pgfpathlineto{\pgfqpoint{6.110811in}{0.722313in}}%
\pgfpathlineto{\pgfqpoint{6.126044in}{0.772044in}}%
\pgfpathlineto{\pgfqpoint{6.141278in}{0.812010in}}%
\pgfpathlineto{\pgfqpoint{6.156511in}{0.821943in}}%
\pgfpathlineto{\pgfqpoint{6.171744in}{0.811626in}}%
\pgfpathlineto{\pgfqpoint{6.186978in}{0.796743in}}%
\pgfpathlineto{\pgfqpoint{6.202211in}{0.789968in}}%
\pgfpathlineto{\pgfqpoint{6.918182in}{0.789968in}}%
\pgfpathlineto{\pgfqpoint{6.918182in}{0.789968in}}%
\pgfusepath{stroke}%
\end{pgfscope}%
\begin{pgfscope}%
\pgfpathrectangle{\pgfqpoint{1.000000in}{0.440000in}}{\pgfqpoint{6.200000in}{3.080000in}} %
\pgfusepath{clip}%
\pgfsetbuttcap%
\pgfsetroundjoin%
\definecolor{currentfill}{rgb}{0.364706,0.517647,0.572549}%
\pgfsetfillcolor{currentfill}%
\pgfsetlinewidth{1.003750pt}%
\definecolor{currentstroke}{rgb}{0.364706,0.517647,0.572549}%
\pgfsetstrokecolor{currentstroke}%
\pgfsetdash{}{0pt}%
\pgfsys@defobject{currentmarker}{\pgfqpoint{-0.027778in}{-0.027778in}}{\pgfqpoint{0.027778in}{0.027778in}}{%
\pgfpathmoveto{\pgfqpoint{0.000000in}{-0.027778in}}%
\pgfpathcurveto{\pgfqpoint{0.007367in}{-0.027778in}}{\pgfqpoint{0.014433in}{-0.024851in}}{\pgfqpoint{0.019642in}{-0.019642in}}%
\pgfpathcurveto{\pgfqpoint{0.024851in}{-0.014433in}}{\pgfqpoint{0.027778in}{-0.007367in}}{\pgfqpoint{0.027778in}{0.000000in}}%
\pgfpathcurveto{\pgfqpoint{0.027778in}{0.007367in}}{\pgfqpoint{0.024851in}{0.014433in}}{\pgfqpoint{0.019642in}{0.019642in}}%
\pgfpathcurveto{\pgfqpoint{0.014433in}{0.024851in}}{\pgfqpoint{0.007367in}{0.027778in}}{\pgfqpoint{0.000000in}{0.027778in}}%
\pgfpathcurveto{\pgfqpoint{-0.007367in}{0.027778in}}{\pgfqpoint{-0.014433in}{0.024851in}}{\pgfqpoint{-0.019642in}{0.019642in}}%
\pgfpathcurveto{\pgfqpoint{-0.024851in}{0.014433in}}{\pgfqpoint{-0.027778in}{0.007367in}}{\pgfqpoint{-0.027778in}{0.000000in}}%
\pgfpathcurveto{\pgfqpoint{-0.027778in}{-0.007367in}}{\pgfqpoint{-0.024851in}{-0.014433in}}{\pgfqpoint{-0.019642in}{-0.019642in}}%
\pgfpathcurveto{\pgfqpoint{-0.014433in}{-0.024851in}}{\pgfqpoint{-0.007367in}{-0.027778in}}{\pgfqpoint{0.000000in}{-0.027778in}}%
\pgfpathclose%
\pgfusepath{stroke,fill}%
}%
\begin{pgfscope}%
\pgfsys@transformshift{1.769287in}{1.169484in}%
\pgfsys@useobject{currentmarker}{}%
\end{pgfscope}%
\begin{pgfscope}%
\pgfsys@transformshift{1.967322in}{1.169484in}%
\pgfsys@useobject{currentmarker}{}%
\end{pgfscope}%
\begin{pgfscope}%
\pgfsys@transformshift{2.165356in}{1.000565in}%
\pgfsys@useobject{currentmarker}{}%
\end{pgfscope}%
\begin{pgfscope}%
\pgfsys@transformshift{2.393857in}{2.280446in}%
\pgfsys@useobject{currentmarker}{}%
\end{pgfscope}%
\begin{pgfscope}%
\pgfsys@transformshift{2.622359in}{3.283246in}%
\pgfsys@useobject{currentmarker}{}%
\end{pgfscope}%
\begin{pgfscope}%
\pgfsys@transformshift{3.033661in}{2.189141in}%
\pgfsys@useobject{currentmarker}{}%
\end{pgfscope}%
\begin{pgfscope}%
\pgfsys@transformshift{3.444963in}{1.392157in}%
\pgfsys@useobject{currentmarker}{}%
\end{pgfscope}%
\begin{pgfscope}%
\pgfsys@transformshift{3.673464in}{1.497308in}%
\pgfsys@useobject{currentmarker}{}%
\end{pgfscope}%
\begin{pgfscope}%
\pgfsys@transformshift{3.901966in}{1.300541in}%
\pgfsys@useobject{currentmarker}{}%
\end{pgfscope}%
\begin{pgfscope}%
\pgfsys@transformshift{4.480835in}{1.446877in}%
\pgfsys@useobject{currentmarker}{}%
\end{pgfscope}%
\begin{pgfscope}%
\pgfsys@transformshift{5.059705in}{1.181455in}%
\pgfsys@useobject{currentmarker}{}%
\end{pgfscope}%
\begin{pgfscope}%
\pgfsys@transformshift{5.912776in}{0.588304in}%
\pgfsys@useobject{currentmarker}{}%
\end{pgfscope}%
\begin{pgfscope}%
\pgfsys@transformshift{6.765848in}{0.789968in}%
\pgfsys@useobject{currentmarker}{}%
\end{pgfscope}%
\end{pgfscope}%
\begin{pgfscope}%
\pgfsetrectcap%
\pgfsetmiterjoin%
\pgfsetlinewidth{0.803000pt}%
\definecolor{currentstroke}{rgb}{0.000000,0.000000,0.000000}%
\pgfsetstrokecolor{currentstroke}%
\pgfsetdash{}{0pt}%
\pgfpathmoveto{\pgfqpoint{1.000000in}{0.440000in}}%
\pgfpathlineto{\pgfqpoint{1.000000in}{3.520000in}}%
\pgfusepath{stroke}%
\end{pgfscope}%
\begin{pgfscope}%
\pgfsetrectcap%
\pgfsetmiterjoin%
\pgfsetlinewidth{0.803000pt}%
\definecolor{currentstroke}{rgb}{0.000000,0.000000,0.000000}%
\pgfsetstrokecolor{currentstroke}%
\pgfsetdash{}{0pt}%
\pgfpathmoveto{\pgfqpoint{7.200000in}{0.440000in}}%
\pgfpathlineto{\pgfqpoint{7.200000in}{3.520000in}}%
\pgfusepath{stroke}%
\end{pgfscope}%
\begin{pgfscope}%
\pgfsetrectcap%
\pgfsetmiterjoin%
\pgfsetlinewidth{0.803000pt}%
\definecolor{currentstroke}{rgb}{0.000000,0.000000,0.000000}%
\pgfsetstrokecolor{currentstroke}%
\pgfsetdash{}{0pt}%
\pgfpathmoveto{\pgfqpoint{1.000000in}{0.440000in}}%
\pgfpathlineto{\pgfqpoint{7.200000in}{0.440000in}}%
\pgfusepath{stroke}%
\end{pgfscope}%
\begin{pgfscope}%
\pgfsetrectcap%
\pgfsetmiterjoin%
\pgfsetlinewidth{0.803000pt}%
\definecolor{currentstroke}{rgb}{0.000000,0.000000,0.000000}%
\pgfsetstrokecolor{currentstroke}%
\pgfsetdash{}{0pt}%
\pgfpathmoveto{\pgfqpoint{1.000000in}{3.520000in}}%
\pgfpathlineto{\pgfqpoint{7.200000in}{3.520000in}}%
\pgfusepath{stroke}%
\end{pgfscope}%
\begin{pgfscope}%
\pgftext[x=1.835299in,y=0.580000in,left,base]{\rmfamily\fontsize{12.000000}{14.400000}\selectfont @}%
\end{pgfscope}%
\begin{pgfscope}%
\pgftext[x=2.241523in,y=0.580000in,left,base]{\rmfamily\fontsize{12.000000}{14.400000}\selectfont mid}%
\end{pgfscope}%
\begin{pgfscope}%
\pgftext[x=2.759459in,y=0.580000in,left,base]{\rmfamily\fontsize{12.000000}{14.400000}\selectfont suh}%
\end{pgfscope}%
\begin{pgfscope}%
\pgftext[x=3.521130in,y=0.580000in,left,base]{\rmfamily\fontsize{12.000000}{14.400000}\selectfont m@}%
\end{pgfscope}%
\begin{pgfscope}%
\pgftext[x=4.094922in,y=0.580000in,left,base]{\rmfamily\fontsize{12.000000}{14.400000}\selectfont naits}%
\end{pgfscope}%
\begin{pgfscope}%
\pgftext[x=5.344062in,y=0.580000in,left,base]{\rmfamily\fontsize{12.000000}{14.400000}\selectfont driim}%
\end{pgfscope}%
\begin{pgfscope}%
\pgfsetbuttcap%
\pgfsetmiterjoin%
\definecolor{currentfill}{rgb}{1.000000,1.000000,1.000000}%
\pgfsetfillcolor{currentfill}%
\pgfsetfillopacity{0.800000}%
\pgfsetlinewidth{1.003750pt}%
\definecolor{currentstroke}{rgb}{0.800000,0.800000,0.800000}%
\pgfsetstrokecolor{currentstroke}%
\pgfsetstrokeopacity{0.800000}%
\pgfsetdash{}{0pt}%
\pgfpathmoveto{\pgfqpoint{5.541322in}{3.021556in}}%
\pgfpathlineto{\pgfqpoint{7.102778in}{3.021556in}}%
\pgfpathquadraticcurveto{\pgfqpoint{7.130556in}{3.021556in}}{\pgfqpoint{7.130556in}{3.049334in}}%
\pgfpathlineto{\pgfqpoint{7.130556in}{3.422778in}}%
\pgfpathquadraticcurveto{\pgfqpoint{7.130556in}{3.450556in}}{\pgfqpoint{7.102778in}{3.450556in}}%
\pgfpathlineto{\pgfqpoint{5.541322in}{3.450556in}}%
\pgfpathquadraticcurveto{\pgfqpoint{5.513544in}{3.450556in}}{\pgfqpoint{5.513544in}{3.422778in}}%
\pgfpathlineto{\pgfqpoint{5.513544in}{3.049334in}}%
\pgfpathquadraticcurveto{\pgfqpoint{5.513544in}{3.021556in}}{\pgfqpoint{5.541322in}{3.021556in}}%
\pgfpathclose%
\pgfusepath{stroke,fill}%
\end{pgfscope}%
\begin{pgfscope}%
\pgfsetbuttcap%
\pgfsetmiterjoin%
\definecolor{currentfill}{rgb}{0.364706,0.517647,0.572549}%
\pgfsetfillcolor{currentfill}%
\pgfsetlinewidth{1.003750pt}%
\definecolor{currentstroke}{rgb}{0.364706,0.517647,0.572549}%
\pgfsetstrokecolor{currentstroke}%
\pgfsetdash{}{0pt}%
\pgfpathmoveto{\pgfqpoint{5.569100in}{3.297778in}}%
\pgfpathlineto{\pgfqpoint{5.846878in}{3.297778in}}%
\pgfpathlineto{\pgfqpoint{5.846878in}{3.395000in}}%
\pgfpathlineto{\pgfqpoint{5.569100in}{3.395000in}}%
\pgfpathclose%
\pgfusepath{stroke,fill}%
\end{pgfscope}%
\begin{pgfscope}%
\pgftext[x=5.957989in,y=3.297778in,left,base]{\rmfamily\fontsize{10.000000}{12.000000}\selectfont Interpolated \(\displaystyle F_0\)}%
\end{pgfscope}%
\begin{pgfscope}%
\pgfsetbuttcap%
\pgfsetmiterjoin%
\definecolor{currentfill}{rgb}{0.800000,0.000000,0.266667}%
\pgfsetfillcolor{currentfill}%
\pgfsetlinewidth{1.003750pt}%
\definecolor{currentstroke}{rgb}{0.800000,0.000000,0.266667}%
\pgfsetstrokecolor{currentstroke}%
\pgfsetdash{}{0pt}%
\pgfpathmoveto{\pgfqpoint{5.569100in}{3.104112in}}%
\pgfpathlineto{\pgfqpoint{5.846878in}{3.104112in}}%
\pgfpathlineto{\pgfqpoint{5.846878in}{3.201334in}}%
\pgfpathlineto{\pgfqpoint{5.569100in}{3.201334in}}%
\pgfpathclose%
\pgfusepath{stroke,fill}%
\end{pgfscope}%
\begin{pgfscope}%
\pgftext[x=5.957989in,y=3.104112in,left,base]{\rmfamily\fontsize{10.000000}{12.000000}\selectfont Syllable Boundary}%
\end{pgfscope}%
\end{pgfpicture}%
\makeatother%
\endgroup%
}
\caption[3 anchor-point sampling]{Location of extraction points in an \ac{F0} contour with 3 anchor points.}
\label{fig:low-anchor}
\end{figure}


By the same token, if we want to increase the number of anchor points to capture more complex behaviors in longer syllables, as it was done in \autoref{fig:high-anchor}, we end up with too many points in shorter syllables, where contours are less complex.


\begin{figure}[!h]
\centering
\resizebox{\textwidth}{!}{%% Creator: Matplotlib, PGF backend
%%
%% To include the figure in your LaTeX document, write
%%   \input{<filename>.pgf}
%%
%% Make sure the required packages are loaded in your preamble
%%   \usepackage{pgf}
%%
%% Figures using additional raster images can only be included by \input if
%% they are in the same directory as the main LaTeX file. For loading figures
%% from other directories you can use the `import` package
%%   \usepackage{import}
%% and then include the figures with
%%   \import{<path to file>}{<filename>.pgf}
%%
%% Matplotlib used the following preamble
%%   \usepackage[utf8x]{inputenc}
%%   \usepackage[T1]{fontenc}
%%   \usepackage{cmbright}
%%
\begingroup%
\makeatletter%
\begin{pgfpicture}%
\pgfpathrectangle{\pgfpointorigin}{\pgfqpoint{8.000000in}{4.000000in}}%
\pgfusepath{use as bounding box, clip}%
\begin{pgfscope}%
\pgfsetbuttcap%
\pgfsetmiterjoin%
\definecolor{currentfill}{rgb}{1.000000,1.000000,1.000000}%
\pgfsetfillcolor{currentfill}%
\pgfsetlinewidth{0.000000pt}%
\definecolor{currentstroke}{rgb}{1.000000,1.000000,1.000000}%
\pgfsetstrokecolor{currentstroke}%
\pgfsetdash{}{0pt}%
\pgfpathmoveto{\pgfqpoint{0.000000in}{0.000000in}}%
\pgfpathlineto{\pgfqpoint{8.000000in}{0.000000in}}%
\pgfpathlineto{\pgfqpoint{8.000000in}{4.000000in}}%
\pgfpathlineto{\pgfqpoint{0.000000in}{4.000000in}}%
\pgfpathclose%
\pgfusepath{fill}%
\end{pgfscope}%
\begin{pgfscope}%
\pgfsetbuttcap%
\pgfsetmiterjoin%
\definecolor{currentfill}{rgb}{1.000000,1.000000,1.000000}%
\pgfsetfillcolor{currentfill}%
\pgfsetlinewidth{0.000000pt}%
\definecolor{currentstroke}{rgb}{0.000000,0.000000,0.000000}%
\pgfsetstrokecolor{currentstroke}%
\pgfsetstrokeopacity{0.000000}%
\pgfsetdash{}{0pt}%
\pgfpathmoveto{\pgfqpoint{1.000000in}{0.440000in}}%
\pgfpathlineto{\pgfqpoint{7.200000in}{0.440000in}}%
\pgfpathlineto{\pgfqpoint{7.200000in}{3.520000in}}%
\pgfpathlineto{\pgfqpoint{1.000000in}{3.520000in}}%
\pgfpathclose%
\pgfusepath{fill}%
\end{pgfscope}%
\begin{pgfscope}%
\pgfsetbuttcap%
\pgfsetroundjoin%
\definecolor{currentfill}{rgb}{0.000000,0.000000,0.000000}%
\pgfsetfillcolor{currentfill}%
\pgfsetlinewidth{0.803000pt}%
\definecolor{currentstroke}{rgb}{0.000000,0.000000,0.000000}%
\pgfsetstrokecolor{currentstroke}%
\pgfsetdash{}{0pt}%
\pgfsys@defobject{currentmarker}{\pgfqpoint{0.000000in}{-0.048611in}}{\pgfqpoint{0.000000in}{0.000000in}}{%
\pgfpathmoveto{\pgfqpoint{0.000000in}{0.000000in}}%
\pgfpathlineto{\pgfqpoint{0.000000in}{-0.048611in}}%
\pgfusepath{stroke,fill}%
}%
\begin{pgfscope}%
\pgfsys@transformshift{1.281818in}{0.440000in}%
\pgfsys@useobject{currentmarker}{}%
\end{pgfscope}%
\end{pgfscope}%
\begin{pgfscope}%
\pgftext[x=1.281818in,y=0.342778in,,top]{\rmfamily\fontsize{10.000000}{12.000000}\selectfont \(\displaystyle 0.00\)}%
\end{pgfscope}%
\begin{pgfscope}%
\pgfsetbuttcap%
\pgfsetroundjoin%
\definecolor{currentfill}{rgb}{0.000000,0.000000,0.000000}%
\pgfsetfillcolor{currentfill}%
\pgfsetlinewidth{0.803000pt}%
\definecolor{currentstroke}{rgb}{0.000000,0.000000,0.000000}%
\pgfsetstrokecolor{currentstroke}%
\pgfsetdash{}{0pt}%
\pgfsys@defobject{currentmarker}{\pgfqpoint{0.000000in}{-0.048611in}}{\pgfqpoint{0.000000in}{0.000000in}}{%
\pgfpathmoveto{\pgfqpoint{0.000000in}{0.000000in}}%
\pgfpathlineto{\pgfqpoint{0.000000in}{-0.048611in}}%
\pgfusepath{stroke,fill}%
}%
\begin{pgfscope}%
\pgfsys@transformshift{2.043489in}{0.440000in}%
\pgfsys@useobject{currentmarker}{}%
\end{pgfscope}%
\end{pgfscope}%
\begin{pgfscope}%
\pgftext[x=2.043489in,y=0.342778in,,top]{\rmfamily\fontsize{10.000000}{12.000000}\selectfont \(\displaystyle 0.25\)}%
\end{pgfscope}%
\begin{pgfscope}%
\pgfsetbuttcap%
\pgfsetroundjoin%
\definecolor{currentfill}{rgb}{0.000000,0.000000,0.000000}%
\pgfsetfillcolor{currentfill}%
\pgfsetlinewidth{0.803000pt}%
\definecolor{currentstroke}{rgb}{0.000000,0.000000,0.000000}%
\pgfsetstrokecolor{currentstroke}%
\pgfsetdash{}{0pt}%
\pgfsys@defobject{currentmarker}{\pgfqpoint{0.000000in}{-0.048611in}}{\pgfqpoint{0.000000in}{0.000000in}}{%
\pgfpathmoveto{\pgfqpoint{0.000000in}{0.000000in}}%
\pgfpathlineto{\pgfqpoint{0.000000in}{-0.048611in}}%
\pgfusepath{stroke,fill}%
}%
\begin{pgfscope}%
\pgfsys@transformshift{2.805160in}{0.440000in}%
\pgfsys@useobject{currentmarker}{}%
\end{pgfscope}%
\end{pgfscope}%
\begin{pgfscope}%
\pgftext[x=2.805160in,y=0.342778in,,top]{\rmfamily\fontsize{10.000000}{12.000000}\selectfont \(\displaystyle 0.50\)}%
\end{pgfscope}%
\begin{pgfscope}%
\pgfsetbuttcap%
\pgfsetroundjoin%
\definecolor{currentfill}{rgb}{0.000000,0.000000,0.000000}%
\pgfsetfillcolor{currentfill}%
\pgfsetlinewidth{0.803000pt}%
\definecolor{currentstroke}{rgb}{0.000000,0.000000,0.000000}%
\pgfsetstrokecolor{currentstroke}%
\pgfsetdash{}{0pt}%
\pgfsys@defobject{currentmarker}{\pgfqpoint{0.000000in}{-0.048611in}}{\pgfqpoint{0.000000in}{0.000000in}}{%
\pgfpathmoveto{\pgfqpoint{0.000000in}{0.000000in}}%
\pgfpathlineto{\pgfqpoint{0.000000in}{-0.048611in}}%
\pgfusepath{stroke,fill}%
}%
\begin{pgfscope}%
\pgfsys@transformshift{3.566830in}{0.440000in}%
\pgfsys@useobject{currentmarker}{}%
\end{pgfscope}%
\end{pgfscope}%
\begin{pgfscope}%
\pgftext[x=3.566830in,y=0.342778in,,top]{\rmfamily\fontsize{10.000000}{12.000000}\selectfont \(\displaystyle 0.75\)}%
\end{pgfscope}%
\begin{pgfscope}%
\pgfsetbuttcap%
\pgfsetroundjoin%
\definecolor{currentfill}{rgb}{0.000000,0.000000,0.000000}%
\pgfsetfillcolor{currentfill}%
\pgfsetlinewidth{0.803000pt}%
\definecolor{currentstroke}{rgb}{0.000000,0.000000,0.000000}%
\pgfsetstrokecolor{currentstroke}%
\pgfsetdash{}{0pt}%
\pgfsys@defobject{currentmarker}{\pgfqpoint{0.000000in}{-0.048611in}}{\pgfqpoint{0.000000in}{0.000000in}}{%
\pgfpathmoveto{\pgfqpoint{0.000000in}{0.000000in}}%
\pgfpathlineto{\pgfqpoint{0.000000in}{-0.048611in}}%
\pgfusepath{stroke,fill}%
}%
\begin{pgfscope}%
\pgfsys@transformshift{4.328501in}{0.440000in}%
\pgfsys@useobject{currentmarker}{}%
\end{pgfscope}%
\end{pgfscope}%
\begin{pgfscope}%
\pgftext[x=4.328501in,y=0.342778in,,top]{\rmfamily\fontsize{10.000000}{12.000000}\selectfont \(\displaystyle 1.00\)}%
\end{pgfscope}%
\begin{pgfscope}%
\pgfsetbuttcap%
\pgfsetroundjoin%
\definecolor{currentfill}{rgb}{0.000000,0.000000,0.000000}%
\pgfsetfillcolor{currentfill}%
\pgfsetlinewidth{0.803000pt}%
\definecolor{currentstroke}{rgb}{0.000000,0.000000,0.000000}%
\pgfsetstrokecolor{currentstroke}%
\pgfsetdash{}{0pt}%
\pgfsys@defobject{currentmarker}{\pgfqpoint{0.000000in}{-0.048611in}}{\pgfqpoint{0.000000in}{0.000000in}}{%
\pgfpathmoveto{\pgfqpoint{0.000000in}{0.000000in}}%
\pgfpathlineto{\pgfqpoint{0.000000in}{-0.048611in}}%
\pgfusepath{stroke,fill}%
}%
\begin{pgfscope}%
\pgfsys@transformshift{5.090172in}{0.440000in}%
\pgfsys@useobject{currentmarker}{}%
\end{pgfscope}%
\end{pgfscope}%
\begin{pgfscope}%
\pgftext[x=5.090172in,y=0.342778in,,top]{\rmfamily\fontsize{10.000000}{12.000000}\selectfont \(\displaystyle 1.25\)}%
\end{pgfscope}%
\begin{pgfscope}%
\pgfsetbuttcap%
\pgfsetroundjoin%
\definecolor{currentfill}{rgb}{0.000000,0.000000,0.000000}%
\pgfsetfillcolor{currentfill}%
\pgfsetlinewidth{0.803000pt}%
\definecolor{currentstroke}{rgb}{0.000000,0.000000,0.000000}%
\pgfsetstrokecolor{currentstroke}%
\pgfsetdash{}{0pt}%
\pgfsys@defobject{currentmarker}{\pgfqpoint{0.000000in}{-0.048611in}}{\pgfqpoint{0.000000in}{0.000000in}}{%
\pgfpathmoveto{\pgfqpoint{0.000000in}{0.000000in}}%
\pgfpathlineto{\pgfqpoint{0.000000in}{-0.048611in}}%
\pgfusepath{stroke,fill}%
}%
\begin{pgfscope}%
\pgfsys@transformshift{5.851843in}{0.440000in}%
\pgfsys@useobject{currentmarker}{}%
\end{pgfscope}%
\end{pgfscope}%
\begin{pgfscope}%
\pgftext[x=5.851843in,y=0.342778in,,top]{\rmfamily\fontsize{10.000000}{12.000000}\selectfont \(\displaystyle 1.50\)}%
\end{pgfscope}%
\begin{pgfscope}%
\pgfsetbuttcap%
\pgfsetroundjoin%
\definecolor{currentfill}{rgb}{0.000000,0.000000,0.000000}%
\pgfsetfillcolor{currentfill}%
\pgfsetlinewidth{0.803000pt}%
\definecolor{currentstroke}{rgb}{0.000000,0.000000,0.000000}%
\pgfsetstrokecolor{currentstroke}%
\pgfsetdash{}{0pt}%
\pgfsys@defobject{currentmarker}{\pgfqpoint{0.000000in}{-0.048611in}}{\pgfqpoint{0.000000in}{0.000000in}}{%
\pgfpathmoveto{\pgfqpoint{0.000000in}{0.000000in}}%
\pgfpathlineto{\pgfqpoint{0.000000in}{-0.048611in}}%
\pgfusepath{stroke,fill}%
}%
\begin{pgfscope}%
\pgfsys@transformshift{6.613514in}{0.440000in}%
\pgfsys@useobject{currentmarker}{}%
\end{pgfscope}%
\end{pgfscope}%
\begin{pgfscope}%
\pgftext[x=6.613514in,y=0.342778in,,top]{\rmfamily\fontsize{10.000000}{12.000000}\selectfont \(\displaystyle 1.75\)}%
\end{pgfscope}%
\begin{pgfscope}%
\pgftext[x=4.100000in,y=0.164567in,,top]{\rmfamily\fontsize{10.000000}{12.000000}\selectfont Time (s)}%
\end{pgfscope}%
\begin{pgfscope}%
\pgfsetbuttcap%
\pgfsetroundjoin%
\definecolor{currentfill}{rgb}{0.000000,0.000000,0.000000}%
\pgfsetfillcolor{currentfill}%
\pgfsetlinewidth{0.803000pt}%
\definecolor{currentstroke}{rgb}{0.000000,0.000000,0.000000}%
\pgfsetstrokecolor{currentstroke}%
\pgfsetdash{}{0pt}%
\pgfsys@defobject{currentmarker}{\pgfqpoint{-0.048611in}{0.000000in}}{\pgfqpoint{0.000000in}{0.000000in}}{%
\pgfpathmoveto{\pgfqpoint{0.000000in}{0.000000in}}%
\pgfpathlineto{\pgfqpoint{-0.048611in}{0.000000in}}%
\pgfusepath{stroke,fill}%
}%
\begin{pgfscope}%
\pgfsys@transformshift{1.000000in}{0.747896in}%
\pgfsys@useobject{currentmarker}{}%
\end{pgfscope}%
\end{pgfscope}%
\begin{pgfscope}%
\pgftext[x=0.684030in,y=0.700068in,left,base]{\rmfamily\fontsize{10.000000}{12.000000}\selectfont \(\displaystyle 140\)}%
\end{pgfscope}%
\begin{pgfscope}%
\pgfsetbuttcap%
\pgfsetroundjoin%
\definecolor{currentfill}{rgb}{0.000000,0.000000,0.000000}%
\pgfsetfillcolor{currentfill}%
\pgfsetlinewidth{0.803000pt}%
\definecolor{currentstroke}{rgb}{0.000000,0.000000,0.000000}%
\pgfsetstrokecolor{currentstroke}%
\pgfsetdash{}{0pt}%
\pgfsys@defobject{currentmarker}{\pgfqpoint{-0.048611in}{0.000000in}}{\pgfqpoint{0.000000in}{0.000000in}}{%
\pgfpathmoveto{\pgfqpoint{0.000000in}{0.000000in}}%
\pgfpathlineto{\pgfqpoint{-0.048611in}{0.000000in}}%
\pgfusepath{stroke,fill}%
}%
\begin{pgfscope}%
\pgfsys@transformshift{1.000000in}{1.175589in}%
\pgfsys@useobject{currentmarker}{}%
\end{pgfscope}%
\end{pgfscope}%
\begin{pgfscope}%
\pgftext[x=0.684030in,y=1.127761in,left,base]{\rmfamily\fontsize{10.000000}{12.000000}\selectfont \(\displaystyle 160\)}%
\end{pgfscope}%
\begin{pgfscope}%
\pgfsetbuttcap%
\pgfsetroundjoin%
\definecolor{currentfill}{rgb}{0.000000,0.000000,0.000000}%
\pgfsetfillcolor{currentfill}%
\pgfsetlinewidth{0.803000pt}%
\definecolor{currentstroke}{rgb}{0.000000,0.000000,0.000000}%
\pgfsetstrokecolor{currentstroke}%
\pgfsetdash{}{0pt}%
\pgfsys@defobject{currentmarker}{\pgfqpoint{-0.048611in}{0.000000in}}{\pgfqpoint{0.000000in}{0.000000in}}{%
\pgfpathmoveto{\pgfqpoint{0.000000in}{0.000000in}}%
\pgfpathlineto{\pgfqpoint{-0.048611in}{0.000000in}}%
\pgfusepath{stroke,fill}%
}%
\begin{pgfscope}%
\pgfsys@transformshift{1.000000in}{1.603282in}%
\pgfsys@useobject{currentmarker}{}%
\end{pgfscope}%
\end{pgfscope}%
\begin{pgfscope}%
\pgftext[x=0.684030in,y=1.555454in,left,base]{\rmfamily\fontsize{10.000000}{12.000000}\selectfont \(\displaystyle 180\)}%
\end{pgfscope}%
\begin{pgfscope}%
\pgfsetbuttcap%
\pgfsetroundjoin%
\definecolor{currentfill}{rgb}{0.000000,0.000000,0.000000}%
\pgfsetfillcolor{currentfill}%
\pgfsetlinewidth{0.803000pt}%
\definecolor{currentstroke}{rgb}{0.000000,0.000000,0.000000}%
\pgfsetstrokecolor{currentstroke}%
\pgfsetdash{}{0pt}%
\pgfsys@defobject{currentmarker}{\pgfqpoint{-0.048611in}{0.000000in}}{\pgfqpoint{0.000000in}{0.000000in}}{%
\pgfpathmoveto{\pgfqpoint{0.000000in}{0.000000in}}%
\pgfpathlineto{\pgfqpoint{-0.048611in}{0.000000in}}%
\pgfusepath{stroke,fill}%
}%
\begin{pgfscope}%
\pgfsys@transformshift{1.000000in}{2.030975in}%
\pgfsys@useobject{currentmarker}{}%
\end{pgfscope}%
\end{pgfscope}%
\begin{pgfscope}%
\pgftext[x=0.684030in,y=1.983147in,left,base]{\rmfamily\fontsize{10.000000}{12.000000}\selectfont \(\displaystyle 200\)}%
\end{pgfscope}%
\begin{pgfscope}%
\pgfsetbuttcap%
\pgfsetroundjoin%
\definecolor{currentfill}{rgb}{0.000000,0.000000,0.000000}%
\pgfsetfillcolor{currentfill}%
\pgfsetlinewidth{0.803000pt}%
\definecolor{currentstroke}{rgb}{0.000000,0.000000,0.000000}%
\pgfsetstrokecolor{currentstroke}%
\pgfsetdash{}{0pt}%
\pgfsys@defobject{currentmarker}{\pgfqpoint{-0.048611in}{0.000000in}}{\pgfqpoint{0.000000in}{0.000000in}}{%
\pgfpathmoveto{\pgfqpoint{0.000000in}{0.000000in}}%
\pgfpathlineto{\pgfqpoint{-0.048611in}{0.000000in}}%
\pgfusepath{stroke,fill}%
}%
\begin{pgfscope}%
\pgfsys@transformshift{1.000000in}{2.458668in}%
\pgfsys@useobject{currentmarker}{}%
\end{pgfscope}%
\end{pgfscope}%
\begin{pgfscope}%
\pgftext[x=0.684030in,y=2.410840in,left,base]{\rmfamily\fontsize{10.000000}{12.000000}\selectfont \(\displaystyle 220\)}%
\end{pgfscope}%
\begin{pgfscope}%
\pgfsetbuttcap%
\pgfsetroundjoin%
\definecolor{currentfill}{rgb}{0.000000,0.000000,0.000000}%
\pgfsetfillcolor{currentfill}%
\pgfsetlinewidth{0.803000pt}%
\definecolor{currentstroke}{rgb}{0.000000,0.000000,0.000000}%
\pgfsetstrokecolor{currentstroke}%
\pgfsetdash{}{0pt}%
\pgfsys@defobject{currentmarker}{\pgfqpoint{-0.048611in}{0.000000in}}{\pgfqpoint{0.000000in}{0.000000in}}{%
\pgfpathmoveto{\pgfqpoint{0.000000in}{0.000000in}}%
\pgfpathlineto{\pgfqpoint{-0.048611in}{0.000000in}}%
\pgfusepath{stroke,fill}%
}%
\begin{pgfscope}%
\pgfsys@transformshift{1.000000in}{2.886361in}%
\pgfsys@useobject{currentmarker}{}%
\end{pgfscope}%
\end{pgfscope}%
\begin{pgfscope}%
\pgftext[x=0.684030in,y=2.838533in,left,base]{\rmfamily\fontsize{10.000000}{12.000000}\selectfont \(\displaystyle 240\)}%
\end{pgfscope}%
\begin{pgfscope}%
\pgfsetbuttcap%
\pgfsetroundjoin%
\definecolor{currentfill}{rgb}{0.000000,0.000000,0.000000}%
\pgfsetfillcolor{currentfill}%
\pgfsetlinewidth{0.803000pt}%
\definecolor{currentstroke}{rgb}{0.000000,0.000000,0.000000}%
\pgfsetstrokecolor{currentstroke}%
\pgfsetdash{}{0pt}%
\pgfsys@defobject{currentmarker}{\pgfqpoint{-0.048611in}{0.000000in}}{\pgfqpoint{0.000000in}{0.000000in}}{%
\pgfpathmoveto{\pgfqpoint{0.000000in}{0.000000in}}%
\pgfpathlineto{\pgfqpoint{-0.048611in}{0.000000in}}%
\pgfusepath{stroke,fill}%
}%
\begin{pgfscope}%
\pgfsys@transformshift{1.000000in}{3.314054in}%
\pgfsys@useobject{currentmarker}{}%
\end{pgfscope}%
\end{pgfscope}%
\begin{pgfscope}%
\pgftext[x=0.684030in,y=3.266226in,left,base]{\rmfamily\fontsize{10.000000}{12.000000}\selectfont \(\displaystyle 260\)}%
\end{pgfscope}%
\begin{pgfscope}%
\pgftext[x=0.628474in,y=1.980000in,,bottom,rotate=90.000000]{\rmfamily\fontsize{10.000000}{12.000000}\selectfont Frequency (Hz)}%
\end{pgfscope}%
\begin{pgfscope}%
\pgfpathrectangle{\pgfqpoint{1.000000in}{0.440000in}}{\pgfqpoint{6.200000in}{3.080000in}} %
\pgfusepath{clip}%
\pgfsetrectcap%
\pgfsetroundjoin%
\pgfsetlinewidth{0.501875pt}%
\definecolor{currentstroke}{rgb}{0.800000,0.000000,0.266667}%
\pgfsetstrokecolor{currentstroke}%
\pgfsetdash{}{0pt}%
\pgfpathmoveto{\pgfqpoint{1.769287in}{0.440000in}}%
\pgfpathlineto{\pgfqpoint{1.769287in}{3.520000in}}%
\pgfusepath{stroke}%
\end{pgfscope}%
\begin{pgfscope}%
\pgfpathrectangle{\pgfqpoint{1.000000in}{0.440000in}}{\pgfqpoint{6.200000in}{3.080000in}} %
\pgfusepath{clip}%
\pgfsetrectcap%
\pgfsetroundjoin%
\pgfsetlinewidth{0.501875pt}%
\definecolor{currentstroke}{rgb}{0.800000,0.000000,0.266667}%
\pgfsetstrokecolor{currentstroke}%
\pgfsetdash{}{0pt}%
\pgfpathmoveto{\pgfqpoint{2.165356in}{0.440000in}}%
\pgfpathlineto{\pgfqpoint{2.165356in}{3.520000in}}%
\pgfusepath{stroke}%
\end{pgfscope}%
\begin{pgfscope}%
\pgfpathrectangle{\pgfqpoint{1.000000in}{0.440000in}}{\pgfqpoint{6.200000in}{3.080000in}} %
\pgfusepath{clip}%
\pgfsetrectcap%
\pgfsetroundjoin%
\pgfsetlinewidth{0.501875pt}%
\definecolor{currentstroke}{rgb}{0.501961,0.501961,0.501961}%
\pgfsetstrokecolor{currentstroke}%
\pgfsetdash{}{0pt}%
\pgfpathmoveto{\pgfqpoint{1.835299in}{0.440000in}}%
\pgfpathlineto{\pgfqpoint{1.835299in}{3.520000in}}%
\pgfusepath{stroke}%
\end{pgfscope}%
\begin{pgfscope}%
\pgfpathrectangle{\pgfqpoint{1.000000in}{0.440000in}}{\pgfqpoint{6.200000in}{3.080000in}} %
\pgfusepath{clip}%
\pgfsetrectcap%
\pgfsetroundjoin%
\pgfsetlinewidth{0.501875pt}%
\definecolor{currentstroke}{rgb}{0.501961,0.501961,0.501961}%
\pgfsetstrokecolor{currentstroke}%
\pgfsetdash{}{0pt}%
\pgfpathmoveto{\pgfqpoint{1.901310in}{0.440000in}}%
\pgfpathlineto{\pgfqpoint{1.901310in}{3.520000in}}%
\pgfusepath{stroke}%
\end{pgfscope}%
\begin{pgfscope}%
\pgfpathrectangle{\pgfqpoint{1.000000in}{0.440000in}}{\pgfqpoint{6.200000in}{3.080000in}} %
\pgfusepath{clip}%
\pgfsetrectcap%
\pgfsetroundjoin%
\pgfsetlinewidth{0.501875pt}%
\definecolor{currentstroke}{rgb}{0.501961,0.501961,0.501961}%
\pgfsetstrokecolor{currentstroke}%
\pgfsetdash{}{0pt}%
\pgfpathmoveto{\pgfqpoint{1.967322in}{0.440000in}}%
\pgfpathlineto{\pgfqpoint{1.967322in}{3.520000in}}%
\pgfusepath{stroke}%
\end{pgfscope}%
\begin{pgfscope}%
\pgfpathrectangle{\pgfqpoint{1.000000in}{0.440000in}}{\pgfqpoint{6.200000in}{3.080000in}} %
\pgfusepath{clip}%
\pgfsetrectcap%
\pgfsetroundjoin%
\pgfsetlinewidth{0.501875pt}%
\definecolor{currentstroke}{rgb}{0.501961,0.501961,0.501961}%
\pgfsetstrokecolor{currentstroke}%
\pgfsetdash{}{0pt}%
\pgfpathmoveto{\pgfqpoint{2.033333in}{0.440000in}}%
\pgfpathlineto{\pgfqpoint{2.033333in}{3.520000in}}%
\pgfusepath{stroke}%
\end{pgfscope}%
\begin{pgfscope}%
\pgfpathrectangle{\pgfqpoint{1.000000in}{0.440000in}}{\pgfqpoint{6.200000in}{3.080000in}} %
\pgfusepath{clip}%
\pgfsetrectcap%
\pgfsetroundjoin%
\pgfsetlinewidth{0.501875pt}%
\definecolor{currentstroke}{rgb}{0.501961,0.501961,0.501961}%
\pgfsetstrokecolor{currentstroke}%
\pgfsetdash{}{0pt}%
\pgfpathmoveto{\pgfqpoint{2.099345in}{0.440000in}}%
\pgfpathlineto{\pgfqpoint{2.099345in}{3.520000in}}%
\pgfusepath{stroke}%
\end{pgfscope}%
\begin{pgfscope}%
\pgfpathrectangle{\pgfqpoint{1.000000in}{0.440000in}}{\pgfqpoint{6.200000in}{3.080000in}} %
\pgfusepath{clip}%
\pgfsetrectcap%
\pgfsetroundjoin%
\pgfsetlinewidth{0.501875pt}%
\definecolor{currentstroke}{rgb}{0.800000,0.000000,0.266667}%
\pgfsetstrokecolor{currentstroke}%
\pgfsetdash{}{0pt}%
\pgfpathmoveto{\pgfqpoint{2.165356in}{0.440000in}}%
\pgfpathlineto{\pgfqpoint{2.165356in}{3.520000in}}%
\pgfusepath{stroke}%
\end{pgfscope}%
\begin{pgfscope}%
\pgfpathrectangle{\pgfqpoint{1.000000in}{0.440000in}}{\pgfqpoint{6.200000in}{3.080000in}} %
\pgfusepath{clip}%
\pgfsetrectcap%
\pgfsetroundjoin%
\pgfsetlinewidth{0.501875pt}%
\definecolor{currentstroke}{rgb}{0.800000,0.000000,0.266667}%
\pgfsetstrokecolor{currentstroke}%
\pgfsetdash{}{0pt}%
\pgfpathmoveto{\pgfqpoint{2.622359in}{0.440000in}}%
\pgfpathlineto{\pgfqpoint{2.622359in}{3.520000in}}%
\pgfusepath{stroke}%
\end{pgfscope}%
\begin{pgfscope}%
\pgfpathrectangle{\pgfqpoint{1.000000in}{0.440000in}}{\pgfqpoint{6.200000in}{3.080000in}} %
\pgfusepath{clip}%
\pgfsetrectcap%
\pgfsetroundjoin%
\pgfsetlinewidth{0.501875pt}%
\definecolor{currentstroke}{rgb}{0.501961,0.501961,0.501961}%
\pgfsetstrokecolor{currentstroke}%
\pgfsetdash{}{0pt}%
\pgfpathmoveto{\pgfqpoint{2.241523in}{0.440000in}}%
\pgfpathlineto{\pgfqpoint{2.241523in}{3.520000in}}%
\pgfusepath{stroke}%
\end{pgfscope}%
\begin{pgfscope}%
\pgfpathrectangle{\pgfqpoint{1.000000in}{0.440000in}}{\pgfqpoint{6.200000in}{3.080000in}} %
\pgfusepath{clip}%
\pgfsetrectcap%
\pgfsetroundjoin%
\pgfsetlinewidth{0.501875pt}%
\definecolor{currentstroke}{rgb}{0.501961,0.501961,0.501961}%
\pgfsetstrokecolor{currentstroke}%
\pgfsetdash{}{0pt}%
\pgfpathmoveto{\pgfqpoint{2.317690in}{0.440000in}}%
\pgfpathlineto{\pgfqpoint{2.317690in}{3.520000in}}%
\pgfusepath{stroke}%
\end{pgfscope}%
\begin{pgfscope}%
\pgfpathrectangle{\pgfqpoint{1.000000in}{0.440000in}}{\pgfqpoint{6.200000in}{3.080000in}} %
\pgfusepath{clip}%
\pgfsetrectcap%
\pgfsetroundjoin%
\pgfsetlinewidth{0.501875pt}%
\definecolor{currentstroke}{rgb}{0.501961,0.501961,0.501961}%
\pgfsetstrokecolor{currentstroke}%
\pgfsetdash{}{0pt}%
\pgfpathmoveto{\pgfqpoint{2.393857in}{0.440000in}}%
\pgfpathlineto{\pgfqpoint{2.393857in}{3.520000in}}%
\pgfusepath{stroke}%
\end{pgfscope}%
\begin{pgfscope}%
\pgfpathrectangle{\pgfqpoint{1.000000in}{0.440000in}}{\pgfqpoint{6.200000in}{3.080000in}} %
\pgfusepath{clip}%
\pgfsetrectcap%
\pgfsetroundjoin%
\pgfsetlinewidth{0.501875pt}%
\definecolor{currentstroke}{rgb}{0.501961,0.501961,0.501961}%
\pgfsetstrokecolor{currentstroke}%
\pgfsetdash{}{0pt}%
\pgfpathmoveto{\pgfqpoint{2.470025in}{0.440000in}}%
\pgfpathlineto{\pgfqpoint{2.470025in}{3.520000in}}%
\pgfusepath{stroke}%
\end{pgfscope}%
\begin{pgfscope}%
\pgfpathrectangle{\pgfqpoint{1.000000in}{0.440000in}}{\pgfqpoint{6.200000in}{3.080000in}} %
\pgfusepath{clip}%
\pgfsetrectcap%
\pgfsetroundjoin%
\pgfsetlinewidth{0.501875pt}%
\definecolor{currentstroke}{rgb}{0.501961,0.501961,0.501961}%
\pgfsetstrokecolor{currentstroke}%
\pgfsetdash{}{0pt}%
\pgfpathmoveto{\pgfqpoint{2.546192in}{0.440000in}}%
\pgfpathlineto{\pgfqpoint{2.546192in}{3.520000in}}%
\pgfusepath{stroke}%
\end{pgfscope}%
\begin{pgfscope}%
\pgfpathrectangle{\pgfqpoint{1.000000in}{0.440000in}}{\pgfqpoint{6.200000in}{3.080000in}} %
\pgfusepath{clip}%
\pgfsetrectcap%
\pgfsetroundjoin%
\pgfsetlinewidth{0.501875pt}%
\definecolor{currentstroke}{rgb}{0.800000,0.000000,0.266667}%
\pgfsetstrokecolor{currentstroke}%
\pgfsetdash{}{0pt}%
\pgfpathmoveto{\pgfqpoint{2.622359in}{0.440000in}}%
\pgfpathlineto{\pgfqpoint{2.622359in}{3.520000in}}%
\pgfusepath{stroke}%
\end{pgfscope}%
\begin{pgfscope}%
\pgfpathrectangle{\pgfqpoint{1.000000in}{0.440000in}}{\pgfqpoint{6.200000in}{3.080000in}} %
\pgfusepath{clip}%
\pgfsetrectcap%
\pgfsetroundjoin%
\pgfsetlinewidth{0.501875pt}%
\definecolor{currentstroke}{rgb}{0.800000,0.000000,0.266667}%
\pgfsetstrokecolor{currentstroke}%
\pgfsetdash{}{0pt}%
\pgfpathmoveto{\pgfqpoint{3.444963in}{0.440000in}}%
\pgfpathlineto{\pgfqpoint{3.444963in}{3.520000in}}%
\pgfusepath{stroke}%
\end{pgfscope}%
\begin{pgfscope}%
\pgfpathrectangle{\pgfqpoint{1.000000in}{0.440000in}}{\pgfqpoint{6.200000in}{3.080000in}} %
\pgfusepath{clip}%
\pgfsetrectcap%
\pgfsetroundjoin%
\pgfsetlinewidth{0.501875pt}%
\definecolor{currentstroke}{rgb}{0.501961,0.501961,0.501961}%
\pgfsetstrokecolor{currentstroke}%
\pgfsetdash{}{0pt}%
\pgfpathmoveto{\pgfqpoint{2.759459in}{0.440000in}}%
\pgfpathlineto{\pgfqpoint{2.759459in}{3.520000in}}%
\pgfusepath{stroke}%
\end{pgfscope}%
\begin{pgfscope}%
\pgfpathrectangle{\pgfqpoint{1.000000in}{0.440000in}}{\pgfqpoint{6.200000in}{3.080000in}} %
\pgfusepath{clip}%
\pgfsetrectcap%
\pgfsetroundjoin%
\pgfsetlinewidth{0.501875pt}%
\definecolor{currentstroke}{rgb}{0.501961,0.501961,0.501961}%
\pgfsetstrokecolor{currentstroke}%
\pgfsetdash{}{0pt}%
\pgfpathmoveto{\pgfqpoint{2.896560in}{0.440000in}}%
\pgfpathlineto{\pgfqpoint{2.896560in}{3.520000in}}%
\pgfusepath{stroke}%
\end{pgfscope}%
\begin{pgfscope}%
\pgfpathrectangle{\pgfqpoint{1.000000in}{0.440000in}}{\pgfqpoint{6.200000in}{3.080000in}} %
\pgfusepath{clip}%
\pgfsetrectcap%
\pgfsetroundjoin%
\pgfsetlinewidth{0.501875pt}%
\definecolor{currentstroke}{rgb}{0.501961,0.501961,0.501961}%
\pgfsetstrokecolor{currentstroke}%
\pgfsetdash{}{0pt}%
\pgfpathmoveto{\pgfqpoint{3.033661in}{0.440000in}}%
\pgfpathlineto{\pgfqpoint{3.033661in}{3.520000in}}%
\pgfusepath{stroke}%
\end{pgfscope}%
\begin{pgfscope}%
\pgfpathrectangle{\pgfqpoint{1.000000in}{0.440000in}}{\pgfqpoint{6.200000in}{3.080000in}} %
\pgfusepath{clip}%
\pgfsetrectcap%
\pgfsetroundjoin%
\pgfsetlinewidth{0.501875pt}%
\definecolor{currentstroke}{rgb}{0.501961,0.501961,0.501961}%
\pgfsetstrokecolor{currentstroke}%
\pgfsetdash{}{0pt}%
\pgfpathmoveto{\pgfqpoint{3.170762in}{0.440000in}}%
\pgfpathlineto{\pgfqpoint{3.170762in}{3.520000in}}%
\pgfusepath{stroke}%
\end{pgfscope}%
\begin{pgfscope}%
\pgfpathrectangle{\pgfqpoint{1.000000in}{0.440000in}}{\pgfqpoint{6.200000in}{3.080000in}} %
\pgfusepath{clip}%
\pgfsetrectcap%
\pgfsetroundjoin%
\pgfsetlinewidth{0.501875pt}%
\definecolor{currentstroke}{rgb}{0.501961,0.501961,0.501961}%
\pgfsetstrokecolor{currentstroke}%
\pgfsetdash{}{0pt}%
\pgfpathmoveto{\pgfqpoint{3.307862in}{0.440000in}}%
\pgfpathlineto{\pgfqpoint{3.307862in}{3.520000in}}%
\pgfusepath{stroke}%
\end{pgfscope}%
\begin{pgfscope}%
\pgfpathrectangle{\pgfqpoint{1.000000in}{0.440000in}}{\pgfqpoint{6.200000in}{3.080000in}} %
\pgfusepath{clip}%
\pgfsetrectcap%
\pgfsetroundjoin%
\pgfsetlinewidth{0.501875pt}%
\definecolor{currentstroke}{rgb}{0.800000,0.000000,0.266667}%
\pgfsetstrokecolor{currentstroke}%
\pgfsetdash{}{0pt}%
\pgfpathmoveto{\pgfqpoint{3.444963in}{0.440000in}}%
\pgfpathlineto{\pgfqpoint{3.444963in}{3.520000in}}%
\pgfusepath{stroke}%
\end{pgfscope}%
\begin{pgfscope}%
\pgfpathrectangle{\pgfqpoint{1.000000in}{0.440000in}}{\pgfqpoint{6.200000in}{3.080000in}} %
\pgfusepath{clip}%
\pgfsetrectcap%
\pgfsetroundjoin%
\pgfsetlinewidth{0.501875pt}%
\definecolor{currentstroke}{rgb}{0.800000,0.000000,0.266667}%
\pgfsetstrokecolor{currentstroke}%
\pgfsetdash{}{0pt}%
\pgfpathmoveto{\pgfqpoint{3.901966in}{0.440000in}}%
\pgfpathlineto{\pgfqpoint{3.901966in}{3.520000in}}%
\pgfusepath{stroke}%
\end{pgfscope}%
\begin{pgfscope}%
\pgfpathrectangle{\pgfqpoint{1.000000in}{0.440000in}}{\pgfqpoint{6.200000in}{3.080000in}} %
\pgfusepath{clip}%
\pgfsetrectcap%
\pgfsetroundjoin%
\pgfsetlinewidth{0.501875pt}%
\definecolor{currentstroke}{rgb}{0.501961,0.501961,0.501961}%
\pgfsetstrokecolor{currentstroke}%
\pgfsetdash{}{0pt}%
\pgfpathmoveto{\pgfqpoint{3.521130in}{0.440000in}}%
\pgfpathlineto{\pgfqpoint{3.521130in}{3.520000in}}%
\pgfusepath{stroke}%
\end{pgfscope}%
\begin{pgfscope}%
\pgfpathrectangle{\pgfqpoint{1.000000in}{0.440000in}}{\pgfqpoint{6.200000in}{3.080000in}} %
\pgfusepath{clip}%
\pgfsetrectcap%
\pgfsetroundjoin%
\pgfsetlinewidth{0.501875pt}%
\definecolor{currentstroke}{rgb}{0.501961,0.501961,0.501961}%
\pgfsetstrokecolor{currentstroke}%
\pgfsetdash{}{0pt}%
\pgfpathmoveto{\pgfqpoint{3.597297in}{0.440000in}}%
\pgfpathlineto{\pgfqpoint{3.597297in}{3.520000in}}%
\pgfusepath{stroke}%
\end{pgfscope}%
\begin{pgfscope}%
\pgfpathrectangle{\pgfqpoint{1.000000in}{0.440000in}}{\pgfqpoint{6.200000in}{3.080000in}} %
\pgfusepath{clip}%
\pgfsetrectcap%
\pgfsetroundjoin%
\pgfsetlinewidth{0.501875pt}%
\definecolor{currentstroke}{rgb}{0.501961,0.501961,0.501961}%
\pgfsetstrokecolor{currentstroke}%
\pgfsetdash{}{0pt}%
\pgfpathmoveto{\pgfqpoint{3.673464in}{0.440000in}}%
\pgfpathlineto{\pgfqpoint{3.673464in}{3.520000in}}%
\pgfusepath{stroke}%
\end{pgfscope}%
\begin{pgfscope}%
\pgfpathrectangle{\pgfqpoint{1.000000in}{0.440000in}}{\pgfqpoint{6.200000in}{3.080000in}} %
\pgfusepath{clip}%
\pgfsetrectcap%
\pgfsetroundjoin%
\pgfsetlinewidth{0.501875pt}%
\definecolor{currentstroke}{rgb}{0.501961,0.501961,0.501961}%
\pgfsetstrokecolor{currentstroke}%
\pgfsetdash{}{0pt}%
\pgfpathmoveto{\pgfqpoint{3.749631in}{0.440000in}}%
\pgfpathlineto{\pgfqpoint{3.749631in}{3.520000in}}%
\pgfusepath{stroke}%
\end{pgfscope}%
\begin{pgfscope}%
\pgfpathrectangle{\pgfqpoint{1.000000in}{0.440000in}}{\pgfqpoint{6.200000in}{3.080000in}} %
\pgfusepath{clip}%
\pgfsetrectcap%
\pgfsetroundjoin%
\pgfsetlinewidth{0.501875pt}%
\definecolor{currentstroke}{rgb}{0.501961,0.501961,0.501961}%
\pgfsetstrokecolor{currentstroke}%
\pgfsetdash{}{0pt}%
\pgfpathmoveto{\pgfqpoint{3.825799in}{0.440000in}}%
\pgfpathlineto{\pgfqpoint{3.825799in}{3.520000in}}%
\pgfusepath{stroke}%
\end{pgfscope}%
\begin{pgfscope}%
\pgfpathrectangle{\pgfqpoint{1.000000in}{0.440000in}}{\pgfqpoint{6.200000in}{3.080000in}} %
\pgfusepath{clip}%
\pgfsetrectcap%
\pgfsetroundjoin%
\pgfsetlinewidth{0.501875pt}%
\definecolor{currentstroke}{rgb}{0.800000,0.000000,0.266667}%
\pgfsetstrokecolor{currentstroke}%
\pgfsetdash{}{0pt}%
\pgfpathmoveto{\pgfqpoint{3.901966in}{0.440000in}}%
\pgfpathlineto{\pgfqpoint{3.901966in}{3.520000in}}%
\pgfusepath{stroke}%
\end{pgfscope}%
\begin{pgfscope}%
\pgfpathrectangle{\pgfqpoint{1.000000in}{0.440000in}}{\pgfqpoint{6.200000in}{3.080000in}} %
\pgfusepath{clip}%
\pgfsetrectcap%
\pgfsetroundjoin%
\pgfsetlinewidth{0.501875pt}%
\definecolor{currentstroke}{rgb}{0.800000,0.000000,0.266667}%
\pgfsetstrokecolor{currentstroke}%
\pgfsetdash{}{0pt}%
\pgfpathmoveto{\pgfqpoint{5.059705in}{0.440000in}}%
\pgfpathlineto{\pgfqpoint{5.059705in}{3.520000in}}%
\pgfusepath{stroke}%
\end{pgfscope}%
\begin{pgfscope}%
\pgfpathrectangle{\pgfqpoint{1.000000in}{0.440000in}}{\pgfqpoint{6.200000in}{3.080000in}} %
\pgfusepath{clip}%
\pgfsetrectcap%
\pgfsetroundjoin%
\pgfsetlinewidth{0.501875pt}%
\definecolor{currentstroke}{rgb}{0.501961,0.501961,0.501961}%
\pgfsetstrokecolor{currentstroke}%
\pgfsetdash{}{0pt}%
\pgfpathmoveto{\pgfqpoint{4.094922in}{0.440000in}}%
\pgfpathlineto{\pgfqpoint{4.094922in}{3.520000in}}%
\pgfusepath{stroke}%
\end{pgfscope}%
\begin{pgfscope}%
\pgfpathrectangle{\pgfqpoint{1.000000in}{0.440000in}}{\pgfqpoint{6.200000in}{3.080000in}} %
\pgfusepath{clip}%
\pgfsetrectcap%
\pgfsetroundjoin%
\pgfsetlinewidth{0.501875pt}%
\definecolor{currentstroke}{rgb}{0.501961,0.501961,0.501961}%
\pgfsetstrokecolor{currentstroke}%
\pgfsetdash{}{0pt}%
\pgfpathmoveto{\pgfqpoint{4.287879in}{0.440000in}}%
\pgfpathlineto{\pgfqpoint{4.287879in}{3.520000in}}%
\pgfusepath{stroke}%
\end{pgfscope}%
\begin{pgfscope}%
\pgfpathrectangle{\pgfqpoint{1.000000in}{0.440000in}}{\pgfqpoint{6.200000in}{3.080000in}} %
\pgfusepath{clip}%
\pgfsetrectcap%
\pgfsetroundjoin%
\pgfsetlinewidth{0.501875pt}%
\definecolor{currentstroke}{rgb}{0.501961,0.501961,0.501961}%
\pgfsetstrokecolor{currentstroke}%
\pgfsetdash{}{0pt}%
\pgfpathmoveto{\pgfqpoint{4.480835in}{0.440000in}}%
\pgfpathlineto{\pgfqpoint{4.480835in}{3.520000in}}%
\pgfusepath{stroke}%
\end{pgfscope}%
\begin{pgfscope}%
\pgfpathrectangle{\pgfqpoint{1.000000in}{0.440000in}}{\pgfqpoint{6.200000in}{3.080000in}} %
\pgfusepath{clip}%
\pgfsetrectcap%
\pgfsetroundjoin%
\pgfsetlinewidth{0.501875pt}%
\definecolor{currentstroke}{rgb}{0.501961,0.501961,0.501961}%
\pgfsetstrokecolor{currentstroke}%
\pgfsetdash{}{0pt}%
\pgfpathmoveto{\pgfqpoint{4.673792in}{0.440000in}}%
\pgfpathlineto{\pgfqpoint{4.673792in}{3.520000in}}%
\pgfusepath{stroke}%
\end{pgfscope}%
\begin{pgfscope}%
\pgfpathrectangle{\pgfqpoint{1.000000in}{0.440000in}}{\pgfqpoint{6.200000in}{3.080000in}} %
\pgfusepath{clip}%
\pgfsetrectcap%
\pgfsetroundjoin%
\pgfsetlinewidth{0.501875pt}%
\definecolor{currentstroke}{rgb}{0.501961,0.501961,0.501961}%
\pgfsetstrokecolor{currentstroke}%
\pgfsetdash{}{0pt}%
\pgfpathmoveto{\pgfqpoint{4.866749in}{0.440000in}}%
\pgfpathlineto{\pgfqpoint{4.866749in}{3.520000in}}%
\pgfusepath{stroke}%
\end{pgfscope}%
\begin{pgfscope}%
\pgfpathrectangle{\pgfqpoint{1.000000in}{0.440000in}}{\pgfqpoint{6.200000in}{3.080000in}} %
\pgfusepath{clip}%
\pgfsetrectcap%
\pgfsetroundjoin%
\pgfsetlinewidth{0.501875pt}%
\definecolor{currentstroke}{rgb}{0.800000,0.000000,0.266667}%
\pgfsetstrokecolor{currentstroke}%
\pgfsetdash{}{0pt}%
\pgfpathmoveto{\pgfqpoint{5.059705in}{0.440000in}}%
\pgfpathlineto{\pgfqpoint{5.059705in}{3.520000in}}%
\pgfusepath{stroke}%
\end{pgfscope}%
\begin{pgfscope}%
\pgfpathrectangle{\pgfqpoint{1.000000in}{0.440000in}}{\pgfqpoint{6.200000in}{3.080000in}} %
\pgfusepath{clip}%
\pgfsetrectcap%
\pgfsetroundjoin%
\pgfsetlinewidth{0.501875pt}%
\definecolor{currentstroke}{rgb}{0.800000,0.000000,0.266667}%
\pgfsetstrokecolor{currentstroke}%
\pgfsetdash{}{0pt}%
\pgfpathmoveto{\pgfqpoint{6.765848in}{0.440000in}}%
\pgfpathlineto{\pgfqpoint{6.765848in}{3.520000in}}%
\pgfusepath{stroke}%
\end{pgfscope}%
\begin{pgfscope}%
\pgfpathrectangle{\pgfqpoint{1.000000in}{0.440000in}}{\pgfqpoint{6.200000in}{3.080000in}} %
\pgfusepath{clip}%
\pgfsetrectcap%
\pgfsetroundjoin%
\pgfsetlinewidth{0.501875pt}%
\definecolor{currentstroke}{rgb}{0.501961,0.501961,0.501961}%
\pgfsetstrokecolor{currentstroke}%
\pgfsetdash{}{0pt}%
\pgfpathmoveto{\pgfqpoint{5.344062in}{0.440000in}}%
\pgfpathlineto{\pgfqpoint{5.344062in}{3.520000in}}%
\pgfusepath{stroke}%
\end{pgfscope}%
\begin{pgfscope}%
\pgfpathrectangle{\pgfqpoint{1.000000in}{0.440000in}}{\pgfqpoint{6.200000in}{3.080000in}} %
\pgfusepath{clip}%
\pgfsetrectcap%
\pgfsetroundjoin%
\pgfsetlinewidth{0.501875pt}%
\definecolor{currentstroke}{rgb}{0.501961,0.501961,0.501961}%
\pgfsetstrokecolor{currentstroke}%
\pgfsetdash{}{0pt}%
\pgfpathmoveto{\pgfqpoint{5.628419in}{0.440000in}}%
\pgfpathlineto{\pgfqpoint{5.628419in}{3.520000in}}%
\pgfusepath{stroke}%
\end{pgfscope}%
\begin{pgfscope}%
\pgfpathrectangle{\pgfqpoint{1.000000in}{0.440000in}}{\pgfqpoint{6.200000in}{3.080000in}} %
\pgfusepath{clip}%
\pgfsetrectcap%
\pgfsetroundjoin%
\pgfsetlinewidth{0.501875pt}%
\definecolor{currentstroke}{rgb}{0.501961,0.501961,0.501961}%
\pgfsetstrokecolor{currentstroke}%
\pgfsetdash{}{0pt}%
\pgfpathmoveto{\pgfqpoint{5.912776in}{0.440000in}}%
\pgfpathlineto{\pgfqpoint{5.912776in}{3.520000in}}%
\pgfusepath{stroke}%
\end{pgfscope}%
\begin{pgfscope}%
\pgfpathrectangle{\pgfqpoint{1.000000in}{0.440000in}}{\pgfqpoint{6.200000in}{3.080000in}} %
\pgfusepath{clip}%
\pgfsetrectcap%
\pgfsetroundjoin%
\pgfsetlinewidth{0.501875pt}%
\definecolor{currentstroke}{rgb}{0.501961,0.501961,0.501961}%
\pgfsetstrokecolor{currentstroke}%
\pgfsetdash{}{0pt}%
\pgfpathmoveto{\pgfqpoint{6.197133in}{0.440000in}}%
\pgfpathlineto{\pgfqpoint{6.197133in}{3.520000in}}%
\pgfusepath{stroke}%
\end{pgfscope}%
\begin{pgfscope}%
\pgfpathrectangle{\pgfqpoint{1.000000in}{0.440000in}}{\pgfqpoint{6.200000in}{3.080000in}} %
\pgfusepath{clip}%
\pgfsetrectcap%
\pgfsetroundjoin%
\pgfsetlinewidth{0.501875pt}%
\definecolor{currentstroke}{rgb}{0.501961,0.501961,0.501961}%
\pgfsetstrokecolor{currentstroke}%
\pgfsetdash{}{0pt}%
\pgfpathmoveto{\pgfqpoint{6.481491in}{0.440000in}}%
\pgfpathlineto{\pgfqpoint{6.481491in}{3.520000in}}%
\pgfusepath{stroke}%
\end{pgfscope}%
\begin{pgfscope}%
\pgfpathrectangle{\pgfqpoint{1.000000in}{0.440000in}}{\pgfqpoint{6.200000in}{3.080000in}} %
\pgfusepath{clip}%
\pgfsetrectcap%
\pgfsetroundjoin%
\pgfsetlinewidth{0.501875pt}%
\definecolor{currentstroke}{rgb}{0.800000,0.000000,0.266667}%
\pgfsetstrokecolor{currentstroke}%
\pgfsetdash{}{0pt}%
\pgfpathmoveto{\pgfqpoint{6.765848in}{0.440000in}}%
\pgfpathlineto{\pgfqpoint{6.765848in}{3.520000in}}%
\pgfusepath{stroke}%
\end{pgfscope}%
\begin{pgfscope}%
\pgfpathrectangle{\pgfqpoint{1.000000in}{0.440000in}}{\pgfqpoint{6.200000in}{3.080000in}} %
\pgfusepath{clip}%
\pgfsetrectcap%
\pgfsetroundjoin%
\pgfsetlinewidth{1.505625pt}%
\definecolor{currentstroke}{rgb}{0.364706,0.517647,0.572549}%
\pgfsetstrokecolor{currentstroke}%
\pgfsetdash{}{0pt}%
\pgfpathmoveto{\pgfqpoint{1.281818in}{1.169484in}}%
\pgfpathlineto{\pgfqpoint{1.997789in}{1.169484in}}%
\pgfpathlineto{\pgfqpoint{2.013022in}{1.180018in}}%
\pgfpathlineto{\pgfqpoint{2.028256in}{1.213117in}}%
\pgfpathlineto{\pgfqpoint{2.043489in}{1.243508in}}%
\pgfpathlineto{\pgfqpoint{2.058722in}{1.249626in}}%
\pgfpathlineto{\pgfqpoint{2.073956in}{1.241922in}}%
\pgfpathlineto{\pgfqpoint{2.104423in}{1.216067in}}%
\pgfpathlineto{\pgfqpoint{2.119656in}{1.180239in}}%
\pgfpathlineto{\pgfqpoint{2.134889in}{1.096283in}}%
\pgfpathlineto{\pgfqpoint{2.150123in}{1.000565in}}%
\pgfpathlineto{\pgfqpoint{2.165356in}{0.981155in}}%
\pgfpathlineto{\pgfqpoint{2.180590in}{1.022198in}}%
\pgfpathlineto{\pgfqpoint{2.195823in}{1.051990in}}%
\pgfpathlineto{\pgfqpoint{2.211057in}{1.073655in}}%
\pgfpathlineto{\pgfqpoint{2.226290in}{1.117670in}}%
\pgfpathlineto{\pgfqpoint{2.241523in}{1.190799in}}%
\pgfpathlineto{\pgfqpoint{2.256757in}{1.283890in}}%
\pgfpathlineto{\pgfqpoint{2.287224in}{1.477771in}}%
\pgfpathlineto{\pgfqpoint{2.302457in}{1.553343in}}%
\pgfpathlineto{\pgfqpoint{2.317690in}{1.610443in}}%
\pgfpathlineto{\pgfqpoint{2.332924in}{1.660888in}}%
\pgfpathlineto{\pgfqpoint{2.348157in}{1.726165in}}%
\pgfpathlineto{\pgfqpoint{2.363391in}{1.855295in}}%
\pgfpathlineto{\pgfqpoint{2.378624in}{2.077817in}}%
\pgfpathlineto{\pgfqpoint{2.393857in}{2.280446in}}%
\pgfpathlineto{\pgfqpoint{2.409091in}{2.393129in}}%
\pgfpathlineto{\pgfqpoint{2.439558in}{2.525077in}}%
\pgfpathlineto{\pgfqpoint{2.454791in}{2.620606in}}%
\pgfpathlineto{\pgfqpoint{2.470025in}{2.727577in}}%
\pgfpathlineto{\pgfqpoint{2.485258in}{2.823790in}}%
\pgfpathlineto{\pgfqpoint{2.500491in}{2.910174in}}%
\pgfpathlineto{\pgfqpoint{2.530958in}{3.061822in}}%
\pgfpathlineto{\pgfqpoint{2.561425in}{3.243144in}}%
\pgfpathlineto{\pgfqpoint{2.576658in}{3.315514in}}%
\pgfpathlineto{\pgfqpoint{2.591892in}{3.372327in}}%
\pgfpathlineto{\pgfqpoint{2.607125in}{3.380000in}}%
\pgfpathlineto{\pgfqpoint{2.622359in}{3.283246in}}%
\pgfpathlineto{\pgfqpoint{2.637592in}{3.165543in}}%
\pgfpathlineto{\pgfqpoint{2.652826in}{3.121637in}}%
\pgfpathlineto{\pgfqpoint{2.698526in}{3.099111in}}%
\pgfpathlineto{\pgfqpoint{2.713759in}{3.086715in}}%
\pgfpathlineto{\pgfqpoint{2.774693in}{2.995830in}}%
\pgfpathlineto{\pgfqpoint{2.835627in}{2.850865in}}%
\pgfpathlineto{\pgfqpoint{2.896560in}{2.655827in}}%
\pgfpathlineto{\pgfqpoint{2.927027in}{2.538655in}}%
\pgfpathlineto{\pgfqpoint{2.972727in}{2.360070in}}%
\pgfpathlineto{\pgfqpoint{3.018428in}{2.189141in}}%
\pgfpathlineto{\pgfqpoint{3.033661in}{2.137035in}}%
\pgfpathlineto{\pgfqpoint{3.094595in}{1.969662in}}%
\pgfpathlineto{\pgfqpoint{3.155528in}{1.853076in}}%
\pgfpathlineto{\pgfqpoint{3.201229in}{1.801005in}}%
\pgfpathlineto{\pgfqpoint{3.216462in}{1.783924in}}%
\pgfpathlineto{\pgfqpoint{3.231695in}{1.775219in}}%
\pgfpathlineto{\pgfqpoint{3.277396in}{1.758339in}}%
\pgfpathlineto{\pgfqpoint{3.292629in}{1.722644in}}%
\pgfpathlineto{\pgfqpoint{3.307862in}{1.673207in}}%
\pgfpathlineto{\pgfqpoint{3.323096in}{1.630925in}}%
\pgfpathlineto{\pgfqpoint{3.338329in}{1.596906in}}%
\pgfpathlineto{\pgfqpoint{3.353563in}{1.570437in}}%
\pgfpathlineto{\pgfqpoint{3.384029in}{1.526751in}}%
\pgfpathlineto{\pgfqpoint{3.399263in}{1.498475in}}%
\pgfpathlineto{\pgfqpoint{3.429730in}{1.406330in}}%
\pgfpathlineto{\pgfqpoint{3.444963in}{1.392157in}}%
\pgfpathlineto{\pgfqpoint{3.460197in}{1.388183in}}%
\pgfpathlineto{\pgfqpoint{3.505897in}{1.337844in}}%
\pgfpathlineto{\pgfqpoint{3.536364in}{1.315695in}}%
\pgfpathlineto{\pgfqpoint{3.551597in}{1.306303in}}%
\pgfpathlineto{\pgfqpoint{3.566830in}{1.309460in}}%
\pgfpathlineto{\pgfqpoint{3.582064in}{1.356383in}}%
\pgfpathlineto{\pgfqpoint{3.597297in}{1.442296in}}%
\pgfpathlineto{\pgfqpoint{3.612531in}{1.498590in}}%
\pgfpathlineto{\pgfqpoint{3.627764in}{1.509395in}}%
\pgfpathlineto{\pgfqpoint{3.642998in}{1.504952in}}%
\pgfpathlineto{\pgfqpoint{3.658231in}{1.497308in}}%
\pgfpathlineto{\pgfqpoint{3.673464in}{1.491746in}}%
\pgfpathlineto{\pgfqpoint{3.703931in}{1.484544in}}%
\pgfpathlineto{\pgfqpoint{3.719165in}{1.485587in}}%
\pgfpathlineto{\pgfqpoint{3.734398in}{1.492379in}}%
\pgfpathlineto{\pgfqpoint{3.749631in}{1.497810in}}%
\pgfpathlineto{\pgfqpoint{3.764865in}{1.495599in}}%
\pgfpathlineto{\pgfqpoint{3.780098in}{1.488639in}}%
\pgfpathlineto{\pgfqpoint{3.795332in}{1.479567in}}%
\pgfpathlineto{\pgfqpoint{3.810565in}{1.462231in}}%
\pgfpathlineto{\pgfqpoint{3.825799in}{1.425528in}}%
\pgfpathlineto{\pgfqpoint{3.841032in}{1.367895in}}%
\pgfpathlineto{\pgfqpoint{3.856265in}{1.338152in}}%
\pgfpathlineto{\pgfqpoint{3.871499in}{1.346327in}}%
\pgfpathlineto{\pgfqpoint{3.886732in}{1.336390in}}%
\pgfpathlineto{\pgfqpoint{3.917199in}{1.267346in}}%
\pgfpathlineto{\pgfqpoint{3.947666in}{1.212521in}}%
\pgfpathlineto{\pgfqpoint{3.962899in}{1.178892in}}%
\pgfpathlineto{\pgfqpoint{3.993366in}{1.100211in}}%
\pgfpathlineto{\pgfqpoint{4.008600in}{1.060065in}}%
\pgfpathlineto{\pgfqpoint{4.023833in}{1.024709in}}%
\pgfpathlineto{\pgfqpoint{4.039066in}{0.999380in}}%
\pgfpathlineto{\pgfqpoint{4.054300in}{0.996954in}}%
\pgfpathlineto{\pgfqpoint{4.084767in}{1.071116in}}%
\pgfpathlineto{\pgfqpoint{4.100000in}{1.068886in}}%
\pgfpathlineto{\pgfqpoint{4.145700in}{1.001995in}}%
\pgfpathlineto{\pgfqpoint{4.160934in}{0.973546in}}%
\pgfpathlineto{\pgfqpoint{4.176167in}{0.947798in}}%
\pgfpathlineto{\pgfqpoint{4.191400in}{0.931365in}}%
\pgfpathlineto{\pgfqpoint{4.206634in}{0.924262in}}%
\pgfpathlineto{\pgfqpoint{4.221867in}{0.925338in}}%
\pgfpathlineto{\pgfqpoint{4.237101in}{0.938405in}}%
\pgfpathlineto{\pgfqpoint{4.252334in}{0.961615in}}%
\pgfpathlineto{\pgfqpoint{4.267568in}{0.991227in}}%
\pgfpathlineto{\pgfqpoint{4.282801in}{1.028989in}}%
\pgfpathlineto{\pgfqpoint{4.298034in}{1.062745in}}%
\pgfpathlineto{\pgfqpoint{4.313268in}{1.084881in}}%
\pgfpathlineto{\pgfqpoint{4.343735in}{1.222592in}}%
\pgfpathlineto{\pgfqpoint{4.358968in}{1.280546in}}%
\pgfpathlineto{\pgfqpoint{4.374201in}{1.343299in}}%
\pgfpathlineto{\pgfqpoint{4.404668in}{1.409018in}}%
\pgfpathlineto{\pgfqpoint{4.419902in}{1.447560in}}%
\pgfpathlineto{\pgfqpoint{4.435135in}{1.462780in}}%
\pgfpathlineto{\pgfqpoint{4.450369in}{1.459128in}}%
\pgfpathlineto{\pgfqpoint{4.465602in}{1.451235in}}%
\pgfpathlineto{\pgfqpoint{4.480835in}{1.446877in}}%
\pgfpathlineto{\pgfqpoint{4.572236in}{1.439168in}}%
\pgfpathlineto{\pgfqpoint{4.663636in}{1.416321in}}%
\pgfpathlineto{\pgfqpoint{4.678870in}{1.411657in}}%
\pgfpathlineto{\pgfqpoint{4.770270in}{1.373638in}}%
\pgfpathlineto{\pgfqpoint{4.800737in}{1.356787in}}%
\pgfpathlineto{\pgfqpoint{4.876904in}{1.312738in}}%
\pgfpathlineto{\pgfqpoint{5.059705in}{1.181455in}}%
\pgfpathlineto{\pgfqpoint{5.074939in}{1.170636in}}%
\pgfpathlineto{\pgfqpoint{5.166339in}{1.120895in}}%
\pgfpathlineto{\pgfqpoint{5.181572in}{1.113443in}}%
\pgfpathlineto{\pgfqpoint{5.272973in}{1.078524in}}%
\pgfpathlineto{\pgfqpoint{5.303440in}{1.070970in}}%
\pgfpathlineto{\pgfqpoint{5.379607in}{1.053974in}}%
\pgfpathlineto{\pgfqpoint{5.471007in}{1.047085in}}%
\pgfpathlineto{\pgfqpoint{5.486241in}{1.038835in}}%
\pgfpathlineto{\pgfqpoint{5.516708in}{1.017903in}}%
\pgfpathlineto{\pgfqpoint{5.547174in}{0.987236in}}%
\pgfpathlineto{\pgfqpoint{5.562408in}{0.972718in}}%
\pgfpathlineto{\pgfqpoint{5.592875in}{0.922519in}}%
\pgfpathlineto{\pgfqpoint{5.608108in}{0.905856in}}%
\pgfpathlineto{\pgfqpoint{5.623342in}{0.870672in}}%
\pgfpathlineto{\pgfqpoint{5.638575in}{0.823965in}}%
\pgfpathlineto{\pgfqpoint{5.653808in}{0.794213in}}%
\pgfpathlineto{\pgfqpoint{5.684275in}{0.761011in}}%
\pgfpathlineto{\pgfqpoint{5.714742in}{0.673842in}}%
\pgfpathlineto{\pgfqpoint{5.745209in}{0.632286in}}%
\pgfpathlineto{\pgfqpoint{5.760442in}{0.605600in}}%
\pgfpathlineto{\pgfqpoint{5.775676in}{0.591469in}}%
\pgfpathlineto{\pgfqpoint{5.790909in}{0.607326in}}%
\pgfpathlineto{\pgfqpoint{5.806143in}{0.626000in}}%
\pgfpathlineto{\pgfqpoint{5.821376in}{0.632378in}}%
\pgfpathlineto{\pgfqpoint{5.836609in}{0.640955in}}%
\pgfpathlineto{\pgfqpoint{5.851843in}{0.642687in}}%
\pgfpathlineto{\pgfqpoint{5.867076in}{0.627918in}}%
\pgfpathlineto{\pgfqpoint{5.897543in}{0.580000in}}%
\pgfpathlineto{\pgfqpoint{5.912776in}{0.588304in}}%
\pgfpathlineto{\pgfqpoint{5.928010in}{0.629685in}}%
\pgfpathlineto{\pgfqpoint{5.943243in}{0.631689in}}%
\pgfpathlineto{\pgfqpoint{5.958477in}{0.607162in}}%
\pgfpathlineto{\pgfqpoint{5.973710in}{0.608787in}}%
\pgfpathlineto{\pgfqpoint{5.988943in}{0.619218in}}%
\pgfpathlineto{\pgfqpoint{6.004177in}{0.619497in}}%
\pgfpathlineto{\pgfqpoint{6.019410in}{0.622254in}}%
\pgfpathlineto{\pgfqpoint{6.034644in}{0.631110in}}%
\pgfpathlineto{\pgfqpoint{6.065111in}{0.642581in}}%
\pgfpathlineto{\pgfqpoint{6.080344in}{0.658514in}}%
\pgfpathlineto{\pgfqpoint{6.095577in}{0.684377in}}%
\pgfpathlineto{\pgfqpoint{6.110811in}{0.722313in}}%
\pgfpathlineto{\pgfqpoint{6.126044in}{0.772044in}}%
\pgfpathlineto{\pgfqpoint{6.141278in}{0.812010in}}%
\pgfpathlineto{\pgfqpoint{6.156511in}{0.821943in}}%
\pgfpathlineto{\pgfqpoint{6.171744in}{0.811626in}}%
\pgfpathlineto{\pgfqpoint{6.186978in}{0.796743in}}%
\pgfpathlineto{\pgfqpoint{6.202211in}{0.789968in}}%
\pgfpathlineto{\pgfqpoint{6.918182in}{0.789968in}}%
\pgfpathlineto{\pgfqpoint{6.918182in}{0.789968in}}%
\pgfusepath{stroke}%
\end{pgfscope}%
\begin{pgfscope}%
\pgfpathrectangle{\pgfqpoint{1.000000in}{0.440000in}}{\pgfqpoint{6.200000in}{3.080000in}} %
\pgfusepath{clip}%
\pgfsetbuttcap%
\pgfsetroundjoin%
\definecolor{currentfill}{rgb}{0.364706,0.517647,0.572549}%
\pgfsetfillcolor{currentfill}%
\pgfsetlinewidth{1.003750pt}%
\definecolor{currentstroke}{rgb}{0.364706,0.517647,0.572549}%
\pgfsetstrokecolor{currentstroke}%
\pgfsetdash{}{0pt}%
\pgfsys@defobject{currentmarker}{\pgfqpoint{-0.027778in}{-0.027778in}}{\pgfqpoint{0.027778in}{0.027778in}}{%
\pgfpathmoveto{\pgfqpoint{0.000000in}{-0.027778in}}%
\pgfpathcurveto{\pgfqpoint{0.007367in}{-0.027778in}}{\pgfqpoint{0.014433in}{-0.024851in}}{\pgfqpoint{0.019642in}{-0.019642in}}%
\pgfpathcurveto{\pgfqpoint{0.024851in}{-0.014433in}}{\pgfqpoint{0.027778in}{-0.007367in}}{\pgfqpoint{0.027778in}{0.000000in}}%
\pgfpathcurveto{\pgfqpoint{0.027778in}{0.007367in}}{\pgfqpoint{0.024851in}{0.014433in}}{\pgfqpoint{0.019642in}{0.019642in}}%
\pgfpathcurveto{\pgfqpoint{0.014433in}{0.024851in}}{\pgfqpoint{0.007367in}{0.027778in}}{\pgfqpoint{0.000000in}{0.027778in}}%
\pgfpathcurveto{\pgfqpoint{-0.007367in}{0.027778in}}{\pgfqpoint{-0.014433in}{0.024851in}}{\pgfqpoint{-0.019642in}{0.019642in}}%
\pgfpathcurveto{\pgfqpoint{-0.024851in}{0.014433in}}{\pgfqpoint{-0.027778in}{0.007367in}}{\pgfqpoint{-0.027778in}{0.000000in}}%
\pgfpathcurveto{\pgfqpoint{-0.027778in}{-0.007367in}}{\pgfqpoint{-0.024851in}{-0.014433in}}{\pgfqpoint{-0.019642in}{-0.019642in}}%
\pgfpathcurveto{\pgfqpoint{-0.014433in}{-0.024851in}}{\pgfqpoint{-0.007367in}{-0.027778in}}{\pgfqpoint{0.000000in}{-0.027778in}}%
\pgfpathclose%
\pgfusepath{stroke,fill}%
}%
\begin{pgfscope}%
\pgfsys@transformshift{1.769287in}{1.169484in}%
\pgfsys@useobject{currentmarker}{}%
\end{pgfscope}%
\begin{pgfscope}%
\pgfsys@transformshift{1.835299in}{1.169484in}%
\pgfsys@useobject{currentmarker}{}%
\end{pgfscope}%
\begin{pgfscope}%
\pgfsys@transformshift{1.901310in}{1.169484in}%
\pgfsys@useobject{currentmarker}{}%
\end{pgfscope}%
\begin{pgfscope}%
\pgfsys@transformshift{1.967322in}{1.169484in}%
\pgfsys@useobject{currentmarker}{}%
\end{pgfscope}%
\begin{pgfscope}%
\pgfsys@transformshift{2.033333in}{1.213117in}%
\pgfsys@useobject{currentmarker}{}%
\end{pgfscope}%
\begin{pgfscope}%
\pgfsys@transformshift{2.099345in}{1.229511in}%
\pgfsys@useobject{currentmarker}{}%
\end{pgfscope}%
\begin{pgfscope}%
\pgfsys@transformshift{2.165356in}{1.000565in}%
\pgfsys@useobject{currentmarker}{}%
\end{pgfscope}%
\begin{pgfscope}%
\pgfsys@transformshift{2.241523in}{1.190799in}%
\pgfsys@useobject{currentmarker}{}%
\end{pgfscope}%
\begin{pgfscope}%
\pgfsys@transformshift{2.317690in}{1.610443in}%
\pgfsys@useobject{currentmarker}{}%
\end{pgfscope}%
\begin{pgfscope}%
\pgfsys@transformshift{2.393857in}{2.280446in}%
\pgfsys@useobject{currentmarker}{}%
\end{pgfscope}%
\begin{pgfscope}%
\pgfsys@transformshift{2.470025in}{2.727577in}%
\pgfsys@useobject{currentmarker}{}%
\end{pgfscope}%
\begin{pgfscope}%
\pgfsys@transformshift{2.546192in}{3.154340in}%
\pgfsys@useobject{currentmarker}{}%
\end{pgfscope}%
\begin{pgfscope}%
\pgfsys@transformshift{2.622359in}{3.283246in}%
\pgfsys@useobject{currentmarker}{}%
\end{pgfscope}%
\begin{pgfscope}%
\pgfsys@transformshift{2.759459in}{3.019928in}%
\pgfsys@useobject{currentmarker}{}%
\end{pgfscope}%
\begin{pgfscope}%
\pgfsys@transformshift{2.896560in}{2.655827in}%
\pgfsys@useobject{currentmarker}{}%
\end{pgfscope}%
\begin{pgfscope}%
\pgfsys@transformshift{3.033661in}{2.137035in}%
\pgfsys@useobject{currentmarker}{}%
\end{pgfscope}%
\begin{pgfscope}%
\pgfsys@transformshift{3.170762in}{1.835166in}%
\pgfsys@useobject{currentmarker}{}%
\end{pgfscope}%
\begin{pgfscope}%
\pgfsys@transformshift{3.307862in}{1.673207in}%
\pgfsys@useobject{currentmarker}{}%
\end{pgfscope}%
\begin{pgfscope}%
\pgfsys@transformshift{3.444963in}{1.392157in}%
\pgfsys@useobject{currentmarker}{}%
\end{pgfscope}%
\begin{pgfscope}%
\pgfsys@transformshift{3.521130in}{1.326289in}%
\pgfsys@useobject{currentmarker}{}%
\end{pgfscope}%
\begin{pgfscope}%
\pgfsys@transformshift{3.597297in}{1.442296in}%
\pgfsys@useobject{currentmarker}{}%
\end{pgfscope}%
\begin{pgfscope}%
\pgfsys@transformshift{3.673464in}{1.491746in}%
\pgfsys@useobject{currentmarker}{}%
\end{pgfscope}%
\begin{pgfscope}%
\pgfsys@transformshift{3.749631in}{1.497810in}%
\pgfsys@useobject{currentmarker}{}%
\end{pgfscope}%
\begin{pgfscope}%
\pgfsys@transformshift{3.825799in}{1.425528in}%
\pgfsys@useobject{currentmarker}{}%
\end{pgfscope}%
\begin{pgfscope}%
\pgfsys@transformshift{3.901966in}{1.300541in}%
\pgfsys@useobject{currentmarker}{}%
\end{pgfscope}%
\begin{pgfscope}%
\pgfsys@transformshift{4.094922in}{1.071116in}%
\pgfsys@useobject{currentmarker}{}%
\end{pgfscope}%
\begin{pgfscope}%
\pgfsys@transformshift{4.287879in}{1.028989in}%
\pgfsys@useobject{currentmarker}{}%
\end{pgfscope}%
\begin{pgfscope}%
\pgfsys@transformshift{4.480835in}{1.446877in}%
\pgfsys@useobject{currentmarker}{}%
\end{pgfscope}%
\begin{pgfscope}%
\pgfsys@transformshift{4.673792in}{1.416321in}%
\pgfsys@useobject{currentmarker}{}%
\end{pgfscope}%
\begin{pgfscope}%
\pgfsys@transformshift{4.866749in}{1.321789in}%
\pgfsys@useobject{currentmarker}{}%
\end{pgfscope}%
\begin{pgfscope}%
\pgfsys@transformshift{5.059705in}{1.181455in}%
\pgfsys@useobject{currentmarker}{}%
\end{pgfscope}%
\begin{pgfscope}%
\pgfsys@transformshift{5.344062in}{1.064050in}%
\pgfsys@useobject{currentmarker}{}%
\end{pgfscope}%
\begin{pgfscope}%
\pgfsys@transformshift{5.628419in}{0.870672in}%
\pgfsys@useobject{currentmarker}{}%
\end{pgfscope}%
\begin{pgfscope}%
\pgfsys@transformshift{5.912776in}{0.580000in}%
\pgfsys@useobject{currentmarker}{}%
\end{pgfscope}%
\begin{pgfscope}%
\pgfsys@transformshift{6.197133in}{0.796743in}%
\pgfsys@useobject{currentmarker}{}%
\end{pgfscope}%
\begin{pgfscope}%
\pgfsys@transformshift{6.481491in}{0.789968in}%
\pgfsys@useobject{currentmarker}{}%
\end{pgfscope}%
\begin{pgfscope}%
\pgfsys@transformshift{6.765848in}{0.789968in}%
\pgfsys@useobject{currentmarker}{}%
\end{pgfscope}%
\end{pgfscope}%
\begin{pgfscope}%
\pgfsetrectcap%
\pgfsetmiterjoin%
\pgfsetlinewidth{0.803000pt}%
\definecolor{currentstroke}{rgb}{0.000000,0.000000,0.000000}%
\pgfsetstrokecolor{currentstroke}%
\pgfsetdash{}{0pt}%
\pgfpathmoveto{\pgfqpoint{1.000000in}{0.440000in}}%
\pgfpathlineto{\pgfqpoint{1.000000in}{3.520000in}}%
\pgfusepath{stroke}%
\end{pgfscope}%
\begin{pgfscope}%
\pgfsetrectcap%
\pgfsetmiterjoin%
\pgfsetlinewidth{0.803000pt}%
\definecolor{currentstroke}{rgb}{0.000000,0.000000,0.000000}%
\pgfsetstrokecolor{currentstroke}%
\pgfsetdash{}{0pt}%
\pgfpathmoveto{\pgfqpoint{7.200000in}{0.440000in}}%
\pgfpathlineto{\pgfqpoint{7.200000in}{3.520000in}}%
\pgfusepath{stroke}%
\end{pgfscope}%
\begin{pgfscope}%
\pgfsetrectcap%
\pgfsetmiterjoin%
\pgfsetlinewidth{0.803000pt}%
\definecolor{currentstroke}{rgb}{0.000000,0.000000,0.000000}%
\pgfsetstrokecolor{currentstroke}%
\pgfsetdash{}{0pt}%
\pgfpathmoveto{\pgfqpoint{1.000000in}{0.440000in}}%
\pgfpathlineto{\pgfqpoint{7.200000in}{0.440000in}}%
\pgfusepath{stroke}%
\end{pgfscope}%
\begin{pgfscope}%
\pgfsetrectcap%
\pgfsetmiterjoin%
\pgfsetlinewidth{0.803000pt}%
\definecolor{currentstroke}{rgb}{0.000000,0.000000,0.000000}%
\pgfsetstrokecolor{currentstroke}%
\pgfsetdash{}{0pt}%
\pgfpathmoveto{\pgfqpoint{1.000000in}{3.520000in}}%
\pgfpathlineto{\pgfqpoint{7.200000in}{3.520000in}}%
\pgfusepath{stroke}%
\end{pgfscope}%
\begin{pgfscope}%
\pgftext[x=1.835299in,y=0.580000in,left,base]{\rmfamily\fontsize{12.000000}{14.400000}\selectfont @}%
\end{pgfscope}%
\begin{pgfscope}%
\pgftext[x=2.241523in,y=0.580000in,left,base]{\rmfamily\fontsize{12.000000}{14.400000}\selectfont mid}%
\end{pgfscope}%
\begin{pgfscope}%
\pgftext[x=2.759459in,y=0.580000in,left,base]{\rmfamily\fontsize{12.000000}{14.400000}\selectfont suh}%
\end{pgfscope}%
\begin{pgfscope}%
\pgftext[x=3.521130in,y=0.580000in,left,base]{\rmfamily\fontsize{12.000000}{14.400000}\selectfont m@}%
\end{pgfscope}%
\begin{pgfscope}%
\pgftext[x=4.094922in,y=0.580000in,left,base]{\rmfamily\fontsize{12.000000}{14.400000}\selectfont naits}%
\end{pgfscope}%
\begin{pgfscope}%
\pgftext[x=5.344062in,y=0.580000in,left,base]{\rmfamily\fontsize{12.000000}{14.400000}\selectfont driim}%
\end{pgfscope}%
\begin{pgfscope}%
\pgfsetbuttcap%
\pgfsetmiterjoin%
\definecolor{currentfill}{rgb}{1.000000,1.000000,1.000000}%
\pgfsetfillcolor{currentfill}%
\pgfsetfillopacity{0.800000}%
\pgfsetlinewidth{1.003750pt}%
\definecolor{currentstroke}{rgb}{0.800000,0.800000,0.800000}%
\pgfsetstrokecolor{currentstroke}%
\pgfsetstrokeopacity{0.800000}%
\pgfsetdash{}{0pt}%
\pgfpathmoveto{\pgfqpoint{5.541322in}{3.021556in}}%
\pgfpathlineto{\pgfqpoint{7.102778in}{3.021556in}}%
\pgfpathquadraticcurveto{\pgfqpoint{7.130556in}{3.021556in}}{\pgfqpoint{7.130556in}{3.049334in}}%
\pgfpathlineto{\pgfqpoint{7.130556in}{3.422778in}}%
\pgfpathquadraticcurveto{\pgfqpoint{7.130556in}{3.450556in}}{\pgfqpoint{7.102778in}{3.450556in}}%
\pgfpathlineto{\pgfqpoint{5.541322in}{3.450556in}}%
\pgfpathquadraticcurveto{\pgfqpoint{5.513544in}{3.450556in}}{\pgfqpoint{5.513544in}{3.422778in}}%
\pgfpathlineto{\pgfqpoint{5.513544in}{3.049334in}}%
\pgfpathquadraticcurveto{\pgfqpoint{5.513544in}{3.021556in}}{\pgfqpoint{5.541322in}{3.021556in}}%
\pgfpathclose%
\pgfusepath{stroke,fill}%
\end{pgfscope}%
\begin{pgfscope}%
\pgfsetbuttcap%
\pgfsetmiterjoin%
\definecolor{currentfill}{rgb}{0.364706,0.517647,0.572549}%
\pgfsetfillcolor{currentfill}%
\pgfsetlinewidth{1.003750pt}%
\definecolor{currentstroke}{rgb}{0.364706,0.517647,0.572549}%
\pgfsetstrokecolor{currentstroke}%
\pgfsetdash{}{0pt}%
\pgfpathmoveto{\pgfqpoint{5.569100in}{3.297778in}}%
\pgfpathlineto{\pgfqpoint{5.846878in}{3.297778in}}%
\pgfpathlineto{\pgfqpoint{5.846878in}{3.395000in}}%
\pgfpathlineto{\pgfqpoint{5.569100in}{3.395000in}}%
\pgfpathclose%
\pgfusepath{stroke,fill}%
\end{pgfscope}%
\begin{pgfscope}%
\pgftext[x=5.957989in,y=3.297778in,left,base]{\rmfamily\fontsize{10.000000}{12.000000}\selectfont Interpolated \(\displaystyle F_0\)}%
\end{pgfscope}%
\begin{pgfscope}%
\pgfsetbuttcap%
\pgfsetmiterjoin%
\definecolor{currentfill}{rgb}{0.800000,0.000000,0.266667}%
\pgfsetfillcolor{currentfill}%
\pgfsetlinewidth{1.003750pt}%
\definecolor{currentstroke}{rgb}{0.800000,0.000000,0.266667}%
\pgfsetstrokecolor{currentstroke}%
\pgfsetdash{}{0pt}%
\pgfpathmoveto{\pgfqpoint{5.569100in}{3.104112in}}%
\pgfpathlineto{\pgfqpoint{5.846878in}{3.104112in}}%
\pgfpathlineto{\pgfqpoint{5.846878in}{3.201334in}}%
\pgfpathlineto{\pgfqpoint{5.569100in}{3.201334in}}%
\pgfpathclose%
\pgfusepath{stroke,fill}%
\end{pgfscope}%
\begin{pgfscope}%
\pgftext[x=5.957989in,y=3.104112in,left,base]{\rmfamily\fontsize{10.000000}{12.000000}\selectfont Syllable Boundary}%
\end{pgfscope}%
\end{pgfpicture}%
\makeatother%
\endgroup%
}
\caption[7 anchor-point sampling]{Location of extraction points in an \ac{F0} contour with 7 anchor points.}
\label{fig:high-anchor}
\end{figure}


\subsection{Adaptive Sampling Rate} \label{subsec:adaptive-sampling-rate}


As previously mentioned, frame-by-frame sampling rates have the advantage of making fewer assumptions about the data and are more conveniently implemented.
However, including rich inputs such as word embeddings is prohibitively expensive.
On the other hand, anchor points can provide us with a more consistent number of points across linguistic segments, but we cannot allocate more anchor points for longer segments.

\begin{figure}[h]
\centering
\resizebox{\textwidth}{!}{%% Creator: Matplotlib, PGF backend
%%
%% To include the figure in your LaTeX document, write
%%   \input{<filename>.pgf}
%%
%% Make sure the required packages are loaded in your preamble
%%   \usepackage{pgf}
%%
%% Figures using additional raster images can only be included by \input if
%% they are in the same directory as the main LaTeX file. For loading figures
%% from other directories you can use the `import` package
%%   \usepackage{import}
%% and then include the figures with
%%   \import{<path to file>}{<filename>.pgf}
%%
%% Matplotlib used the following preamble
%%   \usepackage[utf8x]{inputenc}
%%   \usepackage[T1]{fontenc}
%%   \usepackage{cmbright}
%%
\begingroup%
\makeatletter%
\begin{pgfpicture}%
\pgfpathrectangle{\pgfpointorigin}{\pgfqpoint{8.000000in}{4.000000in}}%
\pgfusepath{use as bounding box, clip}%
\begin{pgfscope}%
\pgfsetbuttcap%
\pgfsetmiterjoin%
\definecolor{currentfill}{rgb}{1.000000,1.000000,1.000000}%
\pgfsetfillcolor{currentfill}%
\pgfsetlinewidth{0.000000pt}%
\definecolor{currentstroke}{rgb}{1.000000,1.000000,1.000000}%
\pgfsetstrokecolor{currentstroke}%
\pgfsetdash{}{0pt}%
\pgfpathmoveto{\pgfqpoint{0.000000in}{0.000000in}}%
\pgfpathlineto{\pgfqpoint{8.000000in}{0.000000in}}%
\pgfpathlineto{\pgfqpoint{8.000000in}{4.000000in}}%
\pgfpathlineto{\pgfqpoint{0.000000in}{4.000000in}}%
\pgfpathclose%
\pgfusepath{fill}%
\end{pgfscope}%
\begin{pgfscope}%
\pgfsetbuttcap%
\pgfsetmiterjoin%
\definecolor{currentfill}{rgb}{1.000000,1.000000,1.000000}%
\pgfsetfillcolor{currentfill}%
\pgfsetlinewidth{0.000000pt}%
\definecolor{currentstroke}{rgb}{0.000000,0.000000,0.000000}%
\pgfsetstrokecolor{currentstroke}%
\pgfsetstrokeopacity{0.000000}%
\pgfsetdash{}{0pt}%
\pgfpathmoveto{\pgfqpoint{1.000000in}{0.440000in}}%
\pgfpathlineto{\pgfqpoint{7.200000in}{0.440000in}}%
\pgfpathlineto{\pgfqpoint{7.200000in}{3.520000in}}%
\pgfpathlineto{\pgfqpoint{1.000000in}{3.520000in}}%
\pgfpathclose%
\pgfusepath{fill}%
\end{pgfscope}%
\begin{pgfscope}%
\pgfsetbuttcap%
\pgfsetroundjoin%
\definecolor{currentfill}{rgb}{0.000000,0.000000,0.000000}%
\pgfsetfillcolor{currentfill}%
\pgfsetlinewidth{0.803000pt}%
\definecolor{currentstroke}{rgb}{0.000000,0.000000,0.000000}%
\pgfsetstrokecolor{currentstroke}%
\pgfsetdash{}{0pt}%
\pgfsys@defobject{currentmarker}{\pgfqpoint{0.000000in}{-0.048611in}}{\pgfqpoint{0.000000in}{0.000000in}}{%
\pgfpathmoveto{\pgfqpoint{0.000000in}{0.000000in}}%
\pgfpathlineto{\pgfqpoint{0.000000in}{-0.048611in}}%
\pgfusepath{stroke,fill}%
}%
\begin{pgfscope}%
\pgfsys@transformshift{1.281818in}{0.440000in}%
\pgfsys@useobject{currentmarker}{}%
\end{pgfscope}%
\end{pgfscope}%
\begin{pgfscope}%
\pgftext[x=1.281818in,y=0.342778in,,top]{\rmfamily\fontsize{10.000000}{12.000000}\selectfont \(\displaystyle 0.00\)}%
\end{pgfscope}%
\begin{pgfscope}%
\pgfsetbuttcap%
\pgfsetroundjoin%
\definecolor{currentfill}{rgb}{0.000000,0.000000,0.000000}%
\pgfsetfillcolor{currentfill}%
\pgfsetlinewidth{0.803000pt}%
\definecolor{currentstroke}{rgb}{0.000000,0.000000,0.000000}%
\pgfsetstrokecolor{currentstroke}%
\pgfsetdash{}{0pt}%
\pgfsys@defobject{currentmarker}{\pgfqpoint{0.000000in}{-0.048611in}}{\pgfqpoint{0.000000in}{0.000000in}}{%
\pgfpathmoveto{\pgfqpoint{0.000000in}{0.000000in}}%
\pgfpathlineto{\pgfqpoint{0.000000in}{-0.048611in}}%
\pgfusepath{stroke,fill}%
}%
\begin{pgfscope}%
\pgfsys@transformshift{2.043489in}{0.440000in}%
\pgfsys@useobject{currentmarker}{}%
\end{pgfscope}%
\end{pgfscope}%
\begin{pgfscope}%
\pgftext[x=2.043489in,y=0.342778in,,top]{\rmfamily\fontsize{10.000000}{12.000000}\selectfont \(\displaystyle 0.25\)}%
\end{pgfscope}%
\begin{pgfscope}%
\pgfsetbuttcap%
\pgfsetroundjoin%
\definecolor{currentfill}{rgb}{0.000000,0.000000,0.000000}%
\pgfsetfillcolor{currentfill}%
\pgfsetlinewidth{0.803000pt}%
\definecolor{currentstroke}{rgb}{0.000000,0.000000,0.000000}%
\pgfsetstrokecolor{currentstroke}%
\pgfsetdash{}{0pt}%
\pgfsys@defobject{currentmarker}{\pgfqpoint{0.000000in}{-0.048611in}}{\pgfqpoint{0.000000in}{0.000000in}}{%
\pgfpathmoveto{\pgfqpoint{0.000000in}{0.000000in}}%
\pgfpathlineto{\pgfqpoint{0.000000in}{-0.048611in}}%
\pgfusepath{stroke,fill}%
}%
\begin{pgfscope}%
\pgfsys@transformshift{2.805160in}{0.440000in}%
\pgfsys@useobject{currentmarker}{}%
\end{pgfscope}%
\end{pgfscope}%
\begin{pgfscope}%
\pgftext[x=2.805160in,y=0.342778in,,top]{\rmfamily\fontsize{10.000000}{12.000000}\selectfont \(\displaystyle 0.50\)}%
\end{pgfscope}%
\begin{pgfscope}%
\pgfsetbuttcap%
\pgfsetroundjoin%
\definecolor{currentfill}{rgb}{0.000000,0.000000,0.000000}%
\pgfsetfillcolor{currentfill}%
\pgfsetlinewidth{0.803000pt}%
\definecolor{currentstroke}{rgb}{0.000000,0.000000,0.000000}%
\pgfsetstrokecolor{currentstroke}%
\pgfsetdash{}{0pt}%
\pgfsys@defobject{currentmarker}{\pgfqpoint{0.000000in}{-0.048611in}}{\pgfqpoint{0.000000in}{0.000000in}}{%
\pgfpathmoveto{\pgfqpoint{0.000000in}{0.000000in}}%
\pgfpathlineto{\pgfqpoint{0.000000in}{-0.048611in}}%
\pgfusepath{stroke,fill}%
}%
\begin{pgfscope}%
\pgfsys@transformshift{3.566830in}{0.440000in}%
\pgfsys@useobject{currentmarker}{}%
\end{pgfscope}%
\end{pgfscope}%
\begin{pgfscope}%
\pgftext[x=3.566830in,y=0.342778in,,top]{\rmfamily\fontsize{10.000000}{12.000000}\selectfont \(\displaystyle 0.75\)}%
\end{pgfscope}%
\begin{pgfscope}%
\pgfsetbuttcap%
\pgfsetroundjoin%
\definecolor{currentfill}{rgb}{0.000000,0.000000,0.000000}%
\pgfsetfillcolor{currentfill}%
\pgfsetlinewidth{0.803000pt}%
\definecolor{currentstroke}{rgb}{0.000000,0.000000,0.000000}%
\pgfsetstrokecolor{currentstroke}%
\pgfsetdash{}{0pt}%
\pgfsys@defobject{currentmarker}{\pgfqpoint{0.000000in}{-0.048611in}}{\pgfqpoint{0.000000in}{0.000000in}}{%
\pgfpathmoveto{\pgfqpoint{0.000000in}{0.000000in}}%
\pgfpathlineto{\pgfqpoint{0.000000in}{-0.048611in}}%
\pgfusepath{stroke,fill}%
}%
\begin{pgfscope}%
\pgfsys@transformshift{4.328501in}{0.440000in}%
\pgfsys@useobject{currentmarker}{}%
\end{pgfscope}%
\end{pgfscope}%
\begin{pgfscope}%
\pgftext[x=4.328501in,y=0.342778in,,top]{\rmfamily\fontsize{10.000000}{12.000000}\selectfont \(\displaystyle 1.00\)}%
\end{pgfscope}%
\begin{pgfscope}%
\pgfsetbuttcap%
\pgfsetroundjoin%
\definecolor{currentfill}{rgb}{0.000000,0.000000,0.000000}%
\pgfsetfillcolor{currentfill}%
\pgfsetlinewidth{0.803000pt}%
\definecolor{currentstroke}{rgb}{0.000000,0.000000,0.000000}%
\pgfsetstrokecolor{currentstroke}%
\pgfsetdash{}{0pt}%
\pgfsys@defobject{currentmarker}{\pgfqpoint{0.000000in}{-0.048611in}}{\pgfqpoint{0.000000in}{0.000000in}}{%
\pgfpathmoveto{\pgfqpoint{0.000000in}{0.000000in}}%
\pgfpathlineto{\pgfqpoint{0.000000in}{-0.048611in}}%
\pgfusepath{stroke,fill}%
}%
\begin{pgfscope}%
\pgfsys@transformshift{5.090172in}{0.440000in}%
\pgfsys@useobject{currentmarker}{}%
\end{pgfscope}%
\end{pgfscope}%
\begin{pgfscope}%
\pgftext[x=5.090172in,y=0.342778in,,top]{\rmfamily\fontsize{10.000000}{12.000000}\selectfont \(\displaystyle 1.25\)}%
\end{pgfscope}%
\begin{pgfscope}%
\pgfsetbuttcap%
\pgfsetroundjoin%
\definecolor{currentfill}{rgb}{0.000000,0.000000,0.000000}%
\pgfsetfillcolor{currentfill}%
\pgfsetlinewidth{0.803000pt}%
\definecolor{currentstroke}{rgb}{0.000000,0.000000,0.000000}%
\pgfsetstrokecolor{currentstroke}%
\pgfsetdash{}{0pt}%
\pgfsys@defobject{currentmarker}{\pgfqpoint{0.000000in}{-0.048611in}}{\pgfqpoint{0.000000in}{0.000000in}}{%
\pgfpathmoveto{\pgfqpoint{0.000000in}{0.000000in}}%
\pgfpathlineto{\pgfqpoint{0.000000in}{-0.048611in}}%
\pgfusepath{stroke,fill}%
}%
\begin{pgfscope}%
\pgfsys@transformshift{5.851843in}{0.440000in}%
\pgfsys@useobject{currentmarker}{}%
\end{pgfscope}%
\end{pgfscope}%
\begin{pgfscope}%
\pgftext[x=5.851843in,y=0.342778in,,top]{\rmfamily\fontsize{10.000000}{12.000000}\selectfont \(\displaystyle 1.50\)}%
\end{pgfscope}%
\begin{pgfscope}%
\pgfsetbuttcap%
\pgfsetroundjoin%
\definecolor{currentfill}{rgb}{0.000000,0.000000,0.000000}%
\pgfsetfillcolor{currentfill}%
\pgfsetlinewidth{0.803000pt}%
\definecolor{currentstroke}{rgb}{0.000000,0.000000,0.000000}%
\pgfsetstrokecolor{currentstroke}%
\pgfsetdash{}{0pt}%
\pgfsys@defobject{currentmarker}{\pgfqpoint{0.000000in}{-0.048611in}}{\pgfqpoint{0.000000in}{0.000000in}}{%
\pgfpathmoveto{\pgfqpoint{0.000000in}{0.000000in}}%
\pgfpathlineto{\pgfqpoint{0.000000in}{-0.048611in}}%
\pgfusepath{stroke,fill}%
}%
\begin{pgfscope}%
\pgfsys@transformshift{6.613514in}{0.440000in}%
\pgfsys@useobject{currentmarker}{}%
\end{pgfscope}%
\end{pgfscope}%
\begin{pgfscope}%
\pgftext[x=6.613514in,y=0.342778in,,top]{\rmfamily\fontsize{10.000000}{12.000000}\selectfont \(\displaystyle 1.75\)}%
\end{pgfscope}%
\begin{pgfscope}%
\pgftext[x=4.100000in,y=0.164567in,,top]{\rmfamily\fontsize{10.000000}{12.000000}\selectfont Time (s)}%
\end{pgfscope}%
\begin{pgfscope}%
\pgfsetbuttcap%
\pgfsetroundjoin%
\definecolor{currentfill}{rgb}{0.000000,0.000000,0.000000}%
\pgfsetfillcolor{currentfill}%
\pgfsetlinewidth{0.803000pt}%
\definecolor{currentstroke}{rgb}{0.000000,0.000000,0.000000}%
\pgfsetstrokecolor{currentstroke}%
\pgfsetdash{}{0pt}%
\pgfsys@defobject{currentmarker}{\pgfqpoint{-0.048611in}{0.000000in}}{\pgfqpoint{0.000000in}{0.000000in}}{%
\pgfpathmoveto{\pgfqpoint{0.000000in}{0.000000in}}%
\pgfpathlineto{\pgfqpoint{-0.048611in}{0.000000in}}%
\pgfusepath{stroke,fill}%
}%
\begin{pgfscope}%
\pgfsys@transformshift{1.000000in}{0.747896in}%
\pgfsys@useobject{currentmarker}{}%
\end{pgfscope}%
\end{pgfscope}%
\begin{pgfscope}%
\pgftext[x=0.684030in,y=0.700068in,left,base]{\rmfamily\fontsize{10.000000}{12.000000}\selectfont \(\displaystyle 140\)}%
\end{pgfscope}%
\begin{pgfscope}%
\pgfsetbuttcap%
\pgfsetroundjoin%
\definecolor{currentfill}{rgb}{0.000000,0.000000,0.000000}%
\pgfsetfillcolor{currentfill}%
\pgfsetlinewidth{0.803000pt}%
\definecolor{currentstroke}{rgb}{0.000000,0.000000,0.000000}%
\pgfsetstrokecolor{currentstroke}%
\pgfsetdash{}{0pt}%
\pgfsys@defobject{currentmarker}{\pgfqpoint{-0.048611in}{0.000000in}}{\pgfqpoint{0.000000in}{0.000000in}}{%
\pgfpathmoveto{\pgfqpoint{0.000000in}{0.000000in}}%
\pgfpathlineto{\pgfqpoint{-0.048611in}{0.000000in}}%
\pgfusepath{stroke,fill}%
}%
\begin{pgfscope}%
\pgfsys@transformshift{1.000000in}{1.175589in}%
\pgfsys@useobject{currentmarker}{}%
\end{pgfscope}%
\end{pgfscope}%
\begin{pgfscope}%
\pgftext[x=0.684030in,y=1.127761in,left,base]{\rmfamily\fontsize{10.000000}{12.000000}\selectfont \(\displaystyle 160\)}%
\end{pgfscope}%
\begin{pgfscope}%
\pgfsetbuttcap%
\pgfsetroundjoin%
\definecolor{currentfill}{rgb}{0.000000,0.000000,0.000000}%
\pgfsetfillcolor{currentfill}%
\pgfsetlinewidth{0.803000pt}%
\definecolor{currentstroke}{rgb}{0.000000,0.000000,0.000000}%
\pgfsetstrokecolor{currentstroke}%
\pgfsetdash{}{0pt}%
\pgfsys@defobject{currentmarker}{\pgfqpoint{-0.048611in}{0.000000in}}{\pgfqpoint{0.000000in}{0.000000in}}{%
\pgfpathmoveto{\pgfqpoint{0.000000in}{0.000000in}}%
\pgfpathlineto{\pgfqpoint{-0.048611in}{0.000000in}}%
\pgfusepath{stroke,fill}%
}%
\begin{pgfscope}%
\pgfsys@transformshift{1.000000in}{1.603282in}%
\pgfsys@useobject{currentmarker}{}%
\end{pgfscope}%
\end{pgfscope}%
\begin{pgfscope}%
\pgftext[x=0.684030in,y=1.555454in,left,base]{\rmfamily\fontsize{10.000000}{12.000000}\selectfont \(\displaystyle 180\)}%
\end{pgfscope}%
\begin{pgfscope}%
\pgfsetbuttcap%
\pgfsetroundjoin%
\definecolor{currentfill}{rgb}{0.000000,0.000000,0.000000}%
\pgfsetfillcolor{currentfill}%
\pgfsetlinewidth{0.803000pt}%
\definecolor{currentstroke}{rgb}{0.000000,0.000000,0.000000}%
\pgfsetstrokecolor{currentstroke}%
\pgfsetdash{}{0pt}%
\pgfsys@defobject{currentmarker}{\pgfqpoint{-0.048611in}{0.000000in}}{\pgfqpoint{0.000000in}{0.000000in}}{%
\pgfpathmoveto{\pgfqpoint{0.000000in}{0.000000in}}%
\pgfpathlineto{\pgfqpoint{-0.048611in}{0.000000in}}%
\pgfusepath{stroke,fill}%
}%
\begin{pgfscope}%
\pgfsys@transformshift{1.000000in}{2.030975in}%
\pgfsys@useobject{currentmarker}{}%
\end{pgfscope}%
\end{pgfscope}%
\begin{pgfscope}%
\pgftext[x=0.684030in,y=1.983147in,left,base]{\rmfamily\fontsize{10.000000}{12.000000}\selectfont \(\displaystyle 200\)}%
\end{pgfscope}%
\begin{pgfscope}%
\pgfsetbuttcap%
\pgfsetroundjoin%
\definecolor{currentfill}{rgb}{0.000000,0.000000,0.000000}%
\pgfsetfillcolor{currentfill}%
\pgfsetlinewidth{0.803000pt}%
\definecolor{currentstroke}{rgb}{0.000000,0.000000,0.000000}%
\pgfsetstrokecolor{currentstroke}%
\pgfsetdash{}{0pt}%
\pgfsys@defobject{currentmarker}{\pgfqpoint{-0.048611in}{0.000000in}}{\pgfqpoint{0.000000in}{0.000000in}}{%
\pgfpathmoveto{\pgfqpoint{0.000000in}{0.000000in}}%
\pgfpathlineto{\pgfqpoint{-0.048611in}{0.000000in}}%
\pgfusepath{stroke,fill}%
}%
\begin{pgfscope}%
\pgfsys@transformshift{1.000000in}{2.458668in}%
\pgfsys@useobject{currentmarker}{}%
\end{pgfscope}%
\end{pgfscope}%
\begin{pgfscope}%
\pgftext[x=0.684030in,y=2.410840in,left,base]{\rmfamily\fontsize{10.000000}{12.000000}\selectfont \(\displaystyle 220\)}%
\end{pgfscope}%
\begin{pgfscope}%
\pgfsetbuttcap%
\pgfsetroundjoin%
\definecolor{currentfill}{rgb}{0.000000,0.000000,0.000000}%
\pgfsetfillcolor{currentfill}%
\pgfsetlinewidth{0.803000pt}%
\definecolor{currentstroke}{rgb}{0.000000,0.000000,0.000000}%
\pgfsetstrokecolor{currentstroke}%
\pgfsetdash{}{0pt}%
\pgfsys@defobject{currentmarker}{\pgfqpoint{-0.048611in}{0.000000in}}{\pgfqpoint{0.000000in}{0.000000in}}{%
\pgfpathmoveto{\pgfqpoint{0.000000in}{0.000000in}}%
\pgfpathlineto{\pgfqpoint{-0.048611in}{0.000000in}}%
\pgfusepath{stroke,fill}%
}%
\begin{pgfscope}%
\pgfsys@transformshift{1.000000in}{2.886361in}%
\pgfsys@useobject{currentmarker}{}%
\end{pgfscope}%
\end{pgfscope}%
\begin{pgfscope}%
\pgftext[x=0.684030in,y=2.838533in,left,base]{\rmfamily\fontsize{10.000000}{12.000000}\selectfont \(\displaystyle 240\)}%
\end{pgfscope}%
\begin{pgfscope}%
\pgfsetbuttcap%
\pgfsetroundjoin%
\definecolor{currentfill}{rgb}{0.000000,0.000000,0.000000}%
\pgfsetfillcolor{currentfill}%
\pgfsetlinewidth{0.803000pt}%
\definecolor{currentstroke}{rgb}{0.000000,0.000000,0.000000}%
\pgfsetstrokecolor{currentstroke}%
\pgfsetdash{}{0pt}%
\pgfsys@defobject{currentmarker}{\pgfqpoint{-0.048611in}{0.000000in}}{\pgfqpoint{0.000000in}{0.000000in}}{%
\pgfpathmoveto{\pgfqpoint{0.000000in}{0.000000in}}%
\pgfpathlineto{\pgfqpoint{-0.048611in}{0.000000in}}%
\pgfusepath{stroke,fill}%
}%
\begin{pgfscope}%
\pgfsys@transformshift{1.000000in}{3.314054in}%
\pgfsys@useobject{currentmarker}{}%
\end{pgfscope}%
\end{pgfscope}%
\begin{pgfscope}%
\pgftext[x=0.684030in,y=3.266226in,left,base]{\rmfamily\fontsize{10.000000}{12.000000}\selectfont \(\displaystyle 260\)}%
\end{pgfscope}%
\begin{pgfscope}%
\pgftext[x=0.628474in,y=1.980000in,,bottom,rotate=90.000000]{\rmfamily\fontsize{10.000000}{12.000000}\selectfont Frequency (Hz)}%
\end{pgfscope}%
\begin{pgfscope}%
\pgfpathrectangle{\pgfqpoint{1.000000in}{0.440000in}}{\pgfqpoint{6.200000in}{3.080000in}} %
\pgfusepath{clip}%
\pgfsetrectcap%
\pgfsetroundjoin%
\pgfsetlinewidth{0.501875pt}%
\definecolor{currentstroke}{rgb}{0.800000,0.000000,0.266667}%
\pgfsetstrokecolor{currentstroke}%
\pgfsetdash{}{0pt}%
\pgfpathmoveto{\pgfqpoint{1.769287in}{0.440000in}}%
\pgfpathlineto{\pgfqpoint{1.769287in}{3.520000in}}%
\pgfusepath{stroke}%
\end{pgfscope}%
\begin{pgfscope}%
\pgfpathrectangle{\pgfqpoint{1.000000in}{0.440000in}}{\pgfqpoint{6.200000in}{3.080000in}} %
\pgfusepath{clip}%
\pgfsetrectcap%
\pgfsetroundjoin%
\pgfsetlinewidth{0.501875pt}%
\definecolor{currentstroke}{rgb}{0.800000,0.000000,0.266667}%
\pgfsetstrokecolor{currentstroke}%
\pgfsetdash{}{0pt}%
\pgfpathmoveto{\pgfqpoint{2.165356in}{0.440000in}}%
\pgfpathlineto{\pgfqpoint{2.165356in}{3.520000in}}%
\pgfusepath{stroke}%
\end{pgfscope}%
\begin{pgfscope}%
\pgfpathrectangle{\pgfqpoint{1.000000in}{0.440000in}}{\pgfqpoint{6.200000in}{3.080000in}} %
\pgfusepath{clip}%
\pgfsetrectcap%
\pgfsetroundjoin%
\pgfsetlinewidth{0.501875pt}%
\definecolor{currentstroke}{rgb}{0.501961,0.501961,0.501961}%
\pgfsetstrokecolor{currentstroke}%
\pgfsetdash{}{0pt}%
\pgfpathmoveto{\pgfqpoint{1.967322in}{0.440000in}}%
\pgfpathlineto{\pgfqpoint{1.967322in}{3.520000in}}%
\pgfusepath{stroke}%
\end{pgfscope}%
\begin{pgfscope}%
\pgfpathrectangle{\pgfqpoint{1.000000in}{0.440000in}}{\pgfqpoint{6.200000in}{3.080000in}} %
\pgfusepath{clip}%
\pgfsetrectcap%
\pgfsetroundjoin%
\pgfsetlinewidth{0.501875pt}%
\definecolor{currentstroke}{rgb}{0.800000,0.000000,0.266667}%
\pgfsetstrokecolor{currentstroke}%
\pgfsetdash{}{0pt}%
\pgfpathmoveto{\pgfqpoint{2.165356in}{0.440000in}}%
\pgfpathlineto{\pgfqpoint{2.165356in}{3.520000in}}%
\pgfusepath{stroke}%
\end{pgfscope}%
\begin{pgfscope}%
\pgfpathrectangle{\pgfqpoint{1.000000in}{0.440000in}}{\pgfqpoint{6.200000in}{3.080000in}} %
\pgfusepath{clip}%
\pgfsetrectcap%
\pgfsetroundjoin%
\pgfsetlinewidth{0.501875pt}%
\definecolor{currentstroke}{rgb}{0.800000,0.000000,0.266667}%
\pgfsetstrokecolor{currentstroke}%
\pgfsetdash{}{0pt}%
\pgfpathmoveto{\pgfqpoint{2.622359in}{0.440000in}}%
\pgfpathlineto{\pgfqpoint{2.622359in}{3.520000in}}%
\pgfusepath{stroke}%
\end{pgfscope}%
\begin{pgfscope}%
\pgfpathrectangle{\pgfqpoint{1.000000in}{0.440000in}}{\pgfqpoint{6.200000in}{3.080000in}} %
\pgfusepath{clip}%
\pgfsetrectcap%
\pgfsetroundjoin%
\pgfsetlinewidth{0.501875pt}%
\definecolor{currentstroke}{rgb}{0.501961,0.501961,0.501961}%
\pgfsetstrokecolor{currentstroke}%
\pgfsetdash{}{0pt}%
\pgfpathmoveto{\pgfqpoint{2.317690in}{0.440000in}}%
\pgfpathlineto{\pgfqpoint{2.317690in}{3.520000in}}%
\pgfusepath{stroke}%
\end{pgfscope}%
\begin{pgfscope}%
\pgfpathrectangle{\pgfqpoint{1.000000in}{0.440000in}}{\pgfqpoint{6.200000in}{3.080000in}} %
\pgfusepath{clip}%
\pgfsetrectcap%
\pgfsetroundjoin%
\pgfsetlinewidth{0.501875pt}%
\definecolor{currentstroke}{rgb}{0.501961,0.501961,0.501961}%
\pgfsetstrokecolor{currentstroke}%
\pgfsetdash{}{0pt}%
\pgfpathmoveto{\pgfqpoint{2.470025in}{0.440000in}}%
\pgfpathlineto{\pgfqpoint{2.470025in}{3.520000in}}%
\pgfusepath{stroke}%
\end{pgfscope}%
\begin{pgfscope}%
\pgfpathrectangle{\pgfqpoint{1.000000in}{0.440000in}}{\pgfqpoint{6.200000in}{3.080000in}} %
\pgfusepath{clip}%
\pgfsetrectcap%
\pgfsetroundjoin%
\pgfsetlinewidth{0.501875pt}%
\definecolor{currentstroke}{rgb}{0.800000,0.000000,0.266667}%
\pgfsetstrokecolor{currentstroke}%
\pgfsetdash{}{0pt}%
\pgfpathmoveto{\pgfqpoint{2.622359in}{0.440000in}}%
\pgfpathlineto{\pgfqpoint{2.622359in}{3.520000in}}%
\pgfusepath{stroke}%
\end{pgfscope}%
\begin{pgfscope}%
\pgfpathrectangle{\pgfqpoint{1.000000in}{0.440000in}}{\pgfqpoint{6.200000in}{3.080000in}} %
\pgfusepath{clip}%
\pgfsetrectcap%
\pgfsetroundjoin%
\pgfsetlinewidth{0.501875pt}%
\definecolor{currentstroke}{rgb}{0.800000,0.000000,0.266667}%
\pgfsetstrokecolor{currentstroke}%
\pgfsetdash{}{0pt}%
\pgfpathmoveto{\pgfqpoint{3.444963in}{0.440000in}}%
\pgfpathlineto{\pgfqpoint{3.444963in}{3.520000in}}%
\pgfusepath{stroke}%
\end{pgfscope}%
\begin{pgfscope}%
\pgfpathrectangle{\pgfqpoint{1.000000in}{0.440000in}}{\pgfqpoint{6.200000in}{3.080000in}} %
\pgfusepath{clip}%
\pgfsetrectcap%
\pgfsetroundjoin%
\pgfsetlinewidth{0.501875pt}%
\definecolor{currentstroke}{rgb}{0.501961,0.501961,0.501961}%
\pgfsetstrokecolor{currentstroke}%
\pgfsetdash{}{0pt}%
\pgfpathmoveto{\pgfqpoint{2.786880in}{0.440000in}}%
\pgfpathlineto{\pgfqpoint{2.786880in}{3.520000in}}%
\pgfusepath{stroke}%
\end{pgfscope}%
\begin{pgfscope}%
\pgfpathrectangle{\pgfqpoint{1.000000in}{0.440000in}}{\pgfqpoint{6.200000in}{3.080000in}} %
\pgfusepath{clip}%
\pgfsetrectcap%
\pgfsetroundjoin%
\pgfsetlinewidth{0.501875pt}%
\definecolor{currentstroke}{rgb}{0.501961,0.501961,0.501961}%
\pgfsetstrokecolor{currentstroke}%
\pgfsetdash{}{0pt}%
\pgfpathmoveto{\pgfqpoint{2.951400in}{0.440000in}}%
\pgfpathlineto{\pgfqpoint{2.951400in}{3.520000in}}%
\pgfusepath{stroke}%
\end{pgfscope}%
\begin{pgfscope}%
\pgfpathrectangle{\pgfqpoint{1.000000in}{0.440000in}}{\pgfqpoint{6.200000in}{3.080000in}} %
\pgfusepath{clip}%
\pgfsetrectcap%
\pgfsetroundjoin%
\pgfsetlinewidth{0.501875pt}%
\definecolor{currentstroke}{rgb}{0.501961,0.501961,0.501961}%
\pgfsetstrokecolor{currentstroke}%
\pgfsetdash{}{0pt}%
\pgfpathmoveto{\pgfqpoint{3.115921in}{0.440000in}}%
\pgfpathlineto{\pgfqpoint{3.115921in}{3.520000in}}%
\pgfusepath{stroke}%
\end{pgfscope}%
\begin{pgfscope}%
\pgfpathrectangle{\pgfqpoint{1.000000in}{0.440000in}}{\pgfqpoint{6.200000in}{3.080000in}} %
\pgfusepath{clip}%
\pgfsetrectcap%
\pgfsetroundjoin%
\pgfsetlinewidth{0.501875pt}%
\definecolor{currentstroke}{rgb}{0.501961,0.501961,0.501961}%
\pgfsetstrokecolor{currentstroke}%
\pgfsetdash{}{0pt}%
\pgfpathmoveto{\pgfqpoint{3.280442in}{0.440000in}}%
\pgfpathlineto{\pgfqpoint{3.280442in}{3.520000in}}%
\pgfusepath{stroke}%
\end{pgfscope}%
\begin{pgfscope}%
\pgfpathrectangle{\pgfqpoint{1.000000in}{0.440000in}}{\pgfqpoint{6.200000in}{3.080000in}} %
\pgfusepath{clip}%
\pgfsetrectcap%
\pgfsetroundjoin%
\pgfsetlinewidth{0.501875pt}%
\definecolor{currentstroke}{rgb}{0.800000,0.000000,0.266667}%
\pgfsetstrokecolor{currentstroke}%
\pgfsetdash{}{0pt}%
\pgfpathmoveto{\pgfqpoint{3.444963in}{0.440000in}}%
\pgfpathlineto{\pgfqpoint{3.444963in}{3.520000in}}%
\pgfusepath{stroke}%
\end{pgfscope}%
\begin{pgfscope}%
\pgfpathrectangle{\pgfqpoint{1.000000in}{0.440000in}}{\pgfqpoint{6.200000in}{3.080000in}} %
\pgfusepath{clip}%
\pgfsetrectcap%
\pgfsetroundjoin%
\pgfsetlinewidth{0.501875pt}%
\definecolor{currentstroke}{rgb}{0.800000,0.000000,0.266667}%
\pgfsetstrokecolor{currentstroke}%
\pgfsetdash{}{0pt}%
\pgfpathmoveto{\pgfqpoint{3.901966in}{0.440000in}}%
\pgfpathlineto{\pgfqpoint{3.901966in}{3.520000in}}%
\pgfusepath{stroke}%
\end{pgfscope}%
\begin{pgfscope}%
\pgfpathrectangle{\pgfqpoint{1.000000in}{0.440000in}}{\pgfqpoint{6.200000in}{3.080000in}} %
\pgfusepath{clip}%
\pgfsetrectcap%
\pgfsetroundjoin%
\pgfsetlinewidth{0.501875pt}%
\definecolor{currentstroke}{rgb}{0.501961,0.501961,0.501961}%
\pgfsetstrokecolor{currentstroke}%
\pgfsetdash{}{0pt}%
\pgfpathmoveto{\pgfqpoint{3.597297in}{0.440000in}}%
\pgfpathlineto{\pgfqpoint{3.597297in}{3.520000in}}%
\pgfusepath{stroke}%
\end{pgfscope}%
\begin{pgfscope}%
\pgfpathrectangle{\pgfqpoint{1.000000in}{0.440000in}}{\pgfqpoint{6.200000in}{3.080000in}} %
\pgfusepath{clip}%
\pgfsetrectcap%
\pgfsetroundjoin%
\pgfsetlinewidth{0.501875pt}%
\definecolor{currentstroke}{rgb}{0.501961,0.501961,0.501961}%
\pgfsetstrokecolor{currentstroke}%
\pgfsetdash{}{0pt}%
\pgfpathmoveto{\pgfqpoint{3.749631in}{0.440000in}}%
\pgfpathlineto{\pgfqpoint{3.749631in}{3.520000in}}%
\pgfusepath{stroke}%
\end{pgfscope}%
\begin{pgfscope}%
\pgfpathrectangle{\pgfqpoint{1.000000in}{0.440000in}}{\pgfqpoint{6.200000in}{3.080000in}} %
\pgfusepath{clip}%
\pgfsetrectcap%
\pgfsetroundjoin%
\pgfsetlinewidth{0.501875pt}%
\definecolor{currentstroke}{rgb}{0.800000,0.000000,0.266667}%
\pgfsetstrokecolor{currentstroke}%
\pgfsetdash{}{0pt}%
\pgfpathmoveto{\pgfqpoint{3.901966in}{0.440000in}}%
\pgfpathlineto{\pgfqpoint{3.901966in}{3.520000in}}%
\pgfusepath{stroke}%
\end{pgfscope}%
\begin{pgfscope}%
\pgfpathrectangle{\pgfqpoint{1.000000in}{0.440000in}}{\pgfqpoint{6.200000in}{3.080000in}} %
\pgfusepath{clip}%
\pgfsetrectcap%
\pgfsetroundjoin%
\pgfsetlinewidth{0.501875pt}%
\definecolor{currentstroke}{rgb}{0.800000,0.000000,0.266667}%
\pgfsetstrokecolor{currentstroke}%
\pgfsetdash{}{0pt}%
\pgfpathmoveto{\pgfqpoint{5.059705in}{0.440000in}}%
\pgfpathlineto{\pgfqpoint{5.059705in}{3.520000in}}%
\pgfusepath{stroke}%
\end{pgfscope}%
\begin{pgfscope}%
\pgfpathrectangle{\pgfqpoint{1.000000in}{0.440000in}}{\pgfqpoint{6.200000in}{3.080000in}} %
\pgfusepath{clip}%
\pgfsetrectcap%
\pgfsetroundjoin%
\pgfsetlinewidth{0.501875pt}%
\definecolor{currentstroke}{rgb}{0.501961,0.501961,0.501961}%
\pgfsetstrokecolor{currentstroke}%
\pgfsetdash{}{0pt}%
\pgfpathmoveto{\pgfqpoint{4.067357in}{0.440000in}}%
\pgfpathlineto{\pgfqpoint{4.067357in}{3.520000in}}%
\pgfusepath{stroke}%
\end{pgfscope}%
\begin{pgfscope}%
\pgfpathrectangle{\pgfqpoint{1.000000in}{0.440000in}}{\pgfqpoint{6.200000in}{3.080000in}} %
\pgfusepath{clip}%
\pgfsetrectcap%
\pgfsetroundjoin%
\pgfsetlinewidth{0.501875pt}%
\definecolor{currentstroke}{rgb}{0.501961,0.501961,0.501961}%
\pgfsetstrokecolor{currentstroke}%
\pgfsetdash{}{0pt}%
\pgfpathmoveto{\pgfqpoint{4.232748in}{0.440000in}}%
\pgfpathlineto{\pgfqpoint{4.232748in}{3.520000in}}%
\pgfusepath{stroke}%
\end{pgfscope}%
\begin{pgfscope}%
\pgfpathrectangle{\pgfqpoint{1.000000in}{0.440000in}}{\pgfqpoint{6.200000in}{3.080000in}} %
\pgfusepath{clip}%
\pgfsetrectcap%
\pgfsetroundjoin%
\pgfsetlinewidth{0.501875pt}%
\definecolor{currentstroke}{rgb}{0.501961,0.501961,0.501961}%
\pgfsetstrokecolor{currentstroke}%
\pgfsetdash{}{0pt}%
\pgfpathmoveto{\pgfqpoint{4.398140in}{0.440000in}}%
\pgfpathlineto{\pgfqpoint{4.398140in}{3.520000in}}%
\pgfusepath{stroke}%
\end{pgfscope}%
\begin{pgfscope}%
\pgfpathrectangle{\pgfqpoint{1.000000in}{0.440000in}}{\pgfqpoint{6.200000in}{3.080000in}} %
\pgfusepath{clip}%
\pgfsetrectcap%
\pgfsetroundjoin%
\pgfsetlinewidth{0.501875pt}%
\definecolor{currentstroke}{rgb}{0.501961,0.501961,0.501961}%
\pgfsetstrokecolor{currentstroke}%
\pgfsetdash{}{0pt}%
\pgfpathmoveto{\pgfqpoint{4.563531in}{0.440000in}}%
\pgfpathlineto{\pgfqpoint{4.563531in}{3.520000in}}%
\pgfusepath{stroke}%
\end{pgfscope}%
\begin{pgfscope}%
\pgfpathrectangle{\pgfqpoint{1.000000in}{0.440000in}}{\pgfqpoint{6.200000in}{3.080000in}} %
\pgfusepath{clip}%
\pgfsetrectcap%
\pgfsetroundjoin%
\pgfsetlinewidth{0.501875pt}%
\definecolor{currentstroke}{rgb}{0.501961,0.501961,0.501961}%
\pgfsetstrokecolor{currentstroke}%
\pgfsetdash{}{0pt}%
\pgfpathmoveto{\pgfqpoint{4.728922in}{0.440000in}}%
\pgfpathlineto{\pgfqpoint{4.728922in}{3.520000in}}%
\pgfusepath{stroke}%
\end{pgfscope}%
\begin{pgfscope}%
\pgfpathrectangle{\pgfqpoint{1.000000in}{0.440000in}}{\pgfqpoint{6.200000in}{3.080000in}} %
\pgfusepath{clip}%
\pgfsetrectcap%
\pgfsetroundjoin%
\pgfsetlinewidth{0.501875pt}%
\definecolor{currentstroke}{rgb}{0.501961,0.501961,0.501961}%
\pgfsetstrokecolor{currentstroke}%
\pgfsetdash{}{0pt}%
\pgfpathmoveto{\pgfqpoint{4.894314in}{0.440000in}}%
\pgfpathlineto{\pgfqpoint{4.894314in}{3.520000in}}%
\pgfusepath{stroke}%
\end{pgfscope}%
\begin{pgfscope}%
\pgfpathrectangle{\pgfqpoint{1.000000in}{0.440000in}}{\pgfqpoint{6.200000in}{3.080000in}} %
\pgfusepath{clip}%
\pgfsetrectcap%
\pgfsetroundjoin%
\pgfsetlinewidth{0.501875pt}%
\definecolor{currentstroke}{rgb}{0.800000,0.000000,0.266667}%
\pgfsetstrokecolor{currentstroke}%
\pgfsetdash{}{0pt}%
\pgfpathmoveto{\pgfqpoint{5.059705in}{0.440000in}}%
\pgfpathlineto{\pgfqpoint{5.059705in}{3.520000in}}%
\pgfusepath{stroke}%
\end{pgfscope}%
\begin{pgfscope}%
\pgfpathrectangle{\pgfqpoint{1.000000in}{0.440000in}}{\pgfqpoint{6.200000in}{3.080000in}} %
\pgfusepath{clip}%
\pgfsetrectcap%
\pgfsetroundjoin%
\pgfsetlinewidth{0.501875pt}%
\definecolor{currentstroke}{rgb}{0.800000,0.000000,0.266667}%
\pgfsetstrokecolor{currentstroke}%
\pgfsetdash{}{0pt}%
\pgfpathmoveto{\pgfqpoint{6.765848in}{0.440000in}}%
\pgfpathlineto{\pgfqpoint{6.765848in}{3.520000in}}%
\pgfusepath{stroke}%
\end{pgfscope}%
\begin{pgfscope}%
\pgfpathrectangle{\pgfqpoint{1.000000in}{0.440000in}}{\pgfqpoint{6.200000in}{3.080000in}} %
\pgfusepath{clip}%
\pgfsetrectcap%
\pgfsetroundjoin%
\pgfsetlinewidth{0.501875pt}%
\definecolor{currentstroke}{rgb}{0.501961,0.501961,0.501961}%
\pgfsetstrokecolor{currentstroke}%
\pgfsetdash{}{0pt}%
\pgfpathmoveto{\pgfqpoint{5.214809in}{0.440000in}}%
\pgfpathlineto{\pgfqpoint{5.214809in}{3.520000in}}%
\pgfusepath{stroke}%
\end{pgfscope}%
\begin{pgfscope}%
\pgfpathrectangle{\pgfqpoint{1.000000in}{0.440000in}}{\pgfqpoint{6.200000in}{3.080000in}} %
\pgfusepath{clip}%
\pgfsetrectcap%
\pgfsetroundjoin%
\pgfsetlinewidth{0.501875pt}%
\definecolor{currentstroke}{rgb}{0.501961,0.501961,0.501961}%
\pgfsetstrokecolor{currentstroke}%
\pgfsetdash{}{0pt}%
\pgfpathmoveto{\pgfqpoint{5.369913in}{0.440000in}}%
\pgfpathlineto{\pgfqpoint{5.369913in}{3.520000in}}%
\pgfusepath{stroke}%
\end{pgfscope}%
\begin{pgfscope}%
\pgfpathrectangle{\pgfqpoint{1.000000in}{0.440000in}}{\pgfqpoint{6.200000in}{3.080000in}} %
\pgfusepath{clip}%
\pgfsetrectcap%
\pgfsetroundjoin%
\pgfsetlinewidth{0.501875pt}%
\definecolor{currentstroke}{rgb}{0.501961,0.501961,0.501961}%
\pgfsetstrokecolor{currentstroke}%
\pgfsetdash{}{0pt}%
\pgfpathmoveto{\pgfqpoint{5.525017in}{0.440000in}}%
\pgfpathlineto{\pgfqpoint{5.525017in}{3.520000in}}%
\pgfusepath{stroke}%
\end{pgfscope}%
\begin{pgfscope}%
\pgfpathrectangle{\pgfqpoint{1.000000in}{0.440000in}}{\pgfqpoint{6.200000in}{3.080000in}} %
\pgfusepath{clip}%
\pgfsetrectcap%
\pgfsetroundjoin%
\pgfsetlinewidth{0.501875pt}%
\definecolor{currentstroke}{rgb}{0.501961,0.501961,0.501961}%
\pgfsetstrokecolor{currentstroke}%
\pgfsetdash{}{0pt}%
\pgfpathmoveto{\pgfqpoint{5.680121in}{0.440000in}}%
\pgfpathlineto{\pgfqpoint{5.680121in}{3.520000in}}%
\pgfusepath{stroke}%
\end{pgfscope}%
\begin{pgfscope}%
\pgfpathrectangle{\pgfqpoint{1.000000in}{0.440000in}}{\pgfqpoint{6.200000in}{3.080000in}} %
\pgfusepath{clip}%
\pgfsetrectcap%
\pgfsetroundjoin%
\pgfsetlinewidth{0.501875pt}%
\definecolor{currentstroke}{rgb}{0.501961,0.501961,0.501961}%
\pgfsetstrokecolor{currentstroke}%
\pgfsetdash{}{0pt}%
\pgfpathmoveto{\pgfqpoint{5.835224in}{0.440000in}}%
\pgfpathlineto{\pgfqpoint{5.835224in}{3.520000in}}%
\pgfusepath{stroke}%
\end{pgfscope}%
\begin{pgfscope}%
\pgfpathrectangle{\pgfqpoint{1.000000in}{0.440000in}}{\pgfqpoint{6.200000in}{3.080000in}} %
\pgfusepath{clip}%
\pgfsetrectcap%
\pgfsetroundjoin%
\pgfsetlinewidth{0.501875pt}%
\definecolor{currentstroke}{rgb}{0.501961,0.501961,0.501961}%
\pgfsetstrokecolor{currentstroke}%
\pgfsetdash{}{0pt}%
\pgfpathmoveto{\pgfqpoint{5.990328in}{0.440000in}}%
\pgfpathlineto{\pgfqpoint{5.990328in}{3.520000in}}%
\pgfusepath{stroke}%
\end{pgfscope}%
\begin{pgfscope}%
\pgfpathrectangle{\pgfqpoint{1.000000in}{0.440000in}}{\pgfqpoint{6.200000in}{3.080000in}} %
\pgfusepath{clip}%
\pgfsetrectcap%
\pgfsetroundjoin%
\pgfsetlinewidth{0.501875pt}%
\definecolor{currentstroke}{rgb}{0.501961,0.501961,0.501961}%
\pgfsetstrokecolor{currentstroke}%
\pgfsetdash{}{0pt}%
\pgfpathmoveto{\pgfqpoint{6.145432in}{0.440000in}}%
\pgfpathlineto{\pgfqpoint{6.145432in}{3.520000in}}%
\pgfusepath{stroke}%
\end{pgfscope}%
\begin{pgfscope}%
\pgfpathrectangle{\pgfqpoint{1.000000in}{0.440000in}}{\pgfqpoint{6.200000in}{3.080000in}} %
\pgfusepath{clip}%
\pgfsetrectcap%
\pgfsetroundjoin%
\pgfsetlinewidth{0.501875pt}%
\definecolor{currentstroke}{rgb}{0.501961,0.501961,0.501961}%
\pgfsetstrokecolor{currentstroke}%
\pgfsetdash{}{0pt}%
\pgfpathmoveto{\pgfqpoint{6.300536in}{0.440000in}}%
\pgfpathlineto{\pgfqpoint{6.300536in}{3.520000in}}%
\pgfusepath{stroke}%
\end{pgfscope}%
\begin{pgfscope}%
\pgfpathrectangle{\pgfqpoint{1.000000in}{0.440000in}}{\pgfqpoint{6.200000in}{3.080000in}} %
\pgfusepath{clip}%
\pgfsetrectcap%
\pgfsetroundjoin%
\pgfsetlinewidth{0.501875pt}%
\definecolor{currentstroke}{rgb}{0.501961,0.501961,0.501961}%
\pgfsetstrokecolor{currentstroke}%
\pgfsetdash{}{0pt}%
\pgfpathmoveto{\pgfqpoint{6.455640in}{0.440000in}}%
\pgfpathlineto{\pgfqpoint{6.455640in}{3.520000in}}%
\pgfusepath{stroke}%
\end{pgfscope}%
\begin{pgfscope}%
\pgfpathrectangle{\pgfqpoint{1.000000in}{0.440000in}}{\pgfqpoint{6.200000in}{3.080000in}} %
\pgfusepath{clip}%
\pgfsetrectcap%
\pgfsetroundjoin%
\pgfsetlinewidth{0.501875pt}%
\definecolor{currentstroke}{rgb}{0.501961,0.501961,0.501961}%
\pgfsetstrokecolor{currentstroke}%
\pgfsetdash{}{0pt}%
\pgfpathmoveto{\pgfqpoint{6.610744in}{0.440000in}}%
\pgfpathlineto{\pgfqpoint{6.610744in}{3.520000in}}%
\pgfusepath{stroke}%
\end{pgfscope}%
\begin{pgfscope}%
\pgfpathrectangle{\pgfqpoint{1.000000in}{0.440000in}}{\pgfqpoint{6.200000in}{3.080000in}} %
\pgfusepath{clip}%
\pgfsetrectcap%
\pgfsetroundjoin%
\pgfsetlinewidth{0.501875pt}%
\definecolor{currentstroke}{rgb}{0.800000,0.000000,0.266667}%
\pgfsetstrokecolor{currentstroke}%
\pgfsetdash{}{0pt}%
\pgfpathmoveto{\pgfqpoint{6.765848in}{0.440000in}}%
\pgfpathlineto{\pgfqpoint{6.765848in}{3.520000in}}%
\pgfusepath{stroke}%
\end{pgfscope}%
\begin{pgfscope}%
\pgfpathrectangle{\pgfqpoint{1.000000in}{0.440000in}}{\pgfqpoint{6.200000in}{3.080000in}} %
\pgfusepath{clip}%
\pgfsetrectcap%
\pgfsetroundjoin%
\pgfsetlinewidth{1.505625pt}%
\definecolor{currentstroke}{rgb}{0.364706,0.517647,0.572549}%
\pgfsetstrokecolor{currentstroke}%
\pgfsetdash{}{0pt}%
\pgfpathmoveto{\pgfqpoint{1.281818in}{1.169484in}}%
\pgfpathlineto{\pgfqpoint{1.997789in}{1.169484in}}%
\pgfpathlineto{\pgfqpoint{2.013022in}{1.180018in}}%
\pgfpathlineto{\pgfqpoint{2.028256in}{1.213117in}}%
\pgfpathlineto{\pgfqpoint{2.043489in}{1.243508in}}%
\pgfpathlineto{\pgfqpoint{2.058722in}{1.249626in}}%
\pgfpathlineto{\pgfqpoint{2.073956in}{1.241922in}}%
\pgfpathlineto{\pgfqpoint{2.104423in}{1.216067in}}%
\pgfpathlineto{\pgfqpoint{2.119656in}{1.180239in}}%
\pgfpathlineto{\pgfqpoint{2.134889in}{1.096283in}}%
\pgfpathlineto{\pgfqpoint{2.150123in}{1.000565in}}%
\pgfpathlineto{\pgfqpoint{2.165356in}{0.981155in}}%
\pgfpathlineto{\pgfqpoint{2.180590in}{1.022198in}}%
\pgfpathlineto{\pgfqpoint{2.195823in}{1.051990in}}%
\pgfpathlineto{\pgfqpoint{2.211057in}{1.073655in}}%
\pgfpathlineto{\pgfqpoint{2.226290in}{1.117670in}}%
\pgfpathlineto{\pgfqpoint{2.241523in}{1.190799in}}%
\pgfpathlineto{\pgfqpoint{2.256757in}{1.283890in}}%
\pgfpathlineto{\pgfqpoint{2.287224in}{1.477771in}}%
\pgfpathlineto{\pgfqpoint{2.302457in}{1.553343in}}%
\pgfpathlineto{\pgfqpoint{2.317690in}{1.610443in}}%
\pgfpathlineto{\pgfqpoint{2.332924in}{1.660888in}}%
\pgfpathlineto{\pgfqpoint{2.348157in}{1.726165in}}%
\pgfpathlineto{\pgfqpoint{2.363391in}{1.855295in}}%
\pgfpathlineto{\pgfqpoint{2.378624in}{2.077817in}}%
\pgfpathlineto{\pgfqpoint{2.393857in}{2.280446in}}%
\pgfpathlineto{\pgfqpoint{2.409091in}{2.393129in}}%
\pgfpathlineto{\pgfqpoint{2.439558in}{2.525077in}}%
\pgfpathlineto{\pgfqpoint{2.454791in}{2.620606in}}%
\pgfpathlineto{\pgfqpoint{2.470025in}{2.727577in}}%
\pgfpathlineto{\pgfqpoint{2.485258in}{2.823790in}}%
\pgfpathlineto{\pgfqpoint{2.500491in}{2.910174in}}%
\pgfpathlineto{\pgfqpoint{2.530958in}{3.061822in}}%
\pgfpathlineto{\pgfqpoint{2.561425in}{3.243144in}}%
\pgfpathlineto{\pgfqpoint{2.576658in}{3.315514in}}%
\pgfpathlineto{\pgfqpoint{2.591892in}{3.372327in}}%
\pgfpathlineto{\pgfqpoint{2.607125in}{3.380000in}}%
\pgfpathlineto{\pgfqpoint{2.622359in}{3.283246in}}%
\pgfpathlineto{\pgfqpoint{2.637592in}{3.165543in}}%
\pgfpathlineto{\pgfqpoint{2.652826in}{3.121637in}}%
\pgfpathlineto{\pgfqpoint{2.698526in}{3.099111in}}%
\pgfpathlineto{\pgfqpoint{2.713759in}{3.086715in}}%
\pgfpathlineto{\pgfqpoint{2.774693in}{2.995830in}}%
\pgfpathlineto{\pgfqpoint{2.835627in}{2.850865in}}%
\pgfpathlineto{\pgfqpoint{2.896560in}{2.655827in}}%
\pgfpathlineto{\pgfqpoint{2.927027in}{2.538655in}}%
\pgfpathlineto{\pgfqpoint{2.972727in}{2.360070in}}%
\pgfpathlineto{\pgfqpoint{3.018428in}{2.189141in}}%
\pgfpathlineto{\pgfqpoint{3.033661in}{2.137035in}}%
\pgfpathlineto{\pgfqpoint{3.094595in}{1.969662in}}%
\pgfpathlineto{\pgfqpoint{3.155528in}{1.853076in}}%
\pgfpathlineto{\pgfqpoint{3.201229in}{1.801005in}}%
\pgfpathlineto{\pgfqpoint{3.216462in}{1.783924in}}%
\pgfpathlineto{\pgfqpoint{3.231695in}{1.775219in}}%
\pgfpathlineto{\pgfqpoint{3.277396in}{1.758339in}}%
\pgfpathlineto{\pgfqpoint{3.292629in}{1.722644in}}%
\pgfpathlineto{\pgfqpoint{3.307862in}{1.673207in}}%
\pgfpathlineto{\pgfqpoint{3.323096in}{1.630925in}}%
\pgfpathlineto{\pgfqpoint{3.338329in}{1.596906in}}%
\pgfpathlineto{\pgfqpoint{3.353563in}{1.570437in}}%
\pgfpathlineto{\pgfqpoint{3.384029in}{1.526751in}}%
\pgfpathlineto{\pgfqpoint{3.399263in}{1.498475in}}%
\pgfpathlineto{\pgfqpoint{3.429730in}{1.406330in}}%
\pgfpathlineto{\pgfqpoint{3.444963in}{1.392157in}}%
\pgfpathlineto{\pgfqpoint{3.460197in}{1.388183in}}%
\pgfpathlineto{\pgfqpoint{3.505897in}{1.337844in}}%
\pgfpathlineto{\pgfqpoint{3.536364in}{1.315695in}}%
\pgfpathlineto{\pgfqpoint{3.551597in}{1.306303in}}%
\pgfpathlineto{\pgfqpoint{3.566830in}{1.309460in}}%
\pgfpathlineto{\pgfqpoint{3.582064in}{1.356383in}}%
\pgfpathlineto{\pgfqpoint{3.597297in}{1.442296in}}%
\pgfpathlineto{\pgfqpoint{3.612531in}{1.498590in}}%
\pgfpathlineto{\pgfqpoint{3.627764in}{1.509395in}}%
\pgfpathlineto{\pgfqpoint{3.642998in}{1.504952in}}%
\pgfpathlineto{\pgfqpoint{3.658231in}{1.497308in}}%
\pgfpathlineto{\pgfqpoint{3.673464in}{1.491746in}}%
\pgfpathlineto{\pgfqpoint{3.703931in}{1.484544in}}%
\pgfpathlineto{\pgfqpoint{3.719165in}{1.485587in}}%
\pgfpathlineto{\pgfqpoint{3.734398in}{1.492379in}}%
\pgfpathlineto{\pgfqpoint{3.749631in}{1.497810in}}%
\pgfpathlineto{\pgfqpoint{3.764865in}{1.495599in}}%
\pgfpathlineto{\pgfqpoint{3.780098in}{1.488639in}}%
\pgfpathlineto{\pgfqpoint{3.795332in}{1.479567in}}%
\pgfpathlineto{\pgfqpoint{3.810565in}{1.462231in}}%
\pgfpathlineto{\pgfqpoint{3.825799in}{1.425528in}}%
\pgfpathlineto{\pgfqpoint{3.841032in}{1.367895in}}%
\pgfpathlineto{\pgfqpoint{3.856265in}{1.338152in}}%
\pgfpathlineto{\pgfqpoint{3.871499in}{1.346327in}}%
\pgfpathlineto{\pgfqpoint{3.886732in}{1.336390in}}%
\pgfpathlineto{\pgfqpoint{3.917199in}{1.267346in}}%
\pgfpathlineto{\pgfqpoint{3.947666in}{1.212521in}}%
\pgfpathlineto{\pgfqpoint{3.962899in}{1.178892in}}%
\pgfpathlineto{\pgfqpoint{3.993366in}{1.100211in}}%
\pgfpathlineto{\pgfqpoint{4.008600in}{1.060065in}}%
\pgfpathlineto{\pgfqpoint{4.023833in}{1.024709in}}%
\pgfpathlineto{\pgfqpoint{4.039066in}{0.999380in}}%
\pgfpathlineto{\pgfqpoint{4.054300in}{0.996954in}}%
\pgfpathlineto{\pgfqpoint{4.084767in}{1.071116in}}%
\pgfpathlineto{\pgfqpoint{4.100000in}{1.068886in}}%
\pgfpathlineto{\pgfqpoint{4.145700in}{1.001995in}}%
\pgfpathlineto{\pgfqpoint{4.160934in}{0.973546in}}%
\pgfpathlineto{\pgfqpoint{4.176167in}{0.947798in}}%
\pgfpathlineto{\pgfqpoint{4.191400in}{0.931365in}}%
\pgfpathlineto{\pgfqpoint{4.206634in}{0.924262in}}%
\pgfpathlineto{\pgfqpoint{4.221867in}{0.925338in}}%
\pgfpathlineto{\pgfqpoint{4.237101in}{0.938405in}}%
\pgfpathlineto{\pgfqpoint{4.252334in}{0.961615in}}%
\pgfpathlineto{\pgfqpoint{4.267568in}{0.991227in}}%
\pgfpathlineto{\pgfqpoint{4.282801in}{1.028989in}}%
\pgfpathlineto{\pgfqpoint{4.298034in}{1.062745in}}%
\pgfpathlineto{\pgfqpoint{4.313268in}{1.084881in}}%
\pgfpathlineto{\pgfqpoint{4.343735in}{1.222592in}}%
\pgfpathlineto{\pgfqpoint{4.358968in}{1.280546in}}%
\pgfpathlineto{\pgfqpoint{4.374201in}{1.343299in}}%
\pgfpathlineto{\pgfqpoint{4.404668in}{1.409018in}}%
\pgfpathlineto{\pgfqpoint{4.419902in}{1.447560in}}%
\pgfpathlineto{\pgfqpoint{4.435135in}{1.462780in}}%
\pgfpathlineto{\pgfqpoint{4.450369in}{1.459128in}}%
\pgfpathlineto{\pgfqpoint{4.465602in}{1.451235in}}%
\pgfpathlineto{\pgfqpoint{4.480835in}{1.446877in}}%
\pgfpathlineto{\pgfqpoint{4.572236in}{1.439168in}}%
\pgfpathlineto{\pgfqpoint{4.663636in}{1.416321in}}%
\pgfpathlineto{\pgfqpoint{4.678870in}{1.411657in}}%
\pgfpathlineto{\pgfqpoint{4.770270in}{1.373638in}}%
\pgfpathlineto{\pgfqpoint{4.800737in}{1.356787in}}%
\pgfpathlineto{\pgfqpoint{4.876904in}{1.312738in}}%
\pgfpathlineto{\pgfqpoint{5.059705in}{1.181455in}}%
\pgfpathlineto{\pgfqpoint{5.074939in}{1.170636in}}%
\pgfpathlineto{\pgfqpoint{5.166339in}{1.120895in}}%
\pgfpathlineto{\pgfqpoint{5.181572in}{1.113443in}}%
\pgfpathlineto{\pgfqpoint{5.272973in}{1.078524in}}%
\pgfpathlineto{\pgfqpoint{5.303440in}{1.070970in}}%
\pgfpathlineto{\pgfqpoint{5.379607in}{1.053974in}}%
\pgfpathlineto{\pgfqpoint{5.471007in}{1.047085in}}%
\pgfpathlineto{\pgfqpoint{5.486241in}{1.038835in}}%
\pgfpathlineto{\pgfqpoint{5.516708in}{1.017903in}}%
\pgfpathlineto{\pgfqpoint{5.547174in}{0.987236in}}%
\pgfpathlineto{\pgfqpoint{5.562408in}{0.972718in}}%
\pgfpathlineto{\pgfqpoint{5.592875in}{0.922519in}}%
\pgfpathlineto{\pgfqpoint{5.608108in}{0.905856in}}%
\pgfpathlineto{\pgfqpoint{5.623342in}{0.870672in}}%
\pgfpathlineto{\pgfqpoint{5.638575in}{0.823965in}}%
\pgfpathlineto{\pgfqpoint{5.653808in}{0.794213in}}%
\pgfpathlineto{\pgfqpoint{5.684275in}{0.761011in}}%
\pgfpathlineto{\pgfqpoint{5.714742in}{0.673842in}}%
\pgfpathlineto{\pgfqpoint{5.745209in}{0.632286in}}%
\pgfpathlineto{\pgfqpoint{5.760442in}{0.605600in}}%
\pgfpathlineto{\pgfqpoint{5.775676in}{0.591469in}}%
\pgfpathlineto{\pgfqpoint{5.790909in}{0.607326in}}%
\pgfpathlineto{\pgfqpoint{5.806143in}{0.626000in}}%
\pgfpathlineto{\pgfqpoint{5.821376in}{0.632378in}}%
\pgfpathlineto{\pgfqpoint{5.836609in}{0.640955in}}%
\pgfpathlineto{\pgfqpoint{5.851843in}{0.642687in}}%
\pgfpathlineto{\pgfqpoint{5.867076in}{0.627918in}}%
\pgfpathlineto{\pgfqpoint{5.897543in}{0.580000in}}%
\pgfpathlineto{\pgfqpoint{5.912776in}{0.588304in}}%
\pgfpathlineto{\pgfqpoint{5.928010in}{0.629685in}}%
\pgfpathlineto{\pgfqpoint{5.943243in}{0.631689in}}%
\pgfpathlineto{\pgfqpoint{5.958477in}{0.607162in}}%
\pgfpathlineto{\pgfqpoint{5.973710in}{0.608787in}}%
\pgfpathlineto{\pgfqpoint{5.988943in}{0.619218in}}%
\pgfpathlineto{\pgfqpoint{6.004177in}{0.619497in}}%
\pgfpathlineto{\pgfqpoint{6.019410in}{0.622254in}}%
\pgfpathlineto{\pgfqpoint{6.034644in}{0.631110in}}%
\pgfpathlineto{\pgfqpoint{6.065111in}{0.642581in}}%
\pgfpathlineto{\pgfqpoint{6.080344in}{0.658514in}}%
\pgfpathlineto{\pgfqpoint{6.095577in}{0.684377in}}%
\pgfpathlineto{\pgfqpoint{6.110811in}{0.722313in}}%
\pgfpathlineto{\pgfqpoint{6.126044in}{0.772044in}}%
\pgfpathlineto{\pgfqpoint{6.141278in}{0.812010in}}%
\pgfpathlineto{\pgfqpoint{6.156511in}{0.821943in}}%
\pgfpathlineto{\pgfqpoint{6.171744in}{0.811626in}}%
\pgfpathlineto{\pgfqpoint{6.186978in}{0.796743in}}%
\pgfpathlineto{\pgfqpoint{6.202211in}{0.789968in}}%
\pgfpathlineto{\pgfqpoint{6.918182in}{0.789968in}}%
\pgfpathlineto{\pgfqpoint{6.918182in}{0.789968in}}%
\pgfusepath{stroke}%
\end{pgfscope}%
\begin{pgfscope}%
\pgfpathrectangle{\pgfqpoint{1.000000in}{0.440000in}}{\pgfqpoint{6.200000in}{3.080000in}} %
\pgfusepath{clip}%
\pgfsetbuttcap%
\pgfsetroundjoin%
\definecolor{currentfill}{rgb}{0.364706,0.517647,0.572549}%
\pgfsetfillcolor{currentfill}%
\pgfsetlinewidth{1.003750pt}%
\definecolor{currentstroke}{rgb}{0.364706,0.517647,0.572549}%
\pgfsetstrokecolor{currentstroke}%
\pgfsetdash{}{0pt}%
\pgfsys@defobject{currentmarker}{\pgfqpoint{-0.027778in}{-0.027778in}}{\pgfqpoint{0.027778in}{0.027778in}}{%
\pgfpathmoveto{\pgfqpoint{0.000000in}{-0.027778in}}%
\pgfpathcurveto{\pgfqpoint{0.007367in}{-0.027778in}}{\pgfqpoint{0.014433in}{-0.024851in}}{\pgfqpoint{0.019642in}{-0.019642in}}%
\pgfpathcurveto{\pgfqpoint{0.024851in}{-0.014433in}}{\pgfqpoint{0.027778in}{-0.007367in}}{\pgfqpoint{0.027778in}{0.000000in}}%
\pgfpathcurveto{\pgfqpoint{0.027778in}{0.007367in}}{\pgfqpoint{0.024851in}{0.014433in}}{\pgfqpoint{0.019642in}{0.019642in}}%
\pgfpathcurveto{\pgfqpoint{0.014433in}{0.024851in}}{\pgfqpoint{0.007367in}{0.027778in}}{\pgfqpoint{0.000000in}{0.027778in}}%
\pgfpathcurveto{\pgfqpoint{-0.007367in}{0.027778in}}{\pgfqpoint{-0.014433in}{0.024851in}}{\pgfqpoint{-0.019642in}{0.019642in}}%
\pgfpathcurveto{\pgfqpoint{-0.024851in}{0.014433in}}{\pgfqpoint{-0.027778in}{0.007367in}}{\pgfqpoint{-0.027778in}{0.000000in}}%
\pgfpathcurveto{\pgfqpoint{-0.027778in}{-0.007367in}}{\pgfqpoint{-0.024851in}{-0.014433in}}{\pgfqpoint{-0.019642in}{-0.019642in}}%
\pgfpathcurveto{\pgfqpoint{-0.014433in}{-0.024851in}}{\pgfqpoint{-0.007367in}{-0.027778in}}{\pgfqpoint{0.000000in}{-0.027778in}}%
\pgfpathclose%
\pgfusepath{stroke,fill}%
}%
\begin{pgfscope}%
\pgfsys@transformshift{1.769287in}{1.169484in}%
\pgfsys@useobject{currentmarker}{}%
\end{pgfscope}%
\begin{pgfscope}%
\pgfsys@transformshift{1.967322in}{1.169484in}%
\pgfsys@useobject{currentmarker}{}%
\end{pgfscope}%
\begin{pgfscope}%
\pgfsys@transformshift{2.165356in}{1.000565in}%
\pgfsys@useobject{currentmarker}{}%
\end{pgfscope}%
\begin{pgfscope}%
\pgfsys@transformshift{2.317690in}{1.553343in}%
\pgfsys@useobject{currentmarker}{}%
\end{pgfscope}%
\begin{pgfscope}%
\pgfsys@transformshift{2.470025in}{2.620606in}%
\pgfsys@useobject{currentmarker}{}%
\end{pgfscope}%
\begin{pgfscope}%
\pgfsys@transformshift{2.622359in}{3.283246in}%
\pgfsys@useobject{currentmarker}{}%
\end{pgfscope}%
\begin{pgfscope}%
\pgfsys@transformshift{2.786880in}{2.995830in}%
\pgfsys@useobject{currentmarker}{}%
\end{pgfscope}%
\begin{pgfscope}%
\pgfsys@transformshift{2.951400in}{2.478578in}%
\pgfsys@useobject{currentmarker}{}%
\end{pgfscope}%
\begin{pgfscope}%
\pgfsys@transformshift{3.115921in}{1.940515in}%
\pgfsys@useobject{currentmarker}{}%
\end{pgfscope}%
\begin{pgfscope}%
\pgfsys@transformshift{3.280442in}{1.758339in}%
\pgfsys@useobject{currentmarker}{}%
\end{pgfscope}%
\begin{pgfscope}%
\pgfsys@transformshift{3.444963in}{1.392157in}%
\pgfsys@useobject{currentmarker}{}%
\end{pgfscope}%
\begin{pgfscope}%
\pgfsys@transformshift{3.597297in}{1.442296in}%
\pgfsys@useobject{currentmarker}{}%
\end{pgfscope}%
\begin{pgfscope}%
\pgfsys@transformshift{3.749631in}{1.497810in}%
\pgfsys@useobject{currentmarker}{}%
\end{pgfscope}%
\begin{pgfscope}%
\pgfsys@transformshift{3.901966in}{1.300541in}%
\pgfsys@useobject{currentmarker}{}%
\end{pgfscope}%
\begin{pgfscope}%
\pgfsys@transformshift{4.067357in}{0.996954in}%
\pgfsys@useobject{currentmarker}{}%
\end{pgfscope}%
\begin{pgfscope}%
\pgfsys@transformshift{4.232748in}{0.925338in}%
\pgfsys@useobject{currentmarker}{}%
\end{pgfscope}%
\begin{pgfscope}%
\pgfsys@transformshift{4.398140in}{1.376857in}%
\pgfsys@useobject{currentmarker}{}%
\end{pgfscope}%
\begin{pgfscope}%
\pgfsys@transformshift{4.563531in}{1.440453in}%
\pgfsys@useobject{currentmarker}{}%
\end{pgfscope}%
\begin{pgfscope}%
\pgfsys@transformshift{4.728922in}{1.392647in}%
\pgfsys@useobject{currentmarker}{}%
\end{pgfscope}%
\begin{pgfscope}%
\pgfsys@transformshift{4.894314in}{1.301694in}%
\pgfsys@useobject{currentmarker}{}%
\end{pgfscope}%
\begin{pgfscope}%
\pgfsys@transformshift{5.059705in}{1.181455in}%
\pgfsys@useobject{currentmarker}{}%
\end{pgfscope}%
\begin{pgfscope}%
\pgfsys@transformshift{5.214809in}{1.101803in}%
\pgfsys@useobject{currentmarker}{}%
\end{pgfscope}%
\begin{pgfscope}%
\pgfsys@transformshift{5.369913in}{1.057130in}%
\pgfsys@useobject{currentmarker}{}%
\end{pgfscope}%
\begin{pgfscope}%
\pgfsys@transformshift{5.525017in}{1.017903in}%
\pgfsys@useobject{currentmarker}{}%
\end{pgfscope}%
\begin{pgfscope}%
\pgfsys@transformshift{5.680121in}{0.777266in}%
\pgfsys@useobject{currentmarker}{}%
\end{pgfscope}%
\begin{pgfscope}%
\pgfsys@transformshift{5.835224in}{0.632378in}%
\pgfsys@useobject{currentmarker}{}%
\end{pgfscope}%
\begin{pgfscope}%
\pgfsys@transformshift{5.990328in}{0.619218in}%
\pgfsys@useobject{currentmarker}{}%
\end{pgfscope}%
\begin{pgfscope}%
\pgfsys@transformshift{6.145432in}{0.812010in}%
\pgfsys@useobject{currentmarker}{}%
\end{pgfscope}%
\begin{pgfscope}%
\pgfsys@transformshift{6.300536in}{0.789968in}%
\pgfsys@useobject{currentmarker}{}%
\end{pgfscope}%
\begin{pgfscope}%
\pgfsys@transformshift{6.455640in}{0.789968in}%
\pgfsys@useobject{currentmarker}{}%
\end{pgfscope}%
\begin{pgfscope}%
\pgfsys@transformshift{6.610744in}{0.789968in}%
\pgfsys@useobject{currentmarker}{}%
\end{pgfscope}%
\begin{pgfscope}%
\pgfsys@transformshift{6.765848in}{0.789968in}%
\pgfsys@useobject{currentmarker}{}%
\end{pgfscope}%
\end{pgfscope}%
\begin{pgfscope}%
\pgfsetrectcap%
\pgfsetmiterjoin%
\pgfsetlinewidth{0.803000pt}%
\definecolor{currentstroke}{rgb}{0.000000,0.000000,0.000000}%
\pgfsetstrokecolor{currentstroke}%
\pgfsetdash{}{0pt}%
\pgfpathmoveto{\pgfqpoint{1.000000in}{0.440000in}}%
\pgfpathlineto{\pgfqpoint{1.000000in}{3.520000in}}%
\pgfusepath{stroke}%
\end{pgfscope}%
\begin{pgfscope}%
\pgfsetrectcap%
\pgfsetmiterjoin%
\pgfsetlinewidth{0.803000pt}%
\definecolor{currentstroke}{rgb}{0.000000,0.000000,0.000000}%
\pgfsetstrokecolor{currentstroke}%
\pgfsetdash{}{0pt}%
\pgfpathmoveto{\pgfqpoint{7.200000in}{0.440000in}}%
\pgfpathlineto{\pgfqpoint{7.200000in}{3.520000in}}%
\pgfusepath{stroke}%
\end{pgfscope}%
\begin{pgfscope}%
\pgfsetrectcap%
\pgfsetmiterjoin%
\pgfsetlinewidth{0.803000pt}%
\definecolor{currentstroke}{rgb}{0.000000,0.000000,0.000000}%
\pgfsetstrokecolor{currentstroke}%
\pgfsetdash{}{0pt}%
\pgfpathmoveto{\pgfqpoint{1.000000in}{0.440000in}}%
\pgfpathlineto{\pgfqpoint{7.200000in}{0.440000in}}%
\pgfusepath{stroke}%
\end{pgfscope}%
\begin{pgfscope}%
\pgfsetrectcap%
\pgfsetmiterjoin%
\pgfsetlinewidth{0.803000pt}%
\definecolor{currentstroke}{rgb}{0.000000,0.000000,0.000000}%
\pgfsetstrokecolor{currentstroke}%
\pgfsetdash{}{0pt}%
\pgfpathmoveto{\pgfqpoint{1.000000in}{3.520000in}}%
\pgfpathlineto{\pgfqpoint{7.200000in}{3.520000in}}%
\pgfusepath{stroke}%
\end{pgfscope}%
\begin{pgfscope}%
\pgftext[x=1.835299in,y=0.580000in,left,base]{\rmfamily\fontsize{12.000000}{14.400000}\selectfont @}%
\end{pgfscope}%
\begin{pgfscope}%
\pgftext[x=2.241523in,y=0.580000in,left,base]{\rmfamily\fontsize{12.000000}{14.400000}\selectfont mid}%
\end{pgfscope}%
\begin{pgfscope}%
\pgftext[x=2.759459in,y=0.580000in,left,base]{\rmfamily\fontsize{12.000000}{14.400000}\selectfont suh}%
\end{pgfscope}%
\begin{pgfscope}%
\pgftext[x=3.521130in,y=0.580000in,left,base]{\rmfamily\fontsize{12.000000}{14.400000}\selectfont m@}%
\end{pgfscope}%
\begin{pgfscope}%
\pgftext[x=4.094922in,y=0.580000in,left,base]{\rmfamily\fontsize{12.000000}{14.400000}\selectfont naits}%
\end{pgfscope}%
\begin{pgfscope}%
\pgftext[x=5.344062in,y=0.580000in,left,base]{\rmfamily\fontsize{12.000000}{14.400000}\selectfont driim}%
\end{pgfscope}%
\begin{pgfscope}%
\pgfsetbuttcap%
\pgfsetmiterjoin%
\definecolor{currentfill}{rgb}{1.000000,1.000000,1.000000}%
\pgfsetfillcolor{currentfill}%
\pgfsetfillopacity{0.800000}%
\pgfsetlinewidth{1.003750pt}%
\definecolor{currentstroke}{rgb}{0.800000,0.800000,0.800000}%
\pgfsetstrokecolor{currentstroke}%
\pgfsetstrokeopacity{0.800000}%
\pgfsetdash{}{0pt}%
\pgfpathmoveto{\pgfqpoint{5.541322in}{3.021556in}}%
\pgfpathlineto{\pgfqpoint{7.102778in}{3.021556in}}%
\pgfpathquadraticcurveto{\pgfqpoint{7.130556in}{3.021556in}}{\pgfqpoint{7.130556in}{3.049334in}}%
\pgfpathlineto{\pgfqpoint{7.130556in}{3.422778in}}%
\pgfpathquadraticcurveto{\pgfqpoint{7.130556in}{3.450556in}}{\pgfqpoint{7.102778in}{3.450556in}}%
\pgfpathlineto{\pgfqpoint{5.541322in}{3.450556in}}%
\pgfpathquadraticcurveto{\pgfqpoint{5.513544in}{3.450556in}}{\pgfqpoint{5.513544in}{3.422778in}}%
\pgfpathlineto{\pgfqpoint{5.513544in}{3.049334in}}%
\pgfpathquadraticcurveto{\pgfqpoint{5.513544in}{3.021556in}}{\pgfqpoint{5.541322in}{3.021556in}}%
\pgfpathclose%
\pgfusepath{stroke,fill}%
\end{pgfscope}%
\begin{pgfscope}%
\pgfsetbuttcap%
\pgfsetmiterjoin%
\definecolor{currentfill}{rgb}{0.364706,0.517647,0.572549}%
\pgfsetfillcolor{currentfill}%
\pgfsetlinewidth{1.003750pt}%
\definecolor{currentstroke}{rgb}{0.364706,0.517647,0.572549}%
\pgfsetstrokecolor{currentstroke}%
\pgfsetdash{}{0pt}%
\pgfpathmoveto{\pgfqpoint{5.569100in}{3.297778in}}%
\pgfpathlineto{\pgfqpoint{5.846878in}{3.297778in}}%
\pgfpathlineto{\pgfqpoint{5.846878in}{3.395000in}}%
\pgfpathlineto{\pgfqpoint{5.569100in}{3.395000in}}%
\pgfpathclose%
\pgfusepath{stroke,fill}%
\end{pgfscope}%
\begin{pgfscope}%
\pgftext[x=5.957989in,y=3.297778in,left,base]{\rmfamily\fontsize{10.000000}{12.000000}\selectfont Interpolated \(\displaystyle F_0\)}%
\end{pgfscope}%
\begin{pgfscope}%
\pgfsetbuttcap%
\pgfsetmiterjoin%
\definecolor{currentfill}{rgb}{0.800000,0.000000,0.266667}%
\pgfsetfillcolor{currentfill}%
\pgfsetlinewidth{1.003750pt}%
\definecolor{currentstroke}{rgb}{0.800000,0.000000,0.266667}%
\pgfsetstrokecolor{currentstroke}%
\pgfsetdash{}{0pt}%
\pgfpathmoveto{\pgfqpoint{5.569100in}{3.104112in}}%
\pgfpathlineto{\pgfqpoint{5.846878in}{3.104112in}}%
\pgfpathlineto{\pgfqpoint{5.846878in}{3.201334in}}%
\pgfpathlineto{\pgfqpoint{5.569100in}{3.201334in}}%
\pgfpathclose%
\pgfusepath{stroke,fill}%
\end{pgfscope}%
\begin{pgfscope}%
\pgftext[x=5.957989in,y=3.104112in,left,base]{\rmfamily\fontsize{10.000000}{12.000000}\selectfont Syllable Boundary}%
\end{pgfscope}%
\end{pgfpicture}%
\makeatother%
\endgroup%
}
\caption[Adaptive sampling]{Location of extraction points in an \ac{F0} contour with an adaptive sampling rate.}
\label{fig:adapt-anchor}
\end{figure}

Given the various advantages and disadvantages of both approaches, in my methodology, I decided to adopt a hybrid of both.
This is implemented by first selecting an appropriate linguistic level for interval subdivision (e.g., the syllable level) and a time interval as the default sampling rate (e.g., 0.1~s).
The default sampling rate was determined heuristically, by trying out many different values and plotting the the original contours overlaid with vertical lines representing the sampling sites.
For each syllable, the sampling rate is adapted, based on the default sampling rate and the duration of the syllable, based on \autoref{dur-equation}.

\begin{equation}
sr_{s} = \frac{dur_{s}}{[{\frac{dur_{s}}{dsr}}]}
\label{dur-equation}
\end{equation}
where:
\begin{conditions}
 sr    &  sampling rate for syllable \textit{s} \\
 dur  &  duration of syllable \textit{s}\\
 dsr  &  default sampling rate
\end{conditions}


As we can observe in \autoref{fig:adapt-anchor}, the proposed adaptive sampling rate produces points of varying distance across syllables. 
However, less so than a purely anchor-point-driven approach. 
At the same time, we are able to allocate more sample points to longer syllables, which is the main advantage of the other approach, since we are not limited by a fixed number of anchor points. 


